\section{Source file splitting and organising}

It is appropriate to organize a book into parts, chapters, sections if
it has more than 300 pages. LaTeX has commands of part, chapter,
section, subsection and section. It is good to divide a file into
each section. Part and chapter will be directories. Subsection and
section will be described in the section file.

\begin{center}
\begin{tabular}{cc}
  \hline
  Part-section & Directory/File \\
  \hline
  Part & directory \\
  Chapter & directory \\
  Section & file \\
  Subsection & in a section file \\
  subsubsection & in a section file \\
  \hline
\end{tabular}
\end{center}

This can be expressed in the directory structure as follows, for
example.

\begin{verbatim}
--+-- part1 --+-- chap1 --+-- sec1
  |           |           +-- sec2
  |           +-- chap2 ----- sec1
  +-- part2 --+-- chap1 --+-- sec1
              |           +-- sec2
              +-- chap2 ----- sec1
\end{verbatim}

If a book has about 100 pages, it doesn't need any parts. In that case,
the structure becomes as follows.

\begin{verbatim}
--+-- chap1 --+-- sec1
  |           +-- sec2
  |           +-- sec3
  +-- chap1 --+-- sec1
              +-- sec2
              +-- sec3
\end{verbatim}

For smaller documents, section is enough.

\begin{verbatim}
--+-- sec1
  +-- sec2
  +-- sec3
  +-- sec4
  +-- sec5
\end{verbatim}

These structures depend on the size of the documents. In the above
figure, the file extension is omitted.

There are advantages and disadvantages to limiting the directory/file
names to part1, chap2 and sec1 (sec1.tex etc. if you add an extension).
The advantage is that the order of the files is clear and the program
can easily find the file. The disadvantage is that the contents of the
file cannot be known from the file name. In Buildtools, the former is
taken more important and the file are named to part, chap and sec.

The above files are called subfiles. Subfiles are included from the root
file. The program automatically makes these connections, so the user
does not have to write the {\textbackslash}input\{\} commands in the tex
source files.

Files other than the sec files can be placed anywhere, but it's a good
idea to put all the image files together in one directory.

\begin{verbatim}
--+-- part1 --+-- chap1 --+-- sec1
  |           |           +-- sec2
  |           +-- chap2 ----- sec1
  +-- part2 --+-- chap1 --+-- sec1
  +           |           +-- sec2
  +           +-- chap2 ----- sec1
  +-- image --+-- photo.jpg
              +-- diagram.png
              +-- sample.png
\end{verbatim}

When including these files from the body files, refer to them by the
relative path from the top directory.

\begin{verbatim}
\includegraphics{image/photo.jpg}
\end{verbatim}

This saves you from having to change the command pathnames if the body
file moves across chaps and parts. The program will automatically
convert the pathnames for {\textbackslash}input and
{\textbackslash}incudegraphics into absolute paths.
