
\section{Typeset}

Compiling source files with lualatex is called a ``typeset''. The
typeset requires lualatex to be run more than once for cross-references.
There is a program called latexmk that manages it. Buildtools uses
latexmk. Thanks to that, the user does not have to run lualatex multiple
times. Just type:

\begin{verbatim}
$ rake
\end{verbatim}

After that, the typeset is done automatically.

Typeset take time for large documents. It's a waste of time to typeset
an entire document to see minor corrections. In that case, use the
part\_typeset command. This is a program that typesets only one subfile.
However, section and cross-reference numbers are not reflected
correctly. It is for confirmation only.

\begin{verbatim}
$ part_typeset 1-1-1
\end{verbatim}

This command typessets only the file part1/chap1/sec1. If there is no
part and only chap and sec, for example, ``part\_typeset 1-2'' will
typeset the chap1/sec2 file. Typesetting only one file in this way is
called a partial typeset.

There is a directory `example', where you can try the rake command.

The typeset produces various intermediate files. These are stored in the
\_build directory. When you want to delete the intermediate files, type
as follows.

\begin{verbatim}
$ rake clean
\end{verbatim}

If you want to delete the target pdf file as well, then type:

\begin{verbatim}
4 rake clobber
\end{verbatim}
