\section{Initialization}

If you want to create a new document, use newtex. Newtex makes a new
directory and initializes it. The directory contains templates.

\begin{verbatim}
$ ls
Rakefile  cover.tex  gecko.png  helper.tex  main.tex
\end{verbatim}

In most cases, the Rakefile works. It rarely needs to be rewritten.

\begin{itemize}
\item
  cover.tex is the source file of a cover page. This file is not needed
  if you use the {\textbackslash}maketitile command.
\item
  gecko.png is an image file of a gecko used in cover.tex. You can also
  modify cover.tex to use another image file.
\item
  helper.tex is part of the preamble. It mainly includes packages and
  define macros ({\textbackslash}newcommand etc).
\item
  main.tex is the root file. You don't have to write the
  {\textbackslash}input\{\} commands to i subfiles. Importing is done
  automatically.
\end{itemize}

If you want to create your own converter, put a file named
``converter.rb'' in this directory.

If your document is large, create a part-chap-sec directories/files, and
if it is small, write only sec files in the top directory.
