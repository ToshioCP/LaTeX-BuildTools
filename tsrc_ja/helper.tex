% helper.tex

% 注意
% graphicxのドライバーがluatexになっている。他のドライバーを使う場合は書き換えが必要。

% パッケージのとりこみ
\usepackage{amsmath,amssymb}
\usepackage[luatex]{graphicx}
\usepackage{tikz}
\usetikzlibrary{calc}
\usetikzlibrary{topaths}
\usetikzlibrary{plotmarks}
\usetikzlibrary{intersections}
\usetikzlibrary{arrows,decorations.pathmorphing,backgrounds,positioning,fit,petri}
\usetikzlibrary{arrows.meta}
\usetikzlibrary{3d}
%\usepackage{gnuplot-lua-tikz}

\usepackage{fancybox}
\usepackage{booktabs}

\usepackage[margin=2.4cm]{geometry}

\usepackage[colorlinks=true,linkcolor=black,pdfencoding=auto]{hyperref}

% マクロのサンプル(解答、証明、q.e.d.、解答)
%\newcommand{\solution}{\begin{flushleft}\textbf{解:}\end{flushleft}}
%\newcommand{\proof}{\begin{flushleft}\textbf{証明}\end{flushleft}}
%\newcommand{\qed}{\begin{flushright}\textbf{証明終}\end{flushright}}

% 定理環境(定理、補題、系、定義、例、練習問題)
\newtheorem{theorem}{定理}[section]
\newtheorem{lemma}{補題}[section]
\newtheorem{corollary}{系}[section]
\newtheorem{definition}{定義}[section]
\newtheorem{example}{例}[section]
\newtheorem{exercise}{問題}[section]

% 凹凸増減表の矢印 ----------------------------------------------------------------------------
% concave north east arrow 上に凸で増加
\newcommand{\ccnearrow}{
\begin{tikzpicture}
  \draw[very thin,->] (0,0) .. controls (0,0.2) and (0.05,0.25) .. (0.25,0.25);
\end{tikzpicture}
}
% concave south east arrow 上に凸で減少
\newcommand{\ccsearrow}{
\begin{tikzpicture}
  \draw[very thin,->] (0,0) .. controls (0.2,0) and (0.25,-0.05) .. (0.25,-0.25);
\end{tikzpicture}
}
% convex north east arrow 下に凸で増加
\newcommand{\cvnearrow}{
\begin{tikzpicture}
  \draw[very thin,->] (0,0) .. controls (0.2,0) and (0.25,0.05) .. (0.25,0.25);
\end{tikzpicture}
}
% convex south east arrow 下に凸で減少
\newcommand{\cvsearrow}{
\begin{tikzpicture}
  \draw[very thin,->] (0,0) .. controls (0,-0.2) and (0.05,-0.25) .. (0.25,-0.25);
\end{tikzpicture}
}
%-----------------------------------------------------------------------------------------------

% for long division (割り算の筆算のためのマクロ、Tikzの中で使う)=======================================================
% \divisoroffset is the distance between x-axis 0 and the center of the divisor. It is negative number.
\newcommand{\divisor}[2]{\node[anchor=base] at (#1,0) {#2}}
% dividendoffset is the offset from x-axis 0 to the first term bound of the dividend.
\newcommand{\dividendoffset}{0.8}
% \termwidth is the width between the consecutive term.
\newcommand{\termwidth}{0.9}
% expressionheight is the height of each expression.
\newcommand{\expressionheight}{0.5}
\newcommand{\pt}[3]{\node[base left] at (\dividendoffset+\termwidth*#1,\expressionheight*#2) {#3}}
\newcommand{\hseparator}[3]{\draw ({\dividendoffset+(#1-1)*\termwidth},{-(#3+0.3)*\expressionheight})
                            -- ({\dividendoffset+(#2)*\termwidth},{-(#3+0.3)*\expressionheight})}
\newcommand{\divisionbox}[1]{\draw (0,\expressionheight*0.7) -- (\dividendoffset+\termwidth*#1,\expressionheight*0.7)
                                   (0,\expressionheight*0.7) arc[start angle=27,end angle=-27,radius=0.5]}
%=========================================================================================================================
