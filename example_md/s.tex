% Options for packages loaded elsewhere
\PassOptionsToPackage{unicode}{hyperref}
\PassOptionsToPackage{hyphens}{url}
%
\documentclass[
]{article}
\usepackage{lmodern}
\usepackage{amssymb,amsmath}
\usepackage{ifxetex,ifluatex}
\ifnum 0\ifxetex 1\fi\ifluatex 1\fi=0 % if pdftex
  \usepackage[T1]{fontenc}
  \usepackage[utf8]{inputenc}
  \usepackage{textcomp} % provide euro and other symbols
\else % if luatex or xetex
  \usepackage{unicode-math}
  \defaultfontfeatures{Scale=MatchLowercase}
  \defaultfontfeatures[\rmfamily]{Ligatures=TeX,Scale=1}
\fi
% Use upquote if available, for straight quotes in verbatim environments
\IfFileExists{upquote.sty}{\usepackage{upquote}}{}
\IfFileExists{microtype.sty}{% use microtype if available
  \usepackage[]{microtype}
  \UseMicrotypeSet[protrusion]{basicmath} % disable protrusion for tt fonts
}{}
\makeatletter
\@ifundefined{KOMAClassName}{% if non-KOMA class
  \IfFileExists{parskip.sty}{%
    \usepackage{parskip}
  }{% else
    \setlength{\parindent}{0pt}
    \setlength{\parskip}{6pt plus 2pt minus 1pt}}
}{% if KOMA class
  \KOMAoptions{parskip=half}}
\makeatother
\usepackage{xcolor}
\IfFileExists{xurl.sty}{\usepackage{xurl}}{} % add URL line breaks if available
\IfFileExists{bookmark.sty}{\usepackage{bookmark}}{\usepackage{hyperref}}
\hypersetup{
  hidelinks,
  pdfcreator={LaTeX via pandoc}}
\urlstyle{same} % disable monospaced font for URLs
\usepackage{longtable,booktabs}
% Correct order of tables after \paragraph or \subparagraph
\usepackage{etoolbox}
\makeatletter
\patchcmd\longtable{\par}{\if@noskipsec\mbox{}\fi\par}{}{}
\makeatother
% Allow footnotes in longtable head/foot
\IfFileExists{footnotehyper.sty}{\usepackage{footnotehyper}}{\usepackage{footnote}}
\makesavenoteenv{longtable}
\setlength{\emergencystretch}{3em} % prevent overfull lines
\providecommand{\tightlist}{%
  \setlength{\itemsep}{0pt}\setlength{\parskip}{0pt}}
\setcounter{secnumdepth}{-\maxdimen} % remove section numbering

\author{}
\date{}

\begin{document}

日本語の``Readme.ja.md''があります。

\hypertarget{latex-buildtools}{%
\section{Latex-Buildtools}\label{latex-buildtools}}

The current version is written in ruby. Old bash version has been moved
to the bash branch.

\hypertarget{background}{%
\subsubsection{Background}\label{background}}

It gets more difficult to create and maintain a document if it becomes
larger. The same goes to LaTeX. The followings are some solutions to
avoid such difficulties.

\begin{itemize}
\tightlist
\item
  Splitting and organizing source files
\item
  individual typesetting
\item
  Preprocessing
\end{itemize}

Buildtools is a set of tools that supports them.

\hypertarget{source-file-splitting-and-organising}{%
\subsubsection{Source file splitting and
organising}\label{source-file-splitting-and-organising}}

It is appropriate to organize a book into parts, chapters, sections if
it has more than 300 pages. LaTeX has commands of part, chapter,
section, subsection and subsubsection. It is good to divide a file into
each section. Part and chapter will be directories. Subsection and
subsubsection will be described in the section file.

\begin{longtable}[]{@{}cc@{}}
\toprule
Part-Subsubsection & Directory / File\tabularnewline
\midrule
\endhead
Part & directory\tabularnewline
Chapter & directory\tabularnewline
Section & file\tabularnewline
Subsection & in a section file\tabularnewline
Subsubsection & in a section file\tabularnewline
\bottomrule
\end{longtable}

This can be expressed in the directory structure as follows, for
example.

\begin{verbatim}
--+-- part1 --+-- chap1 --+-- sec1
  |           |           +-- sec2
  |           +-- chap2 ----- sec1
  +-- part2 --+-- chap1 --+-- sec1
              |           +-- sec2
              +-- chap2 ----- sec1
\end{verbatim}

If a book has about 100 pages, it doesn't need any parts. In that case,
the structure becomes as follows.

\begin{verbatim}
--+-- chap1 --+-- sec1
  |           +-- sec2
  |           +-- sec3
  +-- chap1 --+-- sec1
              +-- sec2
              +-- sec3
\end{verbatim}

For smaller documents, section is enough.

\begin{verbatim}
--+-- sec1
  +-- sec2
  +-- sec3
  +-- sec4
  +-- sec5
\end{verbatim}

These structures depend on the size of the documents. In the above
figure, the file extension is omitted.

There are advantages and disadvantages to limiting the directory/file
names to part1, chap2 and sec1 (sec1.tex etc. if you add an extension).
The advantage is that the order of the files is clear and the program
can easily find the file. The disadvantage is that the contents of the
file cannot be known from the file name. In Buildtools, the former is
taken more important and the file are named to part, chap and sec.

The above files are called subfiles. Subfiles are included from the root
file. The program automatically makes these connections, so the user
does not have to write the \textbackslash input\{\} commands in the tex
source files.

Files other than the sec files can be placed anywhere, but it's a good
idea to put all the image files together in one directory.

\begin{verbatim}
--+-- part1 --+-- chap1 --+-- sec1
  |           |           +-- sec2
  |           +-- chap2 ----- sec1
  +-- part2 --+-- chap1 --+-- sec1
  +           |           +-- sec2
  +           +-- chap2 ----- sec1
  +-- image --+-- photo.jpg
              +-- diagram.png
              +-- sample.png
\end{verbatim}

When including these files from the body files, refer to them by the
relative path from the top directory.

\begin{verbatim}
\includegraphics{image/photo.jpg}
\end{verbatim}

This saves you from having to change the command pathnames if the body
file moves across chaps and parts. The program will automatically
convert the pathnames for \textbackslash input and
\textbackslash incudegraphics into absolute paths.

\hypertarget{typeset}{%
\subsubsection{Typeset}\label{typeset}}

Compiling source files with lualatex is called a ``typeset''. The
typeset requires lualatex to be run more than once for cross-references.
There is a program called latexmk that manages it. Buildtools uses
latexmk. Thanks to that, the user does not have to run lualatex multiple
times. Just type:

\begin{verbatim}
$ rake
\end{verbatim}

After that, the typeset is done automatically.

Typeset take time for large documents. It's a waste of time to typeset
an entire document to see minor corrections. In that case, use the
part\_typeset command. This is a program that typesets only one subfile.
However, section and cross-reference numbers are not reflected
correctly. It is for confirmation only.

\begin{verbatim}
$ part_typeset 1-1-1
\end{verbatim}

This command typessets only the file part1/chap1/sec1. If there is no
part and only chap and sec, for example, ``part\_typeset 1-2'' will
typeset the chap1/sec2 file. Typesetting only one file in this way is
called a partial typeset.

There is a directory `example', where you can try the rake command.

The typeset produces various intermediate files. These are stored in the
\_build directory. When you want to delete the intermediate files, type
as follows.

\begin{verbatim}
$ rake clean
\end{verbatim}

If you want to delete the target pdf file as well, then type:

\begin{verbatim}
4 rake clobber
\end{verbatim}

\hypertarget{preprocessing}{%
\subsubsection{Preprocessing}\label{preprocessing}}

Preprocessing is executed before the typeset. For example, converting a
markdown file to a LaTeX file is preprocessing.

\begin{verbatim}
sec1.md == (pandoc) ==> sec1.tex == (lualatex) ==> XXX.pdf
\end{verbatim}

Pandoc is a program that converts one document to another type
document..

\begin{verbatim}
https://pandoc.org/
\end{verbatim}

Since the preprocessing is done automatically, the only thing users need
to do is putting the extension to the source file. For example, put
`.md' to a markdown file sec1.

The user can add their own preprocessing program. For example, suppose
that you want to add a new extension ``src.tex'' and transfer
``m(1,2,3,4)'' into bmatrix environment.

\begin{verbatim}
\begin{bmatrix}1&2\\3&4\end{bmatrix}
\end{verbatim}

You can do this with ruby's gsub or gsub! methods. The preprocessing
created by the user is described in a file called ``converter.rb'' and
placed in the top directory.

\begin{verbatim}
{ :'src.tex' => lambda do |src, dst|
    buf = File.read(src)
    buf = buf.split(/(\\begin\{verbatim\}.*?\\end\{verbatim\}\n)/m)
    buf = buf.map do |s|
      s.match?(/^\\begin\{verbatim\}/) ? s : s.split(/(%.*$)/)
    end
    buf = buf.flatten
    buf.each do |s|
      unless s.match?(/^\\begin\{verbatim\}/) || s.match?(/^%/)
        s.gsub!(/m\((\d),(\d),(\d),(\d)\)/) {"\\begin{bmatrix}#{$1}&#{$2}\\\\#{$3}&#{$4}\\end{bmatrix}"}
      end
    end
    File.write(dst, buf.join)
  end }
\end{verbatim}

From the 3rd line to the 7th line, the verbatim environment, the comment
part and the other parts are separated. In the 9th line, lines without
verbatim and comments are selected. The conversion is performed in the
10th line. Various conversions are possible by customizing the 10th
line. Note that the conversion will not be reflected if gsub is used
instead of gsub! (With an exclamation mark).

The contents of the converter.rb are executed without being checked by
the program. If there is an error in it, it may cause a serious damage.
Therefore, it is recommended to backup the files in advance. Also, it is
very dangerous to execute a program created by another person without
checking its contents. Make a converter after you fully understand these
risks.

\hypertarget{initialization}{%
\subsubsection{Initialization}\label{initialization}}

If you want to create a new document, use newtex. Newtex makes a new
directory and initializes it. The directory contains templates.

\begin{verbatim}
$ ls
Rakefile  cover.tex  gecko.png  helper.tex  main.tex
\end{verbatim}

In most cases, the Rakefile works. It rarely needs to be rewritten.

\begin{itemize}
\tightlist
\item
  cover.tex is the source file of a cover page. This file is not needed
  if you use the \textbackslash maketitile command.
\item
  gecko.png is an image file of a gecko used in cover.tex. You can also
  modify cover.tex to use another image file.
\item
  helper.tex is part of the preamble. It mainly includes packages and
  define macros (\textbackslash newcommand etc).
\item
  main.tex is the root file. You don't have to write the
  \textbackslash input\{\} commands to i subfiles. Importing is done
  automatically.
\end{itemize}

If you want to create your own converter, put a file named
``converter.rb'' in this directory.

If your document is large, create a part-chap-sec directories/files, and
if it is small, write only sec files in the top directory.

\hypertarget{insert-and-renumber-files}{%
\subsubsection{Insert and renumber
files}\label{insert-and-renumber-files}}

If you want to put a new sec file between sec1 and sec2, create a file
with any number between 1 and 2. For example, name the file ``sec1.5''.
You can leave it as it is, but if you want to change the file numbers to
1,2,3,4 \ldots, use the renumber command.

\begin{verbatim}
$ renumber
\end{verbatim}

This makes:

\begin{verbatim}
sec1.tex => sec1.tex
sec1.5.tex => sec2.tex
sec2.tex => sec3.tex
\end{verbatim}

\hypertarget{archive}{%
\subsubsection{Archive}\label{archive}}

The document can be archived after you finish your work.

\begin{verbatim}
$ arch_tex -b
\end{verbatim}

This will archive all documents and Rakefile, lib\_latex\_utils.rb,
converter.rb into a bzip2 file. At this time, readme.md, which is a
brief explanation file, is also included. People who get the archive do
not need Buildtools. All they need is a program to unzip the archive and
ruby \hspace{0pt}\hspace{0pt}environment.

You can choose a compression converter by giving an option to arch\_tex.
Three compression types are supprted.

\begin{itemize}
\tightlist
\item
  -b: bzip2
\item
  -g: gzip
\item
  -z: zip
\end{itemize}

\hypertarget{install-and-uninstall}{%
\subsubsection{Install and uninstall}\label{install-and-uninstall}}

You need ruby \hspace{0pt}\hspace{0pt}and tex environment. Ruby 2.7 is
used to develop Buildtools, but it may work even if the ruby version is
\hspace{0pt}\hspace{0pt}2.0 or later. Texlive is recommended for the tex
environment.

The TeX engine is lualatex. If you want to use another engine, Modify
Rakefile. Only a small modification is enough.

Linux was the OS to develop Buildtools, but I think that it will also
work on Windows.

To install, tyoe as follows for Linux

\begin{verbatim}
$ sudo ruby ​​install.rb
\end{verbatim}

The executable file is set in /usr/local/bin and the library
(lib\_latex\_utils) is set in \$LOAD\_PATH{[}0{]}. The variable
\$LOAD\_PATH{[}0{]} points /usr/local/lib/site\_ruby/2.7.0 etc usually.
But it depends on the system. If you have used rbenv to install ruby,
the library destination is under the .rbenv directory of your home
directory.

At the time of installation, newtex and arch\_tex are rewritten.

Uninstall:

\begin{verbatim}
$ sudo ruby ​​install.rb-
\end{verbatim}

\hypertarget{licence}{%
\subsubsection{licence}\label{licence}}

Copyright (C) 2022 ToshioCP (Toshio Sekiya)

Buildtools is free software; you can redistribute it and/or modify it
under the terms of the GNU General Public License as published by the
Free Software Foundation; either version 3 of the License, or (at your
option) any later version.

Buildtools is distributed in the hope that it will be useful, but
WITHOUT ANY WARRANTY; without even the implied warranty of
MERCHANTABILITY or FITNESS FOR A PARTICULAR PURPOSE. See the
\href{https://www.gnu.org/licenses/gpl-3.0.html}{GNU General Public
License} for more details.

\end{document}
