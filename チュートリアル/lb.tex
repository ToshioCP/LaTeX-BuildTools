\subsection{lbの簡単な使い方}
lbはBuildtoolsの主要なスクリプトである。
このセクションでは、短いTeXファイルを例に用いてその使い方を説明する。

はじめに\verb|example|という名前のディレクトリを作成し、カレント・ディレクトリをそのディレクトリに移動する。
\begin{verbatim}
$ makedir example
$ cd example
\end{verbatim}

そして、そのディレクトリにTeXソースファイルを作成する。
エディタを使い、下記の内容をコピーして、\verb|main.tex|という名前で保存する。
\begin{verbatim}
\documentclass{article}
\begin{document}
Hello \LaTeX !!
\end{document}
\end{verbatim}

それが済んだら、端末から\verb|lb|とタイプすれば良い。
\begin{verbatim}
$ lb
\end{verbatim}
すると、latexmkとpdflatexが起動され、\verb|main.tex|がコンパイルされる。
メッセージが表示され、コンパイルのプロセスが示される。
その中に次のようなメッセージがあれば、コンパイルは正常に終了している。
\begin{verbatim}
Output written on _build/main.pdf (1 page, 19263 bytes).
\end{verbatim}
ディレクトリをチェックしてみよう。
\begin{verbatim}
$ ls -l
total 8
drwxrwxr-x 2 user user 4096 Dec  6 11:59 _build
-rw-rw-r-- 1 user user   72 Dec  6 11:59 main.tex
 ... ...
\end{verbatim}
新しいディレクトリ\verb|_build|が作られている。
そのディレクトリに含まれるファイルを調べてみよう。
\begin{verbatim}
$ cd _build
$ ls
\end{verbatim}
そこには補助ファイルと、ターゲット・ファイル(コンパイルで最終的に生成されたファイル)である\verb|main.pdf|がある。
evinceなどのpdfビューワで\verb|main.pdf|を見てみよう。
\begin{verbatim}
$ evince main.pdf
\end{verbatim}
\begin{center}
\includegraphics[width=3cm]{hellolatex.png}
\end{center}

\subsection{lb.confの使い方}
前のサブセクションではlbはpdflatexを起動した。
lbがpdflatexを選択した理由は、ドキュメントクラスが「article」だったからである。
articleのコンパイルにはlualatexやxelatexも可能だが、長い間pdflatexがスタンダードであった。

例えば、lualatexでコンパイルしたいというときには、\verb|lb.conf|にそれを記述しなければならない。
この初期化ファイルには6つの項目がある。
\begin{description}
\item[rootfile] ルートファイルとは、メインのlatexファイルである。それは通常、{\textbackslash}begin\{document\}と{\textbackslash}end\{document\}を含んでいる。他のファイルはサブファイルという。
\item[builddir] ビルド・ディレクトリの名前を指定する。それは一時的なディレクトリともいわれ、すべての補助ファイルとターゲット・ファイル(通常はpdf)が置かれる。
\item[engine] これはlatexエンジン(ソースファイルをコンパイルするプログラム)を指定する。pdflatex、xelatex、lualatex、latex、platexを指定することができる。
\item[latex\_option] これは\verb|latexmk|に与えるオプションである。lb.confが無くてもlbは`-halt-on-error'のオプションを自動的に与える。オプションはエンジンにも与えられる。
\item[dvipdf] これはdviファイルをpdfファイルに変換するコンバータである。これはlatexまたはplatexがエンジンに指定されたときのみ必要であり、他のエンジンでは必要ない。dvipdfmxが最も良いとされている。
\item[preview] Pdfビューワである。lbが引数にサブファイルを与えられたときにコンパイル後にこれを起動する。
\end{description}

エディタに下記の内容を打ち込み\verb|lb.conf|という名前で保存する。
\begin{verbatim}
rootfile=main
builddir=_build
engine=lualatex
latex_option=-halt-on-error
dvipdf=
preview=evince
\end{verbatim}
保存したら、lbとタイプしてみよう。
\begin{verbatim}
$ lb
\end{verbatim}
すると、今度はlualatexを使ってコンパイルが実行されている。

もしも、ソースファイルの名前を`example.tex'とする場合には、lb.confの1行目を変更する。
\begin{verbatim}
rootfile=example
\end{verbatim}
または
\begin{verbatim}
rootfile=example.tex
\end{verbatim}
このように、拡張子は省略できる。

更に、すべての補助ファイルとターゲット・ファイルをソースディレクトリに起きたいときは2行目を変更する。
\begin{verbatim}
builddir=
\end{verbatim}
これでbuilddirには空文字列が指定される。それはビルド・ディレクトリを作成しない、ということを意味する。

\verb|lb|を下記の\verb|lb.conf|の下で実行してみよう。
\begin{verbatim}
rootfile=example
builddir=
engine=latex
latex_option=-halt-on-error
dvipdf=dvipdfmx
preview=evince
\end{verbatim}
ここでは、エンジンはlatexでdvipdfコンバータはdvipdfmxになっている。
\begin{verbatim}
$ rm -r _build
$ mv main.tex example.tex
$ lb
\end{verbatim}
次のような内容を含むメッセージが表示される。
\begin{verbatim}
This is pdfTeX, Version 3.14159265-2.6-1.40.21 (TeX Live 2020)
 (preloaded format=latex)
  ... ...
  ... ...
Output written on example.dvi (1 page, 332 bytes).
Transcript written on example.log.
Latexmk: Examining 'example.log'
=== TeX engine is 'pdfTeX'
Latexmk: Log file says output to 'example.dvi'
Latexmk: All targets (example.dvi) are up-to-date
example.dvi -> example.pdf
[1]
3662 bytes written
\end{verbatim}
これはエンジンがlatex%
\footnote{
Texlive2020では、latexコマンドは、オリジナルのtexではなく、pdftexをその代わりに用いている。
}
であることを示している。
そして、pdfではなく、dviファイルが生成される。
その後、dvipdfmxがlatexmkに呼び出され、dviファイルはpdfファイルに変換される。
メッセージにはdvipdfmxの文字は現れないが、`example.dvi -{\textgreater} example.pdf'はdvipdfmxによって出力されたものである。
したがって、ビルドの間にlatexmkによってdvipdfmxが呼び出されたことが確認できる。
\begin{verbatim}
$ ls
example.aux  example.fdb_latexmk  example.log  example.tex
example.dvi  example.fls          example.pdf  lb.conf
\end{verbatim}
ディレクトリのリストから、\_buildのような一時ディレクトリが生成されていないことがわかる。
これはlb.confの中でbuilddirに空文字列を指定したことによるものである。

lbの重要な特長にサブファイルを単独でコンパイルできることがある。
これは、後ほどセクション\ref{sec:testcompile} テスト・コンパイル (p. \pageref{sec:testcompile})で説明される。
