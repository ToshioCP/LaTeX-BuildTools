% Helper.tex is a tex file included into the preamble in the rootfile.
% This file contails mainly \usepackage commands and \newcommand commands.
% You need to rewrite this file to fit your tex file.

% Packages
\usepackage{amsmath,amssymb}
\usepackage[luatex]{graphicx}
\usepackage{tikz}
%\usetikzlibrary{calc}

%\usepackage{fancybox}
\usepackage{booktabs}

%\usepackage[margin=2.4cm]{geometry}

\usepackage[colorlinks=true,linkcolor=black]{hyperref}

% Sample macro for English
%\newcommand{\solution}{\begin{flushleft}\textbf{Solution:}\end{flushleft}}
%\newcommand{\proof}{\begin{flushleft}\textbf{Proof}\end{flushleft}}
%\newcommand{\qed}{\begin{flushright}\textbf{q. e. d.}\end{flushright}}
%\newcommand{\answer}[1]{\begin{flushleft}\textbf{Exercise~\ref{#1}}\end{flushleft}}

\providecommand{\tightlist}{%
  \setlength{\itemsep}{0pt}\setlength{\parskip}{0pt}}

% Theorem environment.
%\newtheorem{theorem}{Theorem}[section]
%\newtheorem{lemma}{Lemma}[section]
%\newtheorem{corollary}{Corollary}[section]
%\newtheorem{definition}{Definition}[section]
%\newtheorem{example}{Example}[section]
%\newtheorem{exercise}{Exercise}[section]

% Sample macro for Japanese
% マクロのサンプル(解答、証明、q.e.d.、解答)
%\newcommand{\solution}{\begin{flushleft}\textbf{解:}\end{flushleft}}
%\newcommand{\proof}{\begin{flushleft}\textbf{証明}\end{flushleft}}
%\newcommand{\qed}{\begin{flushright}\textbf{証明終}\end{flushright}}

% 定理環境(定理、補題、系、定義、例、練習問題)
%\newtheorem{theorem}{定理}[section]
%\newtheorem{lemma}{補題}[section]
%\newtheorem{corollary}{系}[section]
%\newtheorem{definition}{定義}[section]
%\newtheorem{example}{例}[section]
%\newtheorem{exercise}{問題}[section]

% 凹凸増減表の矢印 ----------------------------------------------------------------------------
% concave north east arrow 上に凸で増加
%\newcommand{\ccnearrow}{
%\begin{tikzpicture}
%  \draw[very thin,->] (0,0) .. controls (0,0.2) and (0.05,0.25) .. (0.25,0.25);
%\end{tikzpicture}
%}
% concave south east arrow 上に凸で減少
%\newcommand{\ccsearrow}{
%\begin{tikzpicture}
%  \draw[very thin,->] (0,0) .. controls (0.2,0) and (0.25,-0.05) .. (0.25,-0.25);
%\end{tikzpicture}
%}
% convex north east arrow 下に凸で増加
%\newcommand{\cvnearrow}{
%\begin{tikzpicture}
%  \draw[very thin,->] (0,0) .. controls (0.2,0) and (0.25,0.05) .. (0.25,0.25);
%\end{tikzpicture}
%}
% convex south east arrow 下に凸で減少
%\newcommand{\cvsearrow}{
%\begin{tikzpicture}
%  \draw[very thin,->] (0,0) .. controls (0,-0.2) and (0.05,-0.25) .. (0.25,-0.25);
%\end{tikzpicture}
%}
%-----------------------------------------------------------------------------------------------
