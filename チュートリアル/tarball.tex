ソースファイルを配布したい、という場合もあるかもしれない。
そのようなときには、それをアーカイブすることが必要になる。
Buildtoolsに含まれるスクリプトのarlは、ルートファイルが取り込むサブファイルや画像ファイルを検索し、それらアーカイブする。
このアーカイブされたファイルのうち、tarというコマンドで作られたものをtarballという。
\begin{itemize}
\item -gオプションが与えられると、gzipで圧縮されたtarballを作る。
\item -bオプションが与えられると、bzip2で圧縮されたtarballを作る。
\item -zオプションが与えられると、zipファイルを作る。
\item オプションが与えられなければ、非圧縮のtarballを作る。
\end{itemize}

もし、プリプロセッシングで生成されるlatexソースファイルなどがある場合は、arlの実行前にそれらを生成しておかなければならない。
\begin{verbatim}
$ arl
$ tar -tf main.tar
main.tex
edit_tex_files.tex
generate_templates.tex
installation.tex
lb.tex
preprocessing.tex
rake.tex
readme.tex
tarball.tex
test_compile.tex
helper.tex
Tutorial_1.png
Tutorial_2.png
hellolatex.png
tableofcontents.png
test_installation.png
\end{verbatim}

Rakefileも当然ながら、tarballに含めなければならない。
\begin{verbatim}
$ tar -rf main.tar Rakefile
\end{verbatim}
そして、gzipなどに圧縮してtarballが完成する。
\begin{verbatim}
$ gzip main.tar
\end{verbatim}

以上の手続きは、実はすでにRakefileに記述されている。
「rake ar」とタイプすることにより、rakeがtarballを自動生成してくれる。
\begin{verbatim}
$ rm main.tar.gz
$ rake ar
arl main.tex
tar -rf main.tar Rakefile
gzip main.tar
mv main.tar.gz チュートリアル.tar.gz
\end{verbatim}
最後にできあがるtarballの名前は「チュートリアル.tar.gz」である。

もしも、zipファイルを作りたければ、「rake zip」とタイプする。

このセクションで、チュートリアルは終わりである。
次のセクションはBuildtoolsのソースファイルに含まれるReadme.ja.mdのコピーである。
この文書には、Buildtoolsを作成した背景や個々のスクリプトの特長と使い方が書かれている。

