\subsection{動作条件}
Buildtoolsには次のものが必要である。
\begin{enumerate}
\item Linux OS とbash
\item LaTeXシステム
\item MakeまたはRake
\end{enumerate}

\subsubsection{Linux OSとbash}
BuildtoolsはDebianとUbuntuで動作を確認している。
おそらく他のlinuxディストリビューションでも動作すると思われる。
Buildtoolsのスクリプトではbashのコマンドが使われているので、bashは必須である。
\subsubsection{LaTeXシステム}
LaTeXのインストールには2つの方法がある。

ひとつは、ディストリビューションに含まれているLaTeXのパッケージをインストールする方法である。
ubuntuディストリビューションの場合は、次のようにタイプしてインストールする。
\begin{verbatim}
$ sudo apt-get install texlive-full
\end{verbatim}
他のディストリビューションであれば、そのディストリビューションのドキュメントを参照してインストールしてほしい。

もうひとつの方法はTexLive (\url{https://www.tug.org/texlive})からインストールする方法である。
その方法については、ウェブサイトのドキュメントを参照してほしい。

\subsubsection{Makeまたはrake}
これらのアプリケーションはBuildtoolsに含まれるツールの実行には直接的には必要ではない。
しかしながら、これらのツールはmakeまたはrakeのコントロールの下で実行することが望ましい。
makeとrakeの両方をインストールする必要はなく、どちらかひとつ、好みのものをインストールすれば十分である。

Makeは古くから使われているビルド・ツールで、元々はCコンパイラ用に開発されたものである。
ubuntuでは、次のようにタイプしてmakeをインストールする。
\begin{verbatim}
$ sudo apt-get install make
\end{verbatim}

Rakeはmakeに似たビルド・ツールであり、rubyアプリケーションのひとつである。
rakeの良いところは、任意のrubyコードをRakefile(rakeの動作を記述するスクリプト)に書くことができる、ということである。
一般的に、Rakefileの方がMakefileよりも読みやすく、理解しやすい。
ubuntuでは、次のようにタイプしてrakeをインストールする。
\begin{verbatim}
$ sudo apt-get install rake
\end{verbatim}

もしも、rubyの最新版をインストールしたければ、rbenvとruby-buildを使ってインストールするのが良い。
下記のgithubレポジトリのドキュメントをインストールの参考にしてほしい。
\begin{itemize}
\item \url{https://github.com/rbenv/rbenv}
\item \url{https://github.com/rbenv/ruby-build}
\end{itemize}

\subsection{インストール}
\subsubsection{ダウンロード}
まず、次のgithubリポジトリにブラウザでアクセスする。
\begin{itemize}
\item \url{https://github.com/ToshioCP/LaTeX-BuildTools}
\end{itemize}
\verb|Code|ボタンをクリックするとポップアップ・メニューが現れる。
\verb|DOWNLOAD ZIP|メニューをクリックするとzipファイルがダウンロードされるので、それを解凍する。
\subsubsection{インストール}
端末を起動して、カレント・ディレクトリを先程解凍したファイルのディレクトリに移動する。
次のようにタイプしてスクリプトをインストールする。
\begin{verbatim}
$ bash install.sh
\end{verbatim}
このスクリプトは実行スクリプトを\verb|\$HOME/bin|にインストールする。
DebianとUbuntuでは、\verb|\$HOME/bin|ディレクトリがログイン時に存在すれば、それを\verb|PATH|環境変数に追加する。
\verb|install.sh|スクリプトは、\verb|\$HOME/bin|が存在しなければ、それを作成するが、その場合はディレクトリを\verb|PATH|に追加するために再ログインが必要である。
この方法は、ユーザのプライベート・ディレクトリにスクリプトを配置するので、他のユーザはスクリプトにアクセスすることはできない。
このインストールをユーザ・レベル・インストールまたはプライベート・インストールという。

もしも、インストール先を\verb|/user/local/bin|にしたければ、root権限が必要になる。
ubuntuの場合は、
\begin{verbatim}
$ sudo bash install.sh
\end{verbatim}
とタイプすれば、システムレベルのインストールができる。
