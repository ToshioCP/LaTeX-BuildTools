\subsection{newtex.conf}
The script `newtex' makes a directory and generates template files in it.
This is used at the beginning of the work.

First, a configuration file `newtex.conf' needs to be made.
There is a template file included in the Buildtools source files.
\begin{verbatim}
# This is a configuration file for newtex.
# The name of this file is newtex.conf
# A string between # and new line is a commnet and it is ignored
 by newtex.
# Empty line is also ignored. 

# document name
title="Tutorial"

# lb.conf
# Lb.conf has six lines.
# The following six lines are copied to lb.conf.
rootfile=main.tex
builddir=_build
engine=pdflatex
latex_option=-halt-on-error
dvipdf=
preview=evince

# documentclass
documentclass=article

# chapters/sections and subfile names
#   Chapters/sections and subfile names must be surrounded by
 double quotes.
#   Subfile names have no suffix or ".tex" suffix.
# If your LaTeX file is not big and no subfile is necessary, then
 leave out the following lines.
section="Installation" "installation"
section="Run lb to compile tex files" "lb"
section="Generate templates" "generate_templates"   # Subfiles
 are NOT allowed to include space characters. Use underscore
 instead of space. 
section="Edit tex files" "edit_tex_files"
section="Test compile" "test_compile"
section="Preprocessing" "preprocessing"
section="Use rake" "rake"
section="Make tarball" "tarball"

\end{verbatim}
In this tutorial, I want to show you how to make this tutorial pdf file with Buildtools.
The file above is exactly the same as the newtex.conf file as I used to make it.

A string after hash mark (\#) in a line is comment and it is ignored by newtex.
Empty lines are also ignored.
The remaining lines are instructions to newtex.
Each line has a `key=value' structure.
The keys are:
\begin{description}
\item[title] The title of the document you make.
\item[rootfile] The name of the rootfile.
\item[builddir] The name of the build directory
\item[engine] A latex engine to compile source files
\item[latex\_option] The options you want to give to the latex engine
\item[dvipdf] A program that converts dvi to pdf.
\item[preview] A pdf viewer
\item[documentclass] The name of the documentclass you want to use
\item[chapter] Chapters and corresponding subfiles
\item[section] Sections and corresponding subfiles
\end{description}
If you make a book (big document) and use book documentclass, use `chapter' and `section' key.
If you make an article (small document) and use article documentclass, use `section' key only.

\subsection{Run newtex}
After you finish editing newtex.conf, just type:
\begin{verbatim}
$ newtex
\end{verbatim}
Then, newtex makes a directory of which the name is `Tutorial', which is the same as the title in newtex.conf.
If the title includes space characters, they are converted to underscore.
For example, a title `A tutorial for beginners' is converted to `A\_tutorial\_for\_beginners'.
This is because a file name includes space character sometimes causes problems.
Newtex also generates template files under the directory.
\begin{verbatim}
$ cd Tutorial
$ ls
Makefile            generate_templates.tex  main.tex
Rakefile            helper.tex              preprocessing.tex
cover.tex           installation.tex        rake.tex
edit_tex_files.tex  lb.conf                 tarball.tex
gecko.png           lb.tex                  test_compile.tex
\end{verbatim}

Look at some important files.
\begin{verbatim}
$ cat lb.conf
rootfile=main
builddir=_build
engine=pdflatex
latex_option=-halt-on-error
dvipdf=
preview=evince
\end{verbatim}
The contents of this file is the copy of the part of newtex.conf.
\begin{verbatim}
$ cat main.tex
\documentclass{article}
% helper.tex

% 注意
% graphicxのドライバーがluatexになっている。他のドライバーを使う場合は書き換えが必要。

% パッケージのとりこみ
\usepackage{amsmath,amssymb}
\usepackage[luatex]{graphicx}
\usepackage{tikz}
\usetikzlibrary{calc}
\usetikzlibrary{topaths}
\usetikzlibrary{plotmarks}
\usetikzlibrary{intersections}
\usetikzlibrary{arrows,decorations.pathmorphing,backgrounds,positioning,fit,petri}
\usetikzlibrary{arrows.meta}
\usetikzlibrary{3d}
%\usepackage{gnuplot-lua-tikz}

\usepackage{fancybox}
\usepackage{booktabs}

\usepackage[margin=2.4cm]{geometry}

\usepackage[colorlinks=true,linkcolor=black,pdfencoding=auto]{hyperref}

% マクロのサンプル(解答、証明、q.e.d.、解答)
%\newcommand{\solution}{\begin{flushleft}\textbf{解:}\end{flushleft}}
%\newcommand{\proof}{\begin{flushleft}\textbf{証明}\end{flushleft}}
%\newcommand{\qed}{\begin{flushright}\textbf{証明終}\end{flushright}}

% 定理環境(定理、補題、系、定義、例、練習問題)
\newtheorem{theorem}{定理}[section]
\newtheorem{lemma}{補題}[section]
\newtheorem{corollary}{系}[section]
\newtheorem{definition}{定義}[section]
\newtheorem{example}{例}[section]
\newtheorem{exercise}{問題}[section]

% 凹凸増減表の矢印 ----------------------------------------------------------------------------
% concave north east arrow 上に凸で増加
\newcommand{\ccnearrow}{
\begin{tikzpicture}
  \draw[very thin,->] (0,0) .. controls (0,0.2) and (0.05,0.25) .. (0.25,0.25);
\end{tikzpicture}
}
% concave south east arrow 上に凸で減少
\newcommand{\ccsearrow}{
\begin{tikzpicture}
  \draw[very thin,->] (0,0) .. controls (0.2,0) and (0.25,-0.05) .. (0.25,-0.25);
\end{tikzpicture}
}
% convex north east arrow 下に凸で増加
\newcommand{\cvnearrow}{
\begin{tikzpicture}
  \draw[very thin,->] (0,0) .. controls (0.2,0) and (0.25,0.05) .. (0.25,0.25);
\end{tikzpicture}
}
% convex south east arrow 下に凸で減少
\newcommand{\cvsearrow}{
\begin{tikzpicture}
  \draw[very thin,->] (0,0) .. controls (0,-0.2) and (0.05,-0.25) .. (0.25,-0.25);
\end{tikzpicture}
}
%-----------------------------------------------------------------------------------------------

% for long division (割り算の筆算のためのマクロ、Tikzの中で使う)=======================================================
% \divisoroffset is the distance between x-axis 0 and the center of the divisor. It is negative number.
\newcommand{\divisor}[2]{\node[anchor=base] at (#1,0) {#2}}
% dividendoffset is the offset from x-axis 0 to the first term bound of the dividend.
\newcommand{\dividendoffset}{0.8}
% \termwidth is the width between the consecutive term.
\newcommand{\termwidth}{0.9}
% expressionheight is the height of each expression.
\newcommand{\expressionheight}{0.5}
\newcommand{\pt}[3]{\node[base left] at (\dividendoffset+\termwidth*#1,\expressionheight*#2) {#3}}
\newcommand{\hseparator}[3]{\draw ({\dividendoffset+(#1-1)*\termwidth},{-(#3+0.3)*\expressionheight})
                            -- ({\dividendoffset+(#2)*\termwidth},{-(#3+0.3)*\expressionheight})}
\newcommand{\divisionbox}[1]{\draw (0,\expressionheight*0.7) -- (\dividendoffset+\termwidth*#1,\expressionheight*0.7)
                                   (0,\expressionheight*0.7) arc[start angle=27,end angle=-27,radius=0.5]}
%=========================================================================================================================

\title{Tutorial}
\author{} % Write your name if necessary.
\begin{document}
\maketitle
% If you want a table of contents here, uncomment the following
 line.
%\tableofcontents

\section{Installation}
  \subsection{動作条件}
Buildtoolsには次のものが必要である。
\begin{enumerate}
\item Linux OS とbash
\item LaTeXシステム
\item MakeまたはRake
\end{enumerate}

\subsubsection{Linux OSとbash}
BuildtoolsはDebianとUbuntuで動作を確認している。
おそらく他のlinuxディストリビューションでも動作すると思われる。
Buildtoolsのスクリプトではbashのコマンドが使われているので、bashは必須である。
\subsubsection{LaTeXシステム}
LaTeXのインストールには2つの方法がある。

ひとつは、ディストリビューションに含まれているLaTeXのパッケージをインストールする方法である。
ubuntuディストリビューションの場合は、次のようにタイプしてインストールする。
\begin{verbatim}
$ sudo apt-get install texlive-full
\end{verbatim}
他のディストリビューションであれば、そのディストリビューションのドキュメントを参照してインストールしてほしい。

もうひとつの方法はTexLive (\url{https://www.tug.org/texlive})からインストールする方法である。
その方法については、ウェブサイトのドキュメントを参照してほしい。

\subsubsection{Makeまたはrake}
これらのアプリケーションはBuildtoolsに含まれるツールの実行には直接的には必要ではない。
しかしながら、Buildtoolsはmakeまたはrakeのコントロールの下で実行することが望ましい。
makeとrakeの両方をインストールする必要はなく、どちらかひとつ、好みのものをインストールすれば十分である。

Makeは古くから使われているビルド・ツールで、元々はCコンパイラ用に開発されたものである。
ubuntuでは、次のようにタイプしてmakeをインストールする。
\begin{verbatim}
$ sudo apt-get install make
\end{verbatim}

Rakeはmakeに似たビルド・ツールであり、rubyアプリケーションのひとつである。
rakeの良いところは、任意のrubyコードをRakefile(rakeの動作を記述するスクリプト)に書くことができる、ということである。
一般的に、Rakefileの方がMakefileよりも読みやすく、理解しやすい。
ubuntuでは、次のようにタイプしてrakeをインストールする。
\begin{verbatim}
$ sudo apt-get install rake
\end{verbatim}

もしも、rubyの最新版をインストールしたければ、rbenvとruby-buildを使ってインストールするのが良い。
下記のgithubレポジトリのドキュメントをインストールの参考にしてほしい。
\begin{itemize}
\item \url{https://github.com/rbenv/rbenv}
\item \url{https://github.com/rbenv/ruby-build}
\end{itemize}

\subsection{インストール}
\subsubsection{ダウンロード}
まず、次のgithubリポジトリにブラウザでアクセスする。
\begin{itemize}
\item \url{https://github.com/ToshioCP/LaTeX-BuildTools}
\end{itemize}
\verb|Code|ボタンをクリックするとポップアップ・メニューが現れる。
\verb|DOWNLOAD ZIP|メニューをクリックするとzipファイルがダウンロードされるので、それを解凍する。
\subsubsection{インストール}
端末を起動して、カレント・ディレクトリを先程解凍したファイルのディレクトリに移動する。
次のようにタイプしてスクリプトをインストールする。
\begin{verbatim}
$ bash install.sh
\end{verbatim}
このスクリプトは実行スクリプトを\verb|\$HOME/bin|の下に(ディレクトリが存在しなければ、作成して)インストールする。
スクリプトが\verb|\$HOME/bin|を作成した場合は再ログインが必要である。
この方法は、ユーザのプライベート・ディレクトリにスクリプトを配置するので、他のユーザはスクリプトにアクセスすることはできない。
このインストールをユーザ・レベル・インストールまたはプライベート・インストールという。

もしも、インストール先を\verb|/user/local/bin|にしたければ、root権限が必要になる。
ubuntuの場合は、
\begin{verbatim}
$ sudo bash install.sh
\end{verbatim}
とタイプすると、install.shは\verb|/user/local/bin|にスクリプトをインストールする。
これによって、そのマシンにログインするすべてのユーザでスクリプトを使うことができるようになる。


\section{Run lb to compile tex files}
  \subsection{First step}
Lb is the main script in Buildtools.
This section describes how to use it with a small example.

First, make a directory named \verb|example| and change your current directory to it.
\begin{verbatim}
$ makedir example
$ cd example
\end{verbatim}

Then make a tex source file in the directory.
Run your favourite editor and copy the following text, then save it as the name \verb|main.tex|.
\begin{verbatim}
\documentclass{article}
\begin{document}
Hello \LaTeX !!
\end{document}
\end{verbatim}

Then, just type \verb|lb|.
\begin{verbatim}
$ lb
\end{verbatim}
Then, it runs latexmk and pdflatex and compile \verb|main.tex| with them.
Messages appear on your screen, and that shows the process of the compilation.
If there is a line 
\begin{verbatim}
Output written on _build/main.pdf (1 page, 19263 bytes).
\end{verbatim}
then the compilation completes correctly.
Check the directory.
\begin{verbatim}
$ ls -l
total 8
drwxrwxr-x 2 user user 4096 Dec  6 11:59 _build
-rw-rw-r-- 1 user user   72 Dec  6 11:59 main.tex
\end{verbatim}
A new directory \verb|_build| is generated.
Look at the files in the directory.
\begin{verbatim}
$ cd _build
$ ls
\end{verbatim}
There are auxliary files and the target file \verb|main.pdf|.
See \verb|main.pdf| with your pdf-viewer, for example evince.
\begin{verbatim}
$ evince main.pdf
\end{verbatim}
\begin{center}
\includegraphics[width=3cm]{hellolatex.png}
\end{center}

\subsection{Use lb.conf}
In the previous subsection, lb runs pdflatex.
The reason why lb chose pdflatex is the documentclass `article'.
It can also be compiled by lualatex or xelatex, but pdflatex has been a standard latex engine for ages.

If you want to use, for example, lualatex to compile, you need to specify it in \verb|lb.conf|.
This configuration file has six items.
\begin{description}
\item[rootfile] Rootfile is the main tex file, which usually includes {\textbackslash}begin\{document\} and {\textbackslash}end\{document\}. Other tex files are called `subfile'.
\item[builddir] This is a temporary directory includes all the auxliary files and the target file, which is usually a pdf file.
\item[engine] This specifies a latex engine, which is one of pdflatex, xelatex, lualatex, latex and platex.
\item[latex\_option] This specifies options to give \verb|latexmk|. The option `-halt-on-error' is given to \verb|latexmk| even if lb.conf doesn't exist.
\item[dvipdf] This is a program which converts dvi into pdf, which is used only with latex or platex. It is unnecessary with other latex engines. `dvipdfmx' is the best at present.
\item[preview] Pdf viewer. This is used to preview the pdf file when lb is given a subfile as an argument.
\end{description}

Run your editor, type the following and save it as the name \verb|lb.conf|.
\begin{verbatim}
rootfile=main
builddir=_build
engine=lualatex
latex_option=-halt-on-error
dvipdf=
preview=evince
\end{verbatim}
Then, type
\begin{verbatim}
$ lb
\end{verbatim}
it used lualatex to compile.

If you want to change the name of the tex file to `example.tex', then modify the first line in lb.conf to
\begin{verbatim}
rootfile=example
\end{verbatim}
or
\begin{verbatim}
rootfile=example.tex
\end{verbatim}
The suffix can be left out.

In addition, if you want to put all the axiliary files and the target file in the source directory, change the second line in lb.conf to:
\begin{verbatim}
builddir=
\end{verbatim}
This specifies null string for builddir item and that means no build directory is made.

Let's try to run \verb|lb| with the following \verb|lb.conf|.
\begin{verbatim}
rootfile=example
builddir=
engine=latex
latex_option=-halt-on-error
dvipdf=dvipdfmx
preview=evince
\end{verbatim}
Now, the engine is latex and dvipdf program is dvipdfmx.
\begin{verbatim}
$ rm -r _build
$ mv main.tex example.tex
$ lb
\end{verbatim}
Then messages appear.
It includes the following line.
\begin{verbatim}
This is pdfTeX, Version 3.14159265-2.6-1.40.21 (TeX Live 2020)
 (preloaded format=latex)
  ... ...
  ... ...
Output written on example.dvi (1 page, 332 bytes).
Transcript written on example.log.
Latexmk: Examining 'example.log'
=== TeX engine is 'pdfTeX'
Latexmk: Log file says output to 'example.dvi'
Latexmk: All targets (example.dvi) are up-to-date
example.dvi -> example.pdf
[1]
3662 bytes written
\end{verbatim}
This tells us that the engine was `latex'%
\footnote{
In Texlive2020, `latex' command calls `pdftex' instead of `tex' which is the original TeX program.
}.
It generates dvi file instead of pdf file.
After that, dvipdfmx is run by latexmk and it translates the dvi file into a pdf file.
The name `dvipdfmx' doesn't appear in the message but `example.dvi -{\textgreater} example.pdf' is outputed by dvipdfmx.
So we know that dvipdfmx was run by latexmk in the build process.
\begin{verbatim}
$ ls
example.aux  example.fdb_latexmk  example.log  example.tex
example.dvi  example.fls          example.pdf  lb.conf
\end{verbatim}
There's no temprary directory like \_build because we specified null string for builddir.

One of the important feature of lb is compiling a subfile separately.
This will be explained in the section  \ref{sec:testcompile} `Test compile' (p. \pageref{sec:testcompile}).

\section{Generate templates}
  \subsection{newtex.conf}
The script `newtex' makes a directory and generates template files in it.
This is used at the beginning of the work.

First, a configuration file `newtex.conf' needs to be made.
There is a template file included in the Buildtools source files.
\begin{verbatim}
# This is a configuration file for newtex.
# The name of this file is newtex.conf
# A string between # and new line is a commnet and it is ignored
 by newtex.
# Empty line is also ignored. 

# document name
title="Tutorial"

# lb.conf
# Lb.conf has six lines.
# The following six lines are copied to lb.conf.
rootfile=main.tex
builddir=_build
engine=pdflatex
latex_option=-halt-on-error
dvipdf=
preview=evince

# documentclass
documentclass=article

# chapters/sections and subfile names
#   Chapters/sections and subfile names must be surrounded by
 double quotes.
#   Subfile names have no suffix or ".tex" suffix.
# If your LaTeX file is not big and no subfile is necessary, then
 leave out the following lines.
section="Installation" "installation"
section="Run lb to compile tex files" "lb"
section="Generate templates" "generate_templates"   # Subfiles
 are NOT allowed to include space characters. Use underscore
 instead of space. 
section="Edit tex files" "edit_tex_files"
section="Test compile" "test_compile"
section="Preprocessing" "preprocessing"
section="Use rake" "rake"
section="Make tarball" "tarball"

\end{verbatim}
In this tutorial, I want to show you how to make this tutorial pdf file with Buildtools.
The file above is exactly the same as the newtex.conf file as I used to make it.

A string after hash mark (\#) in a line is comment and it is ignored by newtex.
Empty lines are also ignored.
The remaining lines are instructions to newtex.
Each line has a `key=value' structure.
The keys are:
\begin{description}
\item[title] The title of the document you make.
\item[rootfile] The name of the rootfile.
\item[builddir] The name of the build directory
\item[engine] A latex engine to compile source files
\item[latex\_option] The options you want to give to the latex engine
\item[dvipdf] A program that converts dvi to pdf.
\item[preview] A pdf viewer
\item[documentclass] The name of the documentclass you want to use
\item[chapter] Chapters and corresponding subfiles
\item[section] Sections and corresponding subfiles
\end{description}
If you make a book (big document) and use book documentclass, use `chapter' and `section' key.
If you make an article (small document) and use article documentclass, use `section' key only.

\subsection{Run newtex}
After you finish editing newtex.conf, just type:
\begin{verbatim}
$ newtex
\end{verbatim}
Then, newtex makes a directory of which the name is `Tutorial', which is the same as the title in newtex.conf.
If the title includes space characters, they are converted to underscore.
For example, a title `A tutorial for beginners' is converted to `A\_tutorial\_for\_beginners'.
This is because a file name includes space character sometimes causes problems.
Newtex also generates template files under the directory.
\begin{verbatim}
$ cd Tutorial
$ ls
Makefile            generate_templates.tex  main.tex
Rakefile            helper.tex              preprocessing.tex
cover.tex           installation.tex        rake.tex
edit_tex_files.tex  lb.conf                 tarball.tex
gecko.png           lb.tex                  test_compile.tex
\end{verbatim}

Look at some important files.
\begin{verbatim}
$ cat lb.conf
rootfile=main
builddir=_build
engine=pdflatex
latex_option=-halt-on-error
dvipdf=
preview=evince
\end{verbatim}
The contents of this file is the copy of the part of newtex.conf.
\begin{verbatim}
$ cat main.tex
\documentclass{article}
% helper.tex

% 注意
% graphicxのドライバーがluatexになっている。他のドライバーを使う場合は書き換えが必要。

% パッケージのとりこみ
\usepackage{amsmath,amssymb}
\usepackage[luatex]{graphicx}
\usepackage{tikz}
\usetikzlibrary{calc}
\usetikzlibrary{topaths}
\usetikzlibrary{plotmarks}
\usetikzlibrary{intersections}
\usetikzlibrary{arrows,decorations.pathmorphing,backgrounds,positioning,fit,petri}
\usetikzlibrary{arrows.meta}
\usetikzlibrary{3d}
%\usepackage{gnuplot-lua-tikz}

\usepackage{fancybox}
\usepackage{booktabs}

\usepackage[margin=2.4cm]{geometry}

\usepackage[colorlinks=true,linkcolor=black,pdfencoding=auto]{hyperref}

% マクロのサンプル(解答、証明、q.e.d.、解答)
%\newcommand{\solution}{\begin{flushleft}\textbf{解:}\end{flushleft}}
%\newcommand{\proof}{\begin{flushleft}\textbf{証明}\end{flushleft}}
%\newcommand{\qed}{\begin{flushright}\textbf{証明終}\end{flushright}}

% 定理環境(定理、補題、系、定義、例、練習問題)
\newtheorem{theorem}{定理}[section]
\newtheorem{lemma}{補題}[section]
\newtheorem{corollary}{系}[section]
\newtheorem{definition}{定義}[section]
\newtheorem{example}{例}[section]
\newtheorem{exercise}{問題}[section]

% 凹凸増減表の矢印 ----------------------------------------------------------------------------
% concave north east arrow 上に凸で増加
\newcommand{\ccnearrow}{
\begin{tikzpicture}
  \draw[very thin,->] (0,0) .. controls (0,0.2) and (0.05,0.25) .. (0.25,0.25);
\end{tikzpicture}
}
% concave south east arrow 上に凸で減少
\newcommand{\ccsearrow}{
\begin{tikzpicture}
  \draw[very thin,->] (0,0) .. controls (0.2,0) and (0.25,-0.05) .. (0.25,-0.25);
\end{tikzpicture}
}
% convex north east arrow 下に凸で増加
\newcommand{\cvnearrow}{
\begin{tikzpicture}
  \draw[very thin,->] (0,0) .. controls (0.2,0) and (0.25,0.05) .. (0.25,0.25);
\end{tikzpicture}
}
% convex south east arrow 下に凸で減少
\newcommand{\cvsearrow}{
\begin{tikzpicture}
  \draw[very thin,->] (0,0) .. controls (0,-0.2) and (0.05,-0.25) .. (0.25,-0.25);
\end{tikzpicture}
}
%-----------------------------------------------------------------------------------------------

% for long division (割り算の筆算のためのマクロ、Tikzの中で使う)=======================================================
% \divisoroffset is the distance between x-axis 0 and the center of the divisor. It is negative number.
\newcommand{\divisor}[2]{\node[anchor=base] at (#1,0) {#2}}
% dividendoffset is the offset from x-axis 0 to the first term bound of the dividend.
\newcommand{\dividendoffset}{0.8}
% \termwidth is the width between the consecutive term.
\newcommand{\termwidth}{0.9}
% expressionheight is the height of each expression.
\newcommand{\expressionheight}{0.5}
\newcommand{\pt}[3]{\node[base left] at (\dividendoffset+\termwidth*#1,\expressionheight*#2) {#3}}
\newcommand{\hseparator}[3]{\draw ({\dividendoffset+(#1-1)*\termwidth},{-(#3+0.3)*\expressionheight})
                            -- ({\dividendoffset+(#2)*\termwidth},{-(#3+0.3)*\expressionheight})}
\newcommand{\divisionbox}[1]{\draw (0,\expressionheight*0.7) -- (\dividendoffset+\termwidth*#1,\expressionheight*0.7)
                                   (0,\expressionheight*0.7) arc[start angle=27,end angle=-27,radius=0.5]}
%=========================================================================================================================

\title{Tutorial}
\author{} % Write your name if necessary.
\begin{document}
\maketitle
% If you want a table of contents here, uncomment the following
 line.
%\tableofcontents

\section{Installation}
  \subsection{動作条件}
Buildtoolsには次のものが必要である。
\begin{enumerate}
\item Linux OS とbash
\item LaTeXシステム
\item MakeまたはRake
\end{enumerate}

\subsubsection{Linux OSとbash}
BuildtoolsはDebianとUbuntuで動作を確認している。
おそらく他のlinuxディストリビューションでも動作すると思われる。
Buildtoolsのスクリプトではbashのコマンドが使われているので、bashは必須である。
\subsubsection{LaTeXシステム}
LaTeXのインストールには2つの方法がある。

ひとつは、ディストリビューションに含まれているLaTeXのパッケージをインストールする方法である。
ubuntuディストリビューションの場合は、次のようにタイプしてインストールする。
\begin{verbatim}
$ sudo apt-get install texlive-full
\end{verbatim}
他のディストリビューションであれば、そのディストリビューションのドキュメントを参照してインストールしてほしい。

もうひとつの方法はTexLive (\url{https://www.tug.org/texlive})からインストールする方法である。
その方法については、ウェブサイトのドキュメントを参照してほしい。

\subsubsection{Makeまたはrake}
これらのアプリケーションはBuildtoolsに含まれるツールの実行には直接的には必要ではない。
しかしながら、Buildtoolsはmakeまたはrakeのコントロールの下で実行することが望ましい。
makeとrakeの両方をインストールする必要はなく、どちらかひとつ、好みのものをインストールすれば十分である。

Makeは古くから使われているビルド・ツールで、元々はCコンパイラ用に開発されたものである。
ubuntuでは、次のようにタイプしてmakeをインストールする。
\begin{verbatim}
$ sudo apt-get install make
\end{verbatim}

Rakeはmakeに似たビルド・ツールであり、rubyアプリケーションのひとつである。
rakeの良いところは、任意のrubyコードをRakefile(rakeの動作を記述するスクリプト)に書くことができる、ということである。
一般的に、Rakefileの方がMakefileよりも読みやすく、理解しやすい。
ubuntuでは、次のようにタイプしてrakeをインストールする。
\begin{verbatim}
$ sudo apt-get install rake
\end{verbatim}

もしも、rubyの最新版をインストールしたければ、rbenvとruby-buildを使ってインストールするのが良い。
下記のgithubレポジトリのドキュメントをインストールの参考にしてほしい。
\begin{itemize}
\item \url{https://github.com/rbenv/rbenv}
\item \url{https://github.com/rbenv/ruby-build}
\end{itemize}

\subsection{インストール}
\subsubsection{ダウンロード}
まず、次のgithubリポジトリにブラウザでアクセスする。
\begin{itemize}
\item \url{https://github.com/ToshioCP/LaTeX-BuildTools}
\end{itemize}
\verb|Code|ボタンをクリックするとポップアップ・メニューが現れる。
\verb|DOWNLOAD ZIP|メニューをクリックするとzipファイルがダウンロードされるので、それを解凍する。
\subsubsection{インストール}
端末を起動して、カレント・ディレクトリを先程解凍したファイルのディレクトリに移動する。
次のようにタイプしてスクリプトをインストールする。
\begin{verbatim}
$ bash install.sh
\end{verbatim}
このスクリプトは実行スクリプトを\verb|\$HOME/bin|の下に(ディレクトリが存在しなければ、作成して)インストールする。
スクリプトが\verb|\$HOME/bin|を作成した場合は再ログインが必要である。
この方法は、ユーザのプライベート・ディレクトリにスクリプトを配置するので、他のユーザはスクリプトにアクセスすることはできない。
このインストールをユーザ・レベル・インストールまたはプライベート・インストールという。

もしも、インストール先を\verb|/user/local/bin|にしたければ、root権限が必要になる。
ubuntuの場合は、
\begin{verbatim}
$ sudo bash install.sh
\end{verbatim}
とタイプすると、install.shは\verb|/user/local/bin|にスクリプトをインストールする。
これによって、そのマシンにログインするすべてのユーザでスクリプトを使うことができるようになる。


\section{Run lb to compile tex files}
  \subsection{First step}
Lb is the main script in Buildtools.
This section describes how to use it with a small example.

First, make a directory named \verb|example| and change your current directory to it.
\begin{verbatim}
$ makedir example
$ cd example
\end{verbatim}

Then make a tex source file in the directory.
Run your favourite editor and copy the following text, then save it as the name \verb|main.tex|.
\begin{verbatim}
\documentclass{article}
\begin{document}
Hello \LaTeX !!
\end{document}
\end{verbatim}

Then, just type \verb|lb|.
\begin{verbatim}
$ lb
\end{verbatim}
Then, it runs latexmk and pdflatex and compile \verb|main.tex| with them.
Messages appear on your screen, and that shows the process of the compilation.
If there is a line 
\begin{verbatim}
Output written on _build/main.pdf (1 page, 19263 bytes).
\end{verbatim}
then the compilation completes correctly.
Check the directory.
\begin{verbatim}
$ ls -l
total 8
drwxrwxr-x 2 user user 4096 Dec  6 11:59 _build
-rw-rw-r-- 1 user user   72 Dec  6 11:59 main.tex
\end{verbatim}
A new directory \verb|_build| is generated.
Look at the files in the directory.
\begin{verbatim}
$ cd _build
$ ls
\end{verbatim}
There are auxliary files and the target file \verb|main.pdf|.
See \verb|main.pdf| with your pdf-viewer, for example evince.
\begin{verbatim}
$ evince main.pdf
\end{verbatim}
\begin{center}
\includegraphics[width=3cm]{hellolatex.png}
\end{center}

\subsection{Use lb.conf}
In the previous subsection, lb runs pdflatex.
The reason why lb chose pdflatex is the documentclass `article'.
It can also be compiled by lualatex or xelatex, but pdflatex has been a standard latex engine for ages.

If you want to use, for example, lualatex to compile, you need to specify it in \verb|lb.conf|.
This configuration file has six items.
\begin{description}
\item[rootfile] Rootfile is the main tex file, which usually includes {\textbackslash}begin\{document\} and {\textbackslash}end\{document\}. Other tex files are called `subfile'.
\item[builddir] This is a temporary directory includes all the auxliary files and the target file, which is usually a pdf file.
\item[engine] This specifies a latex engine, which is one of pdflatex, xelatex, lualatex, latex and platex.
\item[latex\_option] This specifies options to give \verb|latexmk|. The option `-halt-on-error' is given to \verb|latexmk| even if lb.conf doesn't exist.
\item[dvipdf] This is a program which converts dvi into pdf, which is used only with latex or platex. It is unnecessary with other latex engines. `dvipdfmx' is the best at present.
\item[preview] Pdf viewer. This is used to preview the pdf file when lb is given a subfile as an argument.
\end{description}

Run your editor, type the following and save it as the name \verb|lb.conf|.
\begin{verbatim}
rootfile=main
builddir=_build
engine=lualatex
latex_option=-halt-on-error
dvipdf=
preview=evince
\end{verbatim}
Then, type
\begin{verbatim}
$ lb
\end{verbatim}
it used lualatex to compile.

If you want to change the name of the tex file to `example.tex', then modify the first line in lb.conf to
\begin{verbatim}
rootfile=example
\end{verbatim}
or
\begin{verbatim}
rootfile=example.tex
\end{verbatim}
The suffix can be left out.

In addition, if you want to put all the axiliary files and the target file in the source directory, change the second line in lb.conf to:
\begin{verbatim}
builddir=
\end{verbatim}
This specifies null string for builddir item and that means no build directory is made.

Let's try to run \verb|lb| with the following \verb|lb.conf|.
\begin{verbatim}
rootfile=example
builddir=
engine=latex
latex_option=-halt-on-error
dvipdf=dvipdfmx
preview=evince
\end{verbatim}
Now, the engine is latex and dvipdf program is dvipdfmx.
\begin{verbatim}
$ rm -r _build
$ mv main.tex example.tex
$ lb
\end{verbatim}
Then messages appear.
It includes the following line.
\begin{verbatim}
This is pdfTeX, Version 3.14159265-2.6-1.40.21 (TeX Live 2020)
 (preloaded format=latex)
  ... ...
  ... ...
Output written on example.dvi (1 page, 332 bytes).
Transcript written on example.log.
Latexmk: Examining 'example.log'
=== TeX engine is 'pdfTeX'
Latexmk: Log file says output to 'example.dvi'
Latexmk: All targets (example.dvi) are up-to-date
example.dvi -> example.pdf
[1]
3662 bytes written
\end{verbatim}
This tells us that the engine was `latex'%
\footnote{
In Texlive2020, `latex' command calls `pdftex' instead of `tex' which is the original TeX program.
}.
It generates dvi file instead of pdf file.
After that, dvipdfmx is run by latexmk and it translates the dvi file into a pdf file.
The name `dvipdfmx' doesn't appear in the message but `example.dvi -{\textgreater} example.pdf' is outputed by dvipdfmx.
So we know that dvipdfmx was run by latexmk in the build process.
\begin{verbatim}
$ ls
example.aux  example.fdb_latexmk  example.log  example.tex
example.dvi  example.fls          example.pdf  lb.conf
\end{verbatim}
There's no temprary directory like \_build because we specified null string for builddir.

One of the important feature of lb is compiling a subfile separately.
This will be explained in the section  \ref{sec:testcompile} `Test compile' (p. \pageref{sec:testcompile}).

\section{Generate templates}
  \subsection{newtex.conf}
The script `newtex' makes a directory and generates template files in it.
This is used at the beginning of the work.

First, a configuration file `newtex.conf' needs to be made.
There is a template file included in the Buildtools source files.
\begin{verbatim}
# This is a configuration file for newtex.
# The name of this file is newtex.conf
# A string between # and new line is a commnet and it is ignored
 by newtex.
# Empty line is also ignored. 

# document name
title="Tutorial"

# lb.conf
# Lb.conf has six lines.
# The following six lines are copied to lb.conf.
rootfile=main.tex
builddir=_build
engine=pdflatex
latex_option=-halt-on-error
dvipdf=
preview=evince

# documentclass
documentclass=article

# chapters/sections and subfile names
#   Chapters/sections and subfile names must be surrounded by
 double quotes.
#   Subfile names have no suffix or ".tex" suffix.
# If your LaTeX file is not big and no subfile is necessary, then
 leave out the following lines.
section="Installation" "installation"
section="Run lb to compile tex files" "lb"
section="Generate templates" "generate_templates"   # Subfiles
 are NOT allowed to include space characters. Use underscore
 instead of space. 
section="Edit tex files" "edit_tex_files"
section="Test compile" "test_compile"
section="Preprocessing" "preprocessing"
section="Use rake" "rake"
section="Make tarball" "tarball"

\end{verbatim}
In this tutorial, I want to show you how to make this tutorial pdf file with Buildtools.
The file above is exactly the same as the newtex.conf file as I used to make it.

A string after hash mark (\#) in a line is comment and it is ignored by newtex.
Empty lines are also ignored.
The remaining lines are instructions to newtex.
Each line has a `key=value' structure.
The keys are:
\begin{description}
\item[title] The title of the document you make.
\item[rootfile] The name of the rootfile.
\item[builddir] The name of the build directory
\item[engine] A latex engine to compile source files
\item[latex\_option] The options you want to give to the latex engine
\item[dvipdf] A program that converts dvi to pdf.
\item[preview] A pdf viewer
\item[documentclass] The name of the documentclass you want to use
\item[chapter] Chapters and corresponding subfiles
\item[section] Sections and corresponding subfiles
\end{description}
If you make a book (big document) and use book documentclass, use `chapter' and `section' key.
If you make an article (small document) and use article documentclass, use `section' key only.

\subsection{Run newtex}
After you finish editing newtex.conf, just type:
\begin{verbatim}
$ newtex
\end{verbatim}
Then, newtex makes a directory of which the name is `Tutorial', which is the same as the title in newtex.conf.
If the title includes space characters, they are converted to underscore.
For example, a title `A tutorial for beginners' is converted to `A\_tutorial\_for\_beginners'.
This is because a file name includes space character sometimes causes problems.
Newtex also generates template files under the directory.
\begin{verbatim}
$ cd Tutorial
$ ls
Makefile            generate_templates.tex  main.tex
Rakefile            helper.tex              preprocessing.tex
cover.tex           installation.tex        rake.tex
edit_tex_files.tex  lb.conf                 tarball.tex
gecko.png           lb.tex                  test_compile.tex
\end{verbatim}

Look at some important files.
\begin{verbatim}
$ cat lb.conf
rootfile=main
builddir=_build
engine=pdflatex
latex_option=-halt-on-error
dvipdf=
preview=evince
\end{verbatim}
The contents of this file is the copy of the part of newtex.conf.
\begin{verbatim}
$ cat main.tex
\documentclass{article}
% helper.tex

% 注意
% graphicxのドライバーがluatexになっている。他のドライバーを使う場合は書き換えが必要。

% パッケージのとりこみ
\usepackage{amsmath,amssymb}
\usepackage[luatex]{graphicx}
\usepackage{tikz}
\usetikzlibrary{calc}
\usetikzlibrary{topaths}
\usetikzlibrary{plotmarks}
\usetikzlibrary{intersections}
\usetikzlibrary{arrows,decorations.pathmorphing,backgrounds,positioning,fit,petri}
\usetikzlibrary{arrows.meta}
\usetikzlibrary{3d}
%\usepackage{gnuplot-lua-tikz}

\usepackage{fancybox}
\usepackage{booktabs}

\usepackage[margin=2.4cm]{geometry}

\usepackage[colorlinks=true,linkcolor=black,pdfencoding=auto]{hyperref}

% マクロのサンプル(解答、証明、q.e.d.、解答)
%\newcommand{\solution}{\begin{flushleft}\textbf{解:}\end{flushleft}}
%\newcommand{\proof}{\begin{flushleft}\textbf{証明}\end{flushleft}}
%\newcommand{\qed}{\begin{flushright}\textbf{証明終}\end{flushright}}

% 定理環境(定理、補題、系、定義、例、練習問題)
\newtheorem{theorem}{定理}[section]
\newtheorem{lemma}{補題}[section]
\newtheorem{corollary}{系}[section]
\newtheorem{definition}{定義}[section]
\newtheorem{example}{例}[section]
\newtheorem{exercise}{問題}[section]

% 凹凸増減表の矢印 ----------------------------------------------------------------------------
% concave north east arrow 上に凸で増加
\newcommand{\ccnearrow}{
\begin{tikzpicture}
  \draw[very thin,->] (0,0) .. controls (0,0.2) and (0.05,0.25) .. (0.25,0.25);
\end{tikzpicture}
}
% concave south east arrow 上に凸で減少
\newcommand{\ccsearrow}{
\begin{tikzpicture}
  \draw[very thin,->] (0,0) .. controls (0.2,0) and (0.25,-0.05) .. (0.25,-0.25);
\end{tikzpicture}
}
% convex north east arrow 下に凸で増加
\newcommand{\cvnearrow}{
\begin{tikzpicture}
  \draw[very thin,->] (0,0) .. controls (0.2,0) and (0.25,0.05) .. (0.25,0.25);
\end{tikzpicture}
}
% convex south east arrow 下に凸で減少
\newcommand{\cvsearrow}{
\begin{tikzpicture}
  \draw[very thin,->] (0,0) .. controls (0,-0.2) and (0.05,-0.25) .. (0.25,-0.25);
\end{tikzpicture}
}
%-----------------------------------------------------------------------------------------------

% for long division (割り算の筆算のためのマクロ、Tikzの中で使う)=======================================================
% \divisoroffset is the distance between x-axis 0 and the center of the divisor. It is negative number.
\newcommand{\divisor}[2]{\node[anchor=base] at (#1,0) {#2}}
% dividendoffset is the offset from x-axis 0 to the first term bound of the dividend.
\newcommand{\dividendoffset}{0.8}
% \termwidth is the width between the consecutive term.
\newcommand{\termwidth}{0.9}
% expressionheight is the height of each expression.
\newcommand{\expressionheight}{0.5}
\newcommand{\pt}[3]{\node[base left] at (\dividendoffset+\termwidth*#1,\expressionheight*#2) {#3}}
\newcommand{\hseparator}[3]{\draw ({\dividendoffset+(#1-1)*\termwidth},{-(#3+0.3)*\expressionheight})
                            -- ({\dividendoffset+(#2)*\termwidth},{-(#3+0.3)*\expressionheight})}
\newcommand{\divisionbox}[1]{\draw (0,\expressionheight*0.7) -- (\dividendoffset+\termwidth*#1,\expressionheight*0.7)
                                   (0,\expressionheight*0.7) arc[start angle=27,end angle=-27,radius=0.5]}
%=========================================================================================================================

\title{Tutorial}
\author{} % Write your name if necessary.
\begin{document}
\maketitle
% If you want a table of contents here, uncomment the following
 line.
%\tableofcontents

\section{Installation}
  \subsection{動作条件}
Buildtoolsには次のものが必要である。
\begin{enumerate}
\item Linux OS とbash
\item LaTeXシステム
\item MakeまたはRake
\end{enumerate}

\subsubsection{Linux OSとbash}
BuildtoolsはDebianとUbuntuで動作を確認している。
おそらく他のlinuxディストリビューションでも動作すると思われる。
Buildtoolsのスクリプトではbashのコマンドが使われているので、bashは必須である。
\subsubsection{LaTeXシステム}
LaTeXのインストールには2つの方法がある。

ひとつは、ディストリビューションに含まれているLaTeXのパッケージをインストールする方法である。
ubuntuディストリビューションの場合は、次のようにタイプしてインストールする。
\begin{verbatim}
$ sudo apt-get install texlive-full
\end{verbatim}
他のディストリビューションであれば、そのディストリビューションのドキュメントを参照してインストールしてほしい。

もうひとつの方法はTexLive (\url{https://www.tug.org/texlive})からインストールする方法である。
その方法については、ウェブサイトのドキュメントを参照してほしい。

\subsubsection{Makeまたはrake}
これらのアプリケーションはBuildtoolsに含まれるツールの実行には直接的には必要ではない。
しかしながら、Buildtoolsはmakeまたはrakeのコントロールの下で実行することが望ましい。
makeとrakeの両方をインストールする必要はなく、どちらかひとつ、好みのものをインストールすれば十分である。

Makeは古くから使われているビルド・ツールで、元々はCコンパイラ用に開発されたものである。
ubuntuでは、次のようにタイプしてmakeをインストールする。
\begin{verbatim}
$ sudo apt-get install make
\end{verbatim}

Rakeはmakeに似たビルド・ツールであり、rubyアプリケーションのひとつである。
rakeの良いところは、任意のrubyコードをRakefile(rakeの動作を記述するスクリプト)に書くことができる、ということである。
一般的に、Rakefileの方がMakefileよりも読みやすく、理解しやすい。
ubuntuでは、次のようにタイプしてrakeをインストールする。
\begin{verbatim}
$ sudo apt-get install rake
\end{verbatim}

もしも、rubyの最新版をインストールしたければ、rbenvとruby-buildを使ってインストールするのが良い。
下記のgithubレポジトリのドキュメントをインストールの参考にしてほしい。
\begin{itemize}
\item \url{https://github.com/rbenv/rbenv}
\item \url{https://github.com/rbenv/ruby-build}
\end{itemize}

\subsection{インストール}
\subsubsection{ダウンロード}
まず、次のgithubリポジトリにブラウザでアクセスする。
\begin{itemize}
\item \url{https://github.com/ToshioCP/LaTeX-BuildTools}
\end{itemize}
\verb|Code|ボタンをクリックするとポップアップ・メニューが現れる。
\verb|DOWNLOAD ZIP|メニューをクリックするとzipファイルがダウンロードされるので、それを解凍する。
\subsubsection{インストール}
端末を起動して、カレント・ディレクトリを先程解凍したファイルのディレクトリに移動する。
次のようにタイプしてスクリプトをインストールする。
\begin{verbatim}
$ bash install.sh
\end{verbatim}
このスクリプトは実行スクリプトを\verb|\$HOME/bin|の下に(ディレクトリが存在しなければ、作成して)インストールする。
スクリプトが\verb|\$HOME/bin|を作成した場合は再ログインが必要である。
この方法は、ユーザのプライベート・ディレクトリにスクリプトを配置するので、他のユーザはスクリプトにアクセスすることはできない。
このインストールをユーザ・レベル・インストールまたはプライベート・インストールという。

もしも、インストール先を\verb|/user/local/bin|にしたければ、root権限が必要になる。
ubuntuの場合は、
\begin{verbatim}
$ sudo bash install.sh
\end{verbatim}
とタイプすると、install.shは\verb|/user/local/bin|にスクリプトをインストールする。
これによって、そのマシンにログインするすべてのユーザでスクリプトを使うことができるようになる。


\section{Run lb to compile tex files}
  \subsection{First step}
Lb is the main script in Buildtools.
This section describes how to use it with a small example.

First, make a directory named \verb|example| and change your current directory to it.
\begin{verbatim}
$ makedir example
$ cd example
\end{verbatim}

Then make a tex source file in the directory.
Run your favourite editor and copy the following text, then save it as the name \verb|main.tex|.
\begin{verbatim}
\documentclass{article}
\begin{document}
Hello \LaTeX !!
\end{document}
\end{verbatim}

Then, just type \verb|lb|.
\begin{verbatim}
$ lb
\end{verbatim}
Then, it runs latexmk and pdflatex and compile \verb|main.tex| with them.
Messages appear on your screen, and that shows the process of the compilation.
If there is a line 
\begin{verbatim}
Output written on _build/main.pdf (1 page, 19263 bytes).
\end{verbatim}
then the compilation completes correctly.
Check the directory.
\begin{verbatim}
$ ls -l
total 8
drwxrwxr-x 2 user user 4096 Dec  6 11:59 _build
-rw-rw-r-- 1 user user   72 Dec  6 11:59 main.tex
\end{verbatim}
A new directory \verb|_build| is generated.
Look at the files in the directory.
\begin{verbatim}
$ cd _build
$ ls
\end{verbatim}
There are auxliary files and the target file \verb|main.pdf|.
See \verb|main.pdf| with your pdf-viewer, for example evince.
\begin{verbatim}
$ evince main.pdf
\end{verbatim}
\begin{center}
\includegraphics[width=3cm]{hellolatex.png}
\end{center}

\subsection{Use lb.conf}
In the previous subsection, lb runs pdflatex.
The reason why lb chose pdflatex is the documentclass `article'.
It can also be compiled by lualatex or xelatex, but pdflatex has been a standard latex engine for ages.

If you want to use, for example, lualatex to compile, you need to specify it in \verb|lb.conf|.
This configuration file has six items.
\begin{description}
\item[rootfile] Rootfile is the main tex file, which usually includes {\textbackslash}begin\{document\} and {\textbackslash}end\{document\}. Other tex files are called `subfile'.
\item[builddir] This is a temporary directory includes all the auxliary files and the target file, which is usually a pdf file.
\item[engine] This specifies a latex engine, which is one of pdflatex, xelatex, lualatex, latex and platex.
\item[latex\_option] This specifies options to give \verb|latexmk|. The option `-halt-on-error' is given to \verb|latexmk| even if lb.conf doesn't exist.
\item[dvipdf] This is a program which converts dvi into pdf, which is used only with latex or platex. It is unnecessary with other latex engines. `dvipdfmx' is the best at present.
\item[preview] Pdf viewer. This is used to preview the pdf file when lb is given a subfile as an argument.
\end{description}

Run your editor, type the following and save it as the name \verb|lb.conf|.
\begin{verbatim}
rootfile=main
builddir=_build
engine=lualatex
latex_option=-halt-on-error
dvipdf=
preview=evince
\end{verbatim}
Then, type
\begin{verbatim}
$ lb
\end{verbatim}
it used lualatex to compile.

If you want to change the name of the tex file to `example.tex', then modify the first line in lb.conf to
\begin{verbatim}
rootfile=example
\end{verbatim}
or
\begin{verbatim}
rootfile=example.tex
\end{verbatim}
The suffix can be left out.

In addition, if you want to put all the axiliary files and the target file in the source directory, change the second line in lb.conf to:
\begin{verbatim}
builddir=
\end{verbatim}
This specifies null string for builddir item and that means no build directory is made.

Let's try to run \verb|lb| with the following \verb|lb.conf|.
\begin{verbatim}
rootfile=example
builddir=
engine=latex
latex_option=-halt-on-error
dvipdf=dvipdfmx
preview=evince
\end{verbatim}
Now, the engine is latex and dvipdf program is dvipdfmx.
\begin{verbatim}
$ rm -r _build
$ mv main.tex example.tex
$ lb
\end{verbatim}
Then messages appear.
It includes the following line.
\begin{verbatim}
This is pdfTeX, Version 3.14159265-2.6-1.40.21 (TeX Live 2020)
 (preloaded format=latex)
  ... ...
  ... ...
Output written on example.dvi (1 page, 332 bytes).
Transcript written on example.log.
Latexmk: Examining 'example.log'
=== TeX engine is 'pdfTeX'
Latexmk: Log file says output to 'example.dvi'
Latexmk: All targets (example.dvi) are up-to-date
example.dvi -> example.pdf
[1]
3662 bytes written
\end{verbatim}
This tells us that the engine was `latex'%
\footnote{
In Texlive2020, `latex' command calls `pdftex' instead of `tex' which is the original TeX program.
}.
It generates dvi file instead of pdf file.
After that, dvipdfmx is run by latexmk and it translates the dvi file into a pdf file.
The name `dvipdfmx' doesn't appear in the message but `example.dvi -{\textgreater} example.pdf' is outputed by dvipdfmx.
So we know that dvipdfmx was run by latexmk in the build process.
\begin{verbatim}
$ ls
example.aux  example.fdb_latexmk  example.log  example.tex
example.dvi  example.fls          example.pdf  lb.conf
\end{verbatim}
There's no temprary directory like \_build because we specified null string for builddir.

One of the important feature of lb is compiling a subfile separately.
This will be explained in the section  \ref{sec:testcompile} `Test compile' (p. \pageref{sec:testcompile}).

\section{Generate templates}
  \subsection{newtex.conf}
The script `newtex' makes a directory and generates template files in it.
This is used at the beginning of the work.

First, a configuration file `newtex.conf' needs to be made.
There is a template file included in the Buildtools source files.
\begin{verbatim}
# This is a configuration file for newtex.
# The name of this file is newtex.conf
# A string between # and new line is a commnet and it is ignored
 by newtex.
# Empty line is also ignored. 

# document name
title="Tutorial"

# lb.conf
# Lb.conf has six lines.
# The following six lines are copied to lb.conf.
rootfile=main.tex
builddir=_build
engine=pdflatex
latex_option=-halt-on-error
dvipdf=
preview=evince

# documentclass
documentclass=article

# chapters/sections and subfile names
#   Chapters/sections and subfile names must be surrounded by
 double quotes.
#   Subfile names have no suffix or ".tex" suffix.
# If your LaTeX file is not big and no subfile is necessary, then
 leave out the following lines.
section="Installation" "installation"
section="Run lb to compile tex files" "lb"
section="Generate templates" "generate_templates"   # Subfiles
 are NOT allowed to include space characters. Use underscore
 instead of space. 
section="Edit tex files" "edit_tex_files"
section="Test compile" "test_compile"
section="Preprocessing" "preprocessing"
section="Use rake" "rake"
section="Make tarball" "tarball"

\end{verbatim}
In this tutorial, I want to show you how to make this tutorial pdf file with Buildtools.
The file above is exactly the same as the newtex.conf file as I used to make it.

A string after hash mark (\#) in a line is comment and it is ignored by newtex.
Empty lines are also ignored.
The remaining lines are instructions to newtex.
Each line has a `key=value' structure.
The keys are:
\begin{description}
\item[title] The title of the document you make.
\item[rootfile] The name of the rootfile.
\item[builddir] The name of the build directory
\item[engine] A latex engine to compile source files
\item[latex\_option] The options you want to give to the latex engine
\item[dvipdf] A program that converts dvi to pdf.
\item[preview] A pdf viewer
\item[documentclass] The name of the documentclass you want to use
\item[chapter] Chapters and corresponding subfiles
\item[section] Sections and corresponding subfiles
\end{description}
If you make a book (big document) and use book documentclass, use `chapter' and `section' key.
If you make an article (small document) and use article documentclass, use `section' key only.

\subsection{Run newtex}
After you finish editing newtex.conf, just type:
\begin{verbatim}
$ newtex
\end{verbatim}
Then, newtex makes a directory of which the name is `Tutorial', which is the same as the title in newtex.conf.
If the title includes space characters, they are converted to underscore.
For example, a title `A tutorial for beginners' is converted to `A\_tutorial\_for\_beginners'.
This is because a file name includes space character sometimes causes problems.
Newtex also generates template files under the directory.
\begin{verbatim}
$ cd Tutorial
$ ls
Makefile            generate_templates.tex  main.tex
Rakefile            helper.tex              preprocessing.tex
cover.tex           installation.tex        rake.tex
edit_tex_files.tex  lb.conf                 tarball.tex
gecko.png           lb.tex                  test_compile.tex
\end{verbatim}

Look at some important files.
\begin{verbatim}
$ cat lb.conf
rootfile=main
builddir=_build
engine=pdflatex
latex_option=-halt-on-error
dvipdf=
preview=evince
\end{verbatim}
The contents of this file is the copy of the part of newtex.conf.
\begin{verbatim}
$ cat main.tex
\documentclass{article}
\input{helper.tex}
\title{Tutorial}
\author{} % Write your name if necessary.
\begin{document}
\maketitle
% If you want a table of contents here, uncomment the following
 line.
%\tableofcontents

\section{Installation}
  \input{installation.tex}
\section{Run lb to compile tex files}
  \input{lb.tex}
\section{Generate templates}
  \input{generate_templates.tex}
\section{Edit tex files}
  \input{edit_tex_files.tex}
\section{Test compile}
  \input{test_compile.tex}
\section{Preprocessing}
  \input{preprocessing.tex}
\section{Use rake}
  \input{rake.tex}
\section{Make tarball}
  \input{tarball.tex}
\end{document}
\end{verbatim}
The first line specifies a documentclass which is the same as the value of documentclass key in newtex.conf.
The second line has an input command which includes `helper.tex'.
Helper.tex has a role to include packages with \verb|\usepackage| command, define macros with \verb|\newcommand| command and so on.
Most of the lines in the preamble are described in helper.tex.
It is a good idea to make your own helper.tex because users often use the same preamble in different documents.
If you have your helper.tex, copy and overwrite this file.

You need to edit the fourth line.
For example,
\begin{verbatim}
\author{Toshio Sekiya}
\end{verbatim}
You can add `{\textbackslash}date', `{\textbackslash}thanks', `{\textbackslash}begin\{abstract\}' and `{\textbackslash}end\{abstract\}' if you like.
If you want to make a table of contents, then uncomment the eighth line.
After that, the lines are sections and {\textbackslash}input commands to include subfiles.

Rakefile contains instructions to rake.
You don't need to modify it so far.
Try to use it.
\begin{verbatim}
$ rake
 ... ...
 ... ...
$ ls
Makefile            gecko.png               main.tex
Rakefile            generate_templates.tex  preprocessing.tex
Tutorial.pdf        helper.tex              rake.tex
_build              installation.tex        tarball.tex
cover.tex           lb.conf                 test_compile.tex
edit_tex_files.tex  lb.tex
$ ls _build
main.aux  main.fdb_latexmk  main.fls  main.log  main.out  main.pdf
$ evince Tutorial.pdf
\end{verbatim}
Rake ran lb to compile main.tex and after that it copied \_build/main.pdf to Tutorial.pdf.
Evince shows Tutorial.pdf as follows.
\begin{center}
\includegraphics[width=8cm]{Tutorial_1.png}
\end{center}


\section{Edit tex files}
  There are eight sections and subfiles.
Each subfile is empty just after it is generated by newtex.
Editing subfiles is the main work and you need to allocate most of your time to this work.

The first section and subfile are `Installation' and installation.tex respectively.
Maybe you edit a part of the section and test-compile to see how it looks like in the pdf file.
Usually we come and go between editing and test-compiling repeatedly.

If the document is not so big, using rake is the best to test-compile, because it doesn't take much time and the pdf shows the whole document.
This tutorial is rather a small document, so using rake for test-compiling is fine.

\begin{verbatim}
\subsection{Prerequisite}
Buildtools requires the following items.
\begin{enumerate}
\item Linux OS and bash
\item LaTeX system
\item Make or Rake
\end{enumerate}
 ... ...
 ... ...
\end{verbatim}

Then, type the following line to see the pdf.
\begin{verbatim}
$ rake
$ evince Tutorial.pdf
\end{verbatim}

\begin{center}
\includegraphics[width=8cm]{Tutorial_2.png}
\end{center}

\section{Test compile}
  もし大きな文書、例えば100ページを越える書籍を作るような場合、テスト・コンパイルにrakeを使う(rakeで文書全体をコンパイルするという意味)のは良い方法ではない。
ドキュメントが大きくなればなるほど、コンパイル時間が長くなるからである。

それよりは、lbを使ってサブファイルを単独でコンパイルするのが良い方法である。
lbは一時的なルートファイル(テンポラリ・ルートファイルという)を作り、その中にサブファイルを取り込む命令を記述し、1度だけコンパイルする。
この方法の良い点は短時間でコンパイルできることである。
しかし、良くない点もある。
サブファイルのみのコンパイルであるから、文書全体のpdfを見ることはできない。
また、1度だけのコンパイルなので、相互参照は反映されない。
どちらが良いかは一概には言えない。
それは作成している文書によるが、もしそれが非常に大きな文書であれば、テスト・コンパイルをlbでやるほうがより良いといえる。

次のようにタイプしてほしい。
\begin{verbatim}
$ lb installation
\end{verbatim}
この引数はサブファイル名である。
拡張子は省略することもできる。

すると、lbはテンポラリ・ルートファイル「\_build/test\_installation.tex」を作る。
そのプリアンブルは元のルートファイルのプリアンブルのコピーである。
そして、{\textbackslash}inputコマンドでサブファイルの「installation.tex」を取り込むようになっている。
lbは「\_build/test\_installation.tex」をコンパイルし、lb.confで指定されたpdfビューワを起動してpdfファイルを表示する。
\begin{center}
\includegraphics[width=12cm]{test_installation.png}
\end{center}

lbはコンパイルするときにsynctexのオプションをオンにする。
したがって、ソースファイルの「\_build/test\_installation.tex」と「installation.tex」をエディタで開いておけば、ソースとpdfの間で前方参照と後方参照をすることができる。
もしもgeditをエディタに、evinceをビューワに使っていれば、コントロール・キーを押したまま左クリックすれば後方参照(pdfからソースへの参照)が可能である。
しかし、「installation.tex」からpdfへの前方参照は働かない。
それを機能させるためには、サブファイルの先頭に次の1行を加えなければならない。
\begin{verbatim}
% mainfile: _build/test_installation.tex
\end{verbatim}
しかし、前方参照を使うことは後方参照に比べそう多くはない。
上記の1行を加えるのは、たいていは必要ないだろう。


\section{Preprocessing}
  ときにはlatexソースファイルをコンパイルする前に何かしておきたい、ということがあるかもしれない。
例えば、
\begin{itemize}
\item gnuplotなどのプログラムを使って画像ファイルを生成したい
\item pandocを使ってlatexソースファイルを自動生成したい
\end{itemize}

このチュートリアルではプリプロセッサとしてpandocを使う方法を紹介したい。
pandocは文書コンバータである。
markdown、latex、html、pdfなど多くの種類の文書を変換することができる。
このチュートリアルでは、Buildtoolsのソースファイルの中にある「Readme.ja.md」を「readme.tex」というlatexソースファイルに変換してみる。

たいていのディストリビューションではpandocパッケージが備わっているので、インストールは簡単である。
例えばubuntuでは
\begin{verbatim}
$ sudo apt-get install pandoc
\end{verbatim}
でインストールできる。

readme.texを生成するには次のようにタイプする。
\begin{verbatim}
$ pandoc -o readme.tex ../Readme.ja.md
\end{verbatim}

この他に2つほどmain.texとhelper.texに変更を加える必要がある。
まず、{\textbackslash}inputコマンドを用いてreadme.texを取り込む命令をmain.texの最後に記述する。
\begin{verbatim}
\documentclass{article}
\input{helper.tex}
 ... ...
 ... ...
\section{Make tarball}
  \input{tarball.tex}
\input{readme.tex}
\end{document}
\end{verbatim}
更に、8行目の{\textbackslash}tableofcontentsをアンコメントしてコマンドが利くようにしよう。
\begin{verbatim}
 ... ...
\maketitle
% If you want a table of contents here, uncomment the following
 line.
\tableofcontents
 ... ...
\end{verbatim}

2番めは、{\textbackslash}tightlistのマクロを定義することが必要である。
Helper.texがその定義を書くのに最も適した場所である。
\begin{verbatim}
 ... ...
 ... ...
\providecommand{\tightlist}{%
  \setlength{\itemsep}{0pt}\setlength{\parskip}{0pt}}
 ... ...
 ... ...
\end{verbatim}
このコードは\url{https://github.com/jgm/pandoc-templates/blob/master/default.latex}から引用したものである。

rakeを使ってコンパイルする。
\begin{verbatim}
$ rake
\end{verbatim}

\begin{center}
\includegraphics[width=12cm]{tableofcontents.png}
\end{center}

これで、目次にセクション9、そして「Readme.ja.md」の内容が表示された。

このセクションではpandocを手動で走らせた。
もしも、Readme.ja.md がアップグレードされたならば、再びpandocを実行しなければならない。
それは面倒なことであり、本来自動化されるべきことである。
ひとつの方法はRakefileを変更してrakeがコンパイル前に自動的にプリプロセッシングするようにすることである。
次のセクションでその方法を説明する。


\section{Use rake}
  Rake is a build tool similar to make.
Rakefile describes instructions for rake to build source files.
You can write any ruby commands in Rakefile.
Therefore, it has a high ability to describe the build process even if it is complicated, .

Newtex generates a Rakefile, which is enough to compile the source files if there is no preprocessing procedure.
In the previous section, we used pandoc to generate readme.tex.
So, we need to modify Rakefile to put in pandoc.
Modify the Rakefile as follows.
\begin{verbatim}
require 'rake/clean'

# if readme.tex doesn't exist, generate it first.
# This is necessary because readme.tex is accessed by gfiles in
#  line 12.
if File.exist?("readme.tex") == false
  sh "pandoc -o readme.tex ../Readme.md"
end
# use Latex-BuildTools
@tex_files = (`tfiles -a` + `tfiles -p`).split("\n")
@tex_files <<= "readme.tex"
@graphic_files = []
@tex_files.each do |file|
  @graphic_files += `gfiles #{file}`.split("\n")
end

task default: "Tutorial.pdf"

file "Tutorial.pdf" => "_build/main.pdf" do
  sh "cp _build/main.pdf Tutorial.pdf"
end

file "_build/main.pdf" => (@tex_files+@graphic_files) do
  sh "lb main.tex"
end

file "readme.tex" => "../Readme.md" do
  sh "pandoc -o readme.tex ../Readme.md"
end

CLEAN << "_build"
task :clean

task :ar do
  sh "arl main.tex"
  sh "tar -rf main.tar Rakefile"
  sh "gzip main.tar"
  sh "mv main.tar.gz Tutorial.tar.gz"
end

task :zip do
  sh "arl -z main.tex"
  sh "zip main.zip Rakefile"
  sh "mv main.zip Tutorial.zip"
end
\end{verbatim}

Thanks to this modification, you don't need to run pandoc by hand.
What you need is just type `rake'.

There are websites about ruby and rake.
For example,
\begin{itemize}
\item \url{https://www.ruby-lang.org/en/}
\item \url{http://rubylearning.com/}
\item \url{https://ruby.github.io/rake/}
\end{itemize}

The tutorial finishes at this section.
Next section is the copy of Readme.md in Buildtools source files.
It describes the background of Buildtools and features of each script.

\section{Make tarball}
  ソースファイルを配布したい、という場合もあるかもしれない。
そのようなときには、それをアーカイブすることが必要になる。
Buildtoolsに含まれるスクリプトのarlは、ルートファイルが取り込むサブファイルや画像ファイルを検索し、それらアーカイブする。
このアーカイブされたファイルのうち、tarというコマンドで作られたものをtarballという。
\begin{itemize}
\item -gオプションが与えられると、gzipで圧縮されたtarballを作る。
\item -bオプションが与えられると、bzip2で圧縮されたtarballを作る。
\item -zオプションが与えられると、zipファイルを作る。
\item オプションが与えられなければ、非圧縮のtarballを作る。
\end{itemize}

もし、プリプロセッシングで生成されるlatexソースファイルなどがある場合は、arlの実行前にそれらを生成しておかなければならない。
\begin{verbatim}
$ arl
$ tar -tf main.tar
main.tex
edit_tex_files.tex
generate_templates.tex
installation.tex
lb.tex
preprocessing.tex
rake.tex
readme.tex
tarball.tex
test_compile.tex
helper.tex
Tutorial_1.png
Tutorial_2.png
hellolatex.png
tableofcontents.png
test_installation.png
\end{verbatim}

Rakefileも当然ながら、tarballに含めなければならない。
\begin{verbatim}
$ tar -rf main.tar Rakefile
\end{verbatim}
そして、gzipなどに圧縮してtarballが完成する。
\begin{verbatim}
$ gzip main.tar
\end{verbatim}

以上の手続きは、実はすでにRakefileに記述されている。
「rake ar」とタイプすることにより、rakeがtarballを自動生成してくれる。
\begin{verbatim}
$ rm main.tar.gz
$ rake ar
arl main.tex
tar -rf main.tar Rakefile
gzip main.tar
mv main.tar.gz チュートリアル.tar.gz
\end{verbatim}
最後にできあがるtarballの名前は「チュートリアル.tar.gz」である。

もしも、zipファイルを作りたければ、「rake zip」とタイプする。

このセクションで、チュートリアルは終わりである。
次のセクションはBuildtoolsのソースファイルに含まれるReadme.ja.mdのコピーである。
この文書には、Buildtoolsを作成した背景や個々のスクリプトの特長と使い方が書かれている。


\end{document}
\end{verbatim}
The first line specifies a documentclass which is the same as the value of documentclass key in newtex.conf.
The second line has an input command which includes `helper.tex'.
Helper.tex has a role to include packages with \verb|\usepackage| command, define macros with \verb|\newcommand| command and so on.
Most of the lines in the preamble are described in helper.tex.
It is a good idea to make your own helper.tex because users often use the same preamble in different documents.
If you have your helper.tex, copy and overwrite this file.

You need to edit the fourth line.
For example,
\begin{verbatim}
\author{Toshio Sekiya}
\end{verbatim}
You can add `{\textbackslash}date', `{\textbackslash}thanks', `{\textbackslash}begin\{abstract\}' and `{\textbackslash}end\{abstract\}' if you like.
If you want to make a table of contents, then uncomment the eighth line.
After that, the lines are sections and {\textbackslash}input commands to include subfiles.

Rakefile contains instructions to rake.
You don't need to modify it so far.
Try to use it.
\begin{verbatim}
$ rake
 ... ...
 ... ...
$ ls
Makefile            gecko.png               main.tex
Rakefile            generate_templates.tex  preprocessing.tex
Tutorial.pdf        helper.tex              rake.tex
_build              installation.tex        tarball.tex
cover.tex           lb.conf                 test_compile.tex
edit_tex_files.tex  lb.tex
$ ls _build
main.aux  main.fdb_latexmk  main.fls  main.log  main.out  main.pdf
$ evince Tutorial.pdf
\end{verbatim}
Rake ran lb to compile main.tex and after that it copied \_build/main.pdf to Tutorial.pdf.
Evince shows Tutorial.pdf as follows.
\begin{center}
\includegraphics[width=8cm]{Tutorial_1.png}
\end{center}


\section{Edit tex files}
  There are eight sections and subfiles.
Each subfile is empty just after it is generated by newtex.
Editing subfiles is the main work and you need to allocate most of your time to this work.

The first section and subfile are `Installation' and installation.tex respectively.
Maybe you edit a part of the section and test-compile to see how it looks like in the pdf file.
Usually we come and go between editing and test-compiling repeatedly.

If the document is not so big, using rake is the best to test-compile, because it doesn't take much time and the pdf shows the whole document.
This tutorial is rather a small document, so using rake for test-compiling is fine.

\begin{verbatim}
\subsection{Prerequisite}
Buildtools requires the following items.
\begin{enumerate}
\item Linux OS and bash
\item LaTeX system
\item Make or Rake
\end{enumerate}
 ... ...
 ... ...
\end{verbatim}

Then, type the following line to see the pdf.
\begin{verbatim}
$ rake
$ evince Tutorial.pdf
\end{verbatim}

\begin{center}
\includegraphics[width=8cm]{Tutorial_2.png}
\end{center}

\section{Test compile}
  もし大きな文書、例えば100ページを越える書籍を作るような場合、テスト・コンパイルにrakeを使う(rakeで文書全体をコンパイルするという意味)のは良い方法ではない。
ドキュメントが大きくなればなるほど、コンパイル時間が長くなるからである。

それよりは、lbを使ってサブファイルを単独でコンパイルするのが良い方法である。
lbは一時的なルートファイル(テンポラリ・ルートファイルという)を作り、その中にサブファイルを取り込む命令を記述し、1度だけコンパイルする。
この方法の良い点は短時間でコンパイルできることである。
しかし、良くない点もある。
サブファイルのみのコンパイルであるから、文書全体のpdfを見ることはできない。
また、1度だけのコンパイルなので、相互参照は反映されない。
どちらが良いかは一概には言えない。
それは作成している文書によるが、もしそれが非常に大きな文書であれば、テスト・コンパイルをlbでやるほうがより良いといえる。

次のようにタイプしてほしい。
\begin{verbatim}
$ lb installation
\end{verbatim}
この引数はサブファイル名である。
拡張子は省略することもできる。

すると、lbはテンポラリ・ルートファイル「\_build/test\_installation.tex」を作る。
そのプリアンブルは元のルートファイルのプリアンブルのコピーである。
そして、{\textbackslash}inputコマンドでサブファイルの「installation.tex」を取り込むようになっている。
lbは「\_build/test\_installation.tex」をコンパイルし、lb.confで指定されたpdfビューワを起動してpdfファイルを表示する。
\begin{center}
\includegraphics[width=12cm]{test_installation.png}
\end{center}

lbはコンパイルするときにsynctexのオプションをオンにする。
したがって、ソースファイルの「\_build/test\_installation.tex」と「installation.tex」をエディタで開いておけば、ソースとpdfの間で前方参照と後方参照をすることができる。
もしもgeditをエディタに、evinceをビューワに使っていれば、コントロール・キーを押したまま左クリックすれば後方参照(pdfからソースへの参照)が可能である。
しかし、「installation.tex」からpdfへの前方参照は働かない。
それを機能させるためには、サブファイルの先頭に次の1行を加えなければならない。
\begin{verbatim}
% mainfile: _build/test_installation.tex
\end{verbatim}
しかし、前方参照を使うことは後方参照に比べそう多くはない。
上記の1行を加えるのは、たいていは必要ないだろう。


\section{Preprocessing}
  ときにはlatexソースファイルをコンパイルする前に何かしておきたい、ということがあるかもしれない。
例えば、
\begin{itemize}
\item gnuplotなどのプログラムを使って画像ファイルを生成したい
\item pandocを使ってlatexソースファイルを自動生成したい
\end{itemize}

このチュートリアルではプリプロセッサとしてpandocを使う方法を紹介したい。
pandocは文書コンバータである。
markdown、latex、html、pdfなど多くの種類の文書を変換することができる。
このチュートリアルでは、Buildtoolsのソースファイルの中にある「Readme.ja.md」を「readme.tex」というlatexソースファイルに変換してみる。

たいていのディストリビューションではpandocパッケージが備わっているので、インストールは簡単である。
例えばubuntuでは
\begin{verbatim}
$ sudo apt-get install pandoc
\end{verbatim}
でインストールできる。

readme.texを生成するには次のようにタイプする。
\begin{verbatim}
$ pandoc -o readme.tex ../Readme.ja.md
\end{verbatim}

この他に2つほどmain.texとhelper.texに変更を加える必要がある。
まず、{\textbackslash}inputコマンドを用いてreadme.texを取り込む命令をmain.texの最後に記述する。
\begin{verbatim}
\documentclass{article}
% helper.tex

% 注意
% graphicxのドライバーがluatexになっている。他のドライバーを使う場合は書き換えが必要。

% パッケージのとりこみ
\usepackage{amsmath,amssymb}
\usepackage[luatex]{graphicx}
\usepackage{tikz}
\usetikzlibrary{calc}
\usetikzlibrary{topaths}
\usetikzlibrary{plotmarks}
\usetikzlibrary{intersections}
\usetikzlibrary{arrows,decorations.pathmorphing,backgrounds,positioning,fit,petri}
\usetikzlibrary{arrows.meta}
\usetikzlibrary{3d}
%\usepackage{gnuplot-lua-tikz}

\usepackage{fancybox}
\usepackage{booktabs}

\usepackage[margin=2.4cm]{geometry}

\usepackage[colorlinks=true,linkcolor=black,pdfencoding=auto]{hyperref}

% マクロのサンプル(解答、証明、q.e.d.、解答)
%\newcommand{\solution}{\begin{flushleft}\textbf{解:}\end{flushleft}}
%\newcommand{\proof}{\begin{flushleft}\textbf{証明}\end{flushleft}}
%\newcommand{\qed}{\begin{flushright}\textbf{証明終}\end{flushright}}

% 定理環境(定理、補題、系、定義、例、練習問題)
\newtheorem{theorem}{定理}[section]
\newtheorem{lemma}{補題}[section]
\newtheorem{corollary}{系}[section]
\newtheorem{definition}{定義}[section]
\newtheorem{example}{例}[section]
\newtheorem{exercise}{問題}[section]

% 凹凸増減表の矢印 ----------------------------------------------------------------------------
% concave north east arrow 上に凸で増加
\newcommand{\ccnearrow}{
\begin{tikzpicture}
  \draw[very thin,->] (0,0) .. controls (0,0.2) and (0.05,0.25) .. (0.25,0.25);
\end{tikzpicture}
}
% concave south east arrow 上に凸で減少
\newcommand{\ccsearrow}{
\begin{tikzpicture}
  \draw[very thin,->] (0,0) .. controls (0.2,0) and (0.25,-0.05) .. (0.25,-0.25);
\end{tikzpicture}
}
% convex north east arrow 下に凸で増加
\newcommand{\cvnearrow}{
\begin{tikzpicture}
  \draw[very thin,->] (0,0) .. controls (0.2,0) and (0.25,0.05) .. (0.25,0.25);
\end{tikzpicture}
}
% convex south east arrow 下に凸で減少
\newcommand{\cvsearrow}{
\begin{tikzpicture}
  \draw[very thin,->] (0,0) .. controls (0,-0.2) and (0.05,-0.25) .. (0.25,-0.25);
\end{tikzpicture}
}
%-----------------------------------------------------------------------------------------------

% for long division (割り算の筆算のためのマクロ、Tikzの中で使う)=======================================================
% \divisoroffset is the distance between x-axis 0 and the center of the divisor. It is negative number.
\newcommand{\divisor}[2]{\node[anchor=base] at (#1,0) {#2}}
% dividendoffset is the offset from x-axis 0 to the first term bound of the dividend.
\newcommand{\dividendoffset}{0.8}
% \termwidth is the width between the consecutive term.
\newcommand{\termwidth}{0.9}
% expressionheight is the height of each expression.
\newcommand{\expressionheight}{0.5}
\newcommand{\pt}[3]{\node[base left] at (\dividendoffset+\termwidth*#1,\expressionheight*#2) {#3}}
\newcommand{\hseparator}[3]{\draw ({\dividendoffset+(#1-1)*\termwidth},{-(#3+0.3)*\expressionheight})
                            -- ({\dividendoffset+(#2)*\termwidth},{-(#3+0.3)*\expressionheight})}
\newcommand{\divisionbox}[1]{\draw (0,\expressionheight*0.7) -- (\dividendoffset+\termwidth*#1,\expressionheight*0.7)
                                   (0,\expressionheight*0.7) arc[start angle=27,end angle=-27,radius=0.5]}
%=========================================================================================================================

 ... ...
 ... ...
\section{Make tarball}
  ソースファイルを配布したい、という場合もあるかもしれない。
そのようなときには、それをアーカイブすることが必要になる。
Buildtoolsに含まれるスクリプトのarlは、ルートファイルが取り込むサブファイルや画像ファイルを検索し、それらアーカイブする。
このアーカイブされたファイルのうち、tarというコマンドで作られたものをtarballという。
\begin{itemize}
\item -gオプションが与えられると、gzipで圧縮されたtarballを作る。
\item -bオプションが与えられると、bzip2で圧縮されたtarballを作る。
\item -zオプションが与えられると、zipファイルを作る。
\item オプションが与えられなければ、非圧縮のtarballを作る。
\end{itemize}

もし、プリプロセッシングで生成されるlatexソースファイルなどがある場合は、arlの実行前にそれらを生成しておかなければならない。
\begin{verbatim}
$ arl
$ tar -tf main.tar
main.tex
edit_tex_files.tex
generate_templates.tex
installation.tex
lb.tex
preprocessing.tex
rake.tex
readme.tex
tarball.tex
test_compile.tex
helper.tex
Tutorial_1.png
Tutorial_2.png
hellolatex.png
tableofcontents.png
test_installation.png
\end{verbatim}

Rakefileも当然ながら、tarballに含めなければならない。
\begin{verbatim}
$ tar -rf main.tar Rakefile
\end{verbatim}
そして、gzipなどに圧縮してtarballが完成する。
\begin{verbatim}
$ gzip main.tar
\end{verbatim}

以上の手続きは、実はすでにRakefileに記述されている。
「rake ar」とタイプすることにより、rakeがtarballを自動生成してくれる。
\begin{verbatim}
$ rm main.tar.gz
$ rake ar
arl main.tex
tar -rf main.tar Rakefile
gzip main.tar
mv main.tar.gz チュートリアル.tar.gz
\end{verbatim}
最後にできあがるtarballの名前は「チュートリアル.tar.gz」である。

もしも、zipファイルを作りたければ、「rake zip」とタイプする。

このセクションで、チュートリアルは終わりである。
次のセクションはBuildtoolsのソースファイルに含まれるReadme.ja.mdのコピーである。
この文書には、Buildtoolsを作成した背景や個々のスクリプトの特長と使い方が書かれている。


\hypertarget{latex-buildtools}{%
\section{Latex-Buildtools}\label{latex-buildtools}}

\hypertarget{latex-buildtoolsux4f5cux6210ux306eux80ccux666f}{%
\subsection{Latex-Buildtools作成の背景}\label{latex-buildtoolsux4f5cux6210ux306eux80ccux666f}}

\hypertarget{buildtoolsux306flatexux3067ux5927ux304dux306aux6587ux66f8ux3092ux4f5cux308dux3068ux304dux306bux7528ux3044ux308bux30c4ux30fcux30ebux306eux3072ux3068ux3064}{%
\paragraph{BuildtoolsはLaTeXで大きな文書を作ろときに用いるツールのひとつ}\label{buildtoolsux306flatexux3067ux5927ux304dux306aux6587ux66f8ux3092ux4f5cux308dux3068ux304dux306bux7528ux3044ux308bux30c4ux30fcux30ebux306eux3072ux3068ux3064}}

大きな文書、例えば100ページを越える本などをLaTeXで作る場合は、小さい文書の作成と異なる様々な問題を考える必要がある。

\begin{itemize}
\tightlist
\item
  ソースファイルの分割
\item
  分割コンパイル
\item
  文書の一括置換
\item
  前処理(LaTeXコンパイル前に処理する作業)
\end{itemize}

これらを支援する2つのツールがある。

\begin{itemize}
\tightlist
\item
  Latex-Buildtools、または単にBuildtools。ソースファイルの新規作成、ビルド、分割コンパイルを支援するツール群
\item
  Latex-Substools、または単にSubstools。ソースファイルに対する一括置換をするツール群
\end{itemize}

Buildtoolsは上記の作業の中でその中核をなすツール群である。
このドキュメントではBuildtoolsを解説する。

\hypertarget{ux30bdux30fcux30b9ux30d5ux30a1ux30a4ux30ebux306eux5206ux5272}{%
\paragraph{ソースファイルの分割}\label{ux30bdux30fcux30b9ux30d5ux30a1ux30a4ux30ebux306eux5206ux5272}}

LaTeXソースファイルを単にここではソースファイルと呼ぶ。
大きな文書を1つのソースファイルに記述するのは適切でない。
なぜなら、ファイルが大きくなると、エディタで編集するのが極めて困難になるからである。
そこで、文書を分割することになる。
通常は\texttt{\textbackslash{}begin\{document\}}と\texttt{\textbackslash{}end\{document\}}を含む1つのファイルと、そのファイルから\texttt{\textbackslash{}include}または\texttt{\textbackslash{}input}で呼び出される複数のファイルに分割する。
前者をルートファイル、後者をサブファイルという。
\texttt{\textbackslash{}include}と\texttt{\textbackslash{}input}はどちらもサブファイルの取り込みのコマンドだが、違いがある。

\begin{itemize}
\tightlist
\item
  \texttt{\textbackslash{}include}はネストはできない。\texttt{\textbackslash{}include}はボディ(\texttt{\textbackslash{}begin\{document\}}と\texttt{\textbackslash{}end\{document\}}の間の部分)にのみ書くことができる。\texttt{\textbackslash{}includeonly}でファイルのリストが指定されている場合は、そのリストに書かれているファイルのみが\texttt{\textbackslash{}include}で取り込め、リストにないファイルの\texttt{\textbackslash{}include}は無視される。\texttt{\textbackslash{}includeonly}はプリアンブルのみに書くことができる。\texttt{\textbackslash{}include}はファイルを取り込む前に\texttt{\textbackslash{}clearpage}をする。
\item
  \texttt{\textbackslash{}input}は単にファイルを取り込むだけで\texttt{\textbackslash{}clearpage}はしない。このコマンドはネストできる。
\end{itemize}

文書をコンパイルによって作成することをビルドと呼ぶ。
ビルドはLaTeXのコンパイルだけでなく、前処理(例えばgnuplotによる画像生成だったり、データからtikzのグラフを生成したりなど)も含める。
ビルドは最終的にルートファイルをコンパイルすることによって完了する。
LaTeXソースファイルをコンパイルするプログラムにはいくつかあり、それをエンジンと呼んでいる。
Buildtoolsでサポートしているエンジンには、latex、platex、pdflatex、xelatex、lualatexがある。
コンパイルは文書が大きくなればなるほど時間がかかるようになる。
1箇所だけを変更する場合も同じだけ時間がかかる。
文書の編集の途中で出来栄えをチェックするためにコンパイルすること(これをテストするなどということがある)は頻繁に起こるが、そのたびに長い時間がかかる。
文書が大きくなればなるほどこの問題は深刻になる。
そこで、サブファイルのみをコンパイルできるようにする方法がいろいろ考えられた。

\begin{itemize}
\tightlist
\item
  \texttt{\textbackslash{}includeonly}コマンドの引数(サブファイルのリスト)からコンパイルしたくないファイルをコメントアウトする
\item
  subfilesパッケージを用いる
\end{itemize}

とくにsubfilesパッケージはよくできており、推薦する方も多い。
ただ、パッケージを取り込み、そのコマンドを適切に使うことが必要である。
もちろん、上記のコンパイル時間の問題からすれば、それくらいは何ともないことであるが。

これ以外の方法としては、サブファイル単独のコンパイル用に特別のプリアンブル部などを付け加える方法がある。
具体的には、サブファイルを「\texttt{\textbackslash{}documentclass}から\texttt{\textbackslash{}begin\{document\}}まで」と「\texttt{\textbackslash{}end\{document\}}」でサンドイッチにする。
そのときにサブファイル自体を変更するのではなく、プリアンブルなどを記述したファイルからサブファイルを\texttt{\textbackslash{}input}で取り込むようにする。
そのサブファイルを取り込むファイルを「仮のルートファイル」と呼んだりする。
それに対して元々のルートファイルを「オリジナルのルートファイル」と呼ぶこともある。
仮のルートファイルのプリアンブル部は、オリジナルのルートファイルのコピーである。
この方法の良い点は

\begin{itemize}
\tightlist
\item
  ソースファイルにパッケージの取り込みや特別な命令を書き込む必要がない。
\item
  したがって、ソースファイルを配布するにあたって、被配布者に特定のパッケージをインストールさせる必要がない。
\end{itemize}

つまり、ソースファイルをなんら変更することなく、サブファイルのみのコンパイルが可能だというのが長所である。
ただ、そのためには仮のルートファイルを生成するプログラムが必要である。
Buildtoolsでは、\texttt{ttex}というシェルスクリプトでそれを行っている。

\hypertarget{ux30b3ux30f3ux30d1ux30a4ux30ebux306eux30eaux30d4ux30fcux30c8ux7e70ux308aux8fd4ux3057}{%
\paragraph{コンパイルのリピート(繰り返し)}\label{ux30b3ux30f3ux30d1ux30a4ux30ebux306eux30eaux30d4ux30fcux30c8ux7e70ux308aux8fd4ux3057}}

最終的に文書をコンパイルするときには、相互参照などを実現するためにコンパイルを複数回行わなければならない。
その回数は2回であったり、3回であったりするらしいが、筆者はその事情をよく知らない。
が、ここに優れたプログラムがあり、必要回数を判断し、必要なだけコンパイルしてくれる。
latexmkというプログラムである。
latexmkを用いることによって、ビルドは非常に楽になる。
Buildtoolsではルートファイルのコンパイルにlatexmkを使用する。

\hypertarget{ux4f5cux696dux7528ux30c7ux30a3ux30ecux30afux30c8ux30eaux306eux8a2dux7f6e}{%
\paragraph{作業用ディレクトリの設置}\label{ux4f5cux696dux7528ux30c7ux30a3ux30ecux30afux30c8ux30eaux306eux8a2dux7f6e}}

LaTeXでコンパイルすると、様々な補助ファイルやログファイルが生成されるので、きれい好きの人はそれについて不満を感じることがあると思う。
それで、作業用ディレクトリを設置して、そういうファイルの一切合財を入れてしまうとソースディレクトリをきれいに保つことができる。
そういうソフトにcluttexがあり、きれい好きな人にはお勧めである。
Buildtoolsでは作業用ディレクトリ\texttt{\_build}(名前は変更も可能)を設置し、補助ファイルなどを格納する。
このことにより、ソースディレクトリを汚さずに済む。
また、コンパイルのログや補助ファイルを見たいときは\texttt{\_build}の中を見れば良い。
非常に単純なのである。
蛇足になるが、最近Cのビルドツールとして人気のあるmesonも作業ディレクトリを使う。
これは、一般に作業用ディレクトリをソースディレクトリと区別することが人間にとって非常に分かりやすくなるということの表れである。

Buildtoolsでは、生成された最終文書(pdfファイル)も作業用ディレクトリにできあがる。
それをソースファイルディレクトリに置きたいというのは自然は発想だが、それにはmakeまたはrakeを使うと良い。
rakeはスクリプト言語rubyで書かれたmakeとでもいうべきものであるが、特長はRakefile(rakeのスクリプト)にruby言語を使うことができることである。
そのことによって、makeよりもはるかに強力で分かりやすい記述ができる。
話を元に戻すが、最終文書をソースディレクトリに置くには、MakefileまたはRakefileに、作業用ディレクトリからソースディレクトリに最終文書をコピーするコマンドを書いておくのである。
また、makeやrakeを使うことの利点は、前処理を記述できることである。
前処理はその文書やユーザの使うツールに依存するので、Buildtoolsでカバーするのは難しい。
それに比べて、MakefileやRakefileはフレキシブルなので、前処理を記述するのに適しているのである。

Buildtoolsでは、makeまたはrakeを併用することを推奨している。

\hypertarget{texworksux3068ux306eux9023ux643a}{%
\paragraph{Texworksとの連携}\label{texworksux3068ux306eux9023ux643a}}

Buildtoolsでは、\texttt{lb}というコマンドでルートファイルのビルドもサブファイルのテストコンパイルも行うことができる。
それをTexworksのタイプセッティングに登録すると、Texworksから起動できて大変便利である。
「設定」ー>「タイプセッティング」タブで設定する。
+をクリックして新たなコマンドを設定する。名前=\textgreater{}\texttt{lb}、コマンド=\textgreater{}\texttt{lb}、引数=\textgreater{}\texttt{\$fullname}で良い。
設定後はルートファイルで全体のコンパイル、サブファイルは単独のテスト・コンパイルがワンクリックでできるようになる。

\hypertarget{buildtoolsux306eux69cbux6210}{%
\subsection{Buildtoolsの構成}\label{buildtoolsux306eux69cbux6210}}

Buildtoolsは大きく分けて次のような部分から構成されている。

\begin{itemize}
\tightlist
\item
  \texttt{newtex}: 新規ソースファイルの作成を支援するツール。
\item
  \texttt{lb}: Latex
  Build。ルートファイルをコンパイル、または\texttt{ttex}を使ってサブファイルのテストコンパイルをする。
\item
  アーカイブ作成を支援するツール。
\item
  インストーラ
\item
  ユーティリティ群。上記のプログラムを下支えする。
\end{itemize}

文書作成は、次の手順で行われる。

\begin{enumerate}
\def\labelenumi{\arabic{enumi}.}
\tightlist
\item
  文書全体の構成を決める。章立てを決める。
\item
  \texttt{newtex}を使ってフォルダとソースファイルの雛形を作成する。
\item
  \texttt{Makefile}、\texttt{Rakefile}、表紙(\texttt{cover.tex})、プリアンブル部(\texttt{helper.tex})の雛形を必要に応じて書き換える。
\item
  本文の作成と、テストコンパイル。
\item
  前処理の作成。
\item
  最終ビルド。
\end{enumerate}

上記の3から5は行ったり来たりすることになり、必ずしもこの順に作業が進むわけではない。

\hypertarget{ux4e3bux8981ux306aux30c4ux30fcux30ebux306bux3064ux3044ux3066}{%
\subsection{主要なツールについて}\label{ux4e3bux8981ux306aux30c4ux30fcux30ebux306bux3064ux3044ux3066}}

それぞれのツールの簡単なヘルプは\texttt{-\/-help}オプションをつけて実行することにより表示される。
例えば、\texttt{newtex}に\texttt{-\/-help}オプションをつけて実行すると、次のようなメッセージが表示される。

\begin{verbatim}
$ newtex --help
Usage:
  newtex --help
    Show this message.
  Newtex.conf needs to be edited before running newtex.
  newtex
    A directory is made and some template files are generated under the directory.
\end{verbatim}

各ツールを説明する文書は

\begin{itemize}
\tightlist
\item
  各ツールの\texttt{-\/-help}オプションによるヘルプ・メッセージ
\item
  このドキュメントにおける以下の記述
\end{itemize}

だけである。 より詳細を知りたい場合はソースコードを見ていただきたい。
Buildtoolsのすべてのツールはシェル・スクリプトで書かれている。
それぞれのスクリプトは短く、シェル・スクリプトをご存知の方であれば、比較的簡単にソースコードを理解できる。

\hypertarget{newtex}{%
\paragraph{newtex}\label{newtex}}

\begin{verbatim}
$ newtex
\end{verbatim}

新規にLaTeXの文書を作るときに使うスクリプト。
\texttt{newtex}を使う前に全体の構成と章立てを決めておき、それを事前に\texttt{newtex.conf}に書いておく。
このスクリプトは、\texttt{newtex.conf}に書かれた指示に従って、新しくディレクトリを作り、テンプレート・ファイルを生成する。

このプログラムは2回に分けて使う。

\begin{enumerate}
\def\labelenumi{\arabic{enumi}.}
\tightlist
\item
  Buildtoolsのソースファイルの中に\texttt{newtex.conf}ファイルがある。
  これを書き直して、ユーザの環境やこれから作るLaTeXファイルに合うようにする。
\item
  \texttt{newtex}を実行する。
  このスクリプトは、\texttt{newtex.conf}の中で指定されたタイトル名と同じ名前のディレクトリを新たに作成する。
  ただし、タイトル中の空白文字はアンダースコアに変換されてディレクトリ名となる。
  スクリプトは、そのディレクトリの下にテンプレート・ファイルを生成する。
\end{enumerate}

\hypertarget{lb}{%
\paragraph{lb}\label{lb}}

\begin{verbatim}
$ lb [LaTeXfile]
\end{verbatim}

引数省略の場合は\texttt{main.tex}が引数で与えられた場合と同じ動作をする。
\texttt{lb}は引数のLaTeXファイルをビルドするスクリプトであり、これだけで足りることが多い。

\begin{itemize}
\tightlist
\item
  引数がルートファイルの場合はそれを\texttt{latexmk}を使ってビルドする。サブファイルの場合は\texttt{ttex}でビルドする。
\item
  引数がルートファイルの場合は、synctexを使わない。
\item
  引数がサブファイルの場合は、synctexを使い、コンパイル後に\texttt{lb.conf}で指定されたプリビューワを起動する。
\item
  カレント・ディレクトリ(通常はルートファイルのあるディレクトリになる)に\texttt{lb.conf}があれば、それを読み込んで変数の初期化をする
\item
  \texttt{lb.conf}でエンジン指定を省略すると\texttt{lb}が自分でエンジンを予測する。しかし、\texttt{lb.conf}でエンジンを指定するほうが好ましい。
\end{itemize}

\texttt{lb.conf}で初期値の設定ができる。

\begin{verbatim}
rootfile=main.tex
builddir=_build
engine=
latex_option=-halt-on-error
preview=texworks
\end{verbatim}

\begin{itemize}
\tightlist
\item
  \texttt{rootfile}はルートファイルの名前。ただし、\texttt{lb}の引数でルートファイルを指定した場合は、引数を優先する。
\item
  \texttt{builddir}は作業ディレクトリを指定する。
  そのディレクトリには補助ファイルや出力ファイル、対象がサブファイルの場合は仮のルートファイルが出力される。
  空文字列を指定すると、作業ディレクトリは生成されず、ソースファイルの置かれているディレクトリが作業ディレクトリになる。
\item
  \texttt{engine}はLaTeXエンジンを指定する。latex、platex、pdflatex、xelatex、lualatexを指定することができる。その他のエンジンはサポートしていない。
\item
  \texttt{latex\_option}は\texttt{latexmk}を通じてエンジンに与えるオプション。\texttt{lb.conf}が存在しない場合でも、\texttt{-output-directory}は\texttt{lb}が自動的にエンジンに与える。
\item
  \texttt{preview}はできあがったpdfを見るためのプリビューワ。ただし、サブファイルのときのみ動作する。
\end{itemize}

\hypertarget{arl}{%
\paragraph{arl}\label{arl}}

\begin{verbatim}
$ arl [-b|-g|-z] [rootfile]
\end{verbatim}

\texttt{arl}という名前は、ARchive LaTeX filesから。
ルートファイルの関連ファイル(下記参照)を検索してアーカイブを作る。
ルートファイルが省略された場合は、\texttt{main.tex}を指定されたものとして処理する。

\begin{itemize}
\tightlist
\item
  前処理プログラムがある場合、そのプログラムを実行してから\texttt{arl}を起動する必要がある。
\item
  \texttt{arl}がアーカイブするのは、LaTeXソースファイルと、\texttt{includegraphics}される画像ファイルのみ。
\item
  したがって\texttt{Makefile}や、前処理のソースファイル(例えばgnuplotのソース)などはアーカイブされない。
\end{itemize}

Makefileにターゲットを作り(例えば\texttt{ar}という名前のターゲット)、\texttt{arl}で作ったアーカイブにtarでMakefileや前処理ソースファイルを追加するスクリプトを書いておくと便利である。
同様のことはRakefileでもdきる。

アーカイブを圧縮するオプション
\texttt{-g}、\texttt{-b}、\texttt{-z}で、それぞれ、tar.gz, tar.bz2,
zipをサポート。
オプションが与えられなかった場合は、圧縮なしのtarballを作る。

\hypertarget{ux30e6ux30fcux30c6ux30a3ux30eaux30c6ux30a3ux7fa4}{%
\subsection{ユーティリティ群}\label{ux30e6ux30fcux30c6ux30a3ux30eaux30c6ux30a3ux7fa4}}

この項はスクリプトをメンテナンスするのでなければ読む必要はない。

\begin{verbatim}
$ srf subfile
\end{verbatim}

\texttt{subfile}からルートファイルを探し、その結果(絶対パス)を出力する。
\texttt{srf}は「Search Root File」の意味。

\begin{verbatim}
$ tfiles [-p|-a|-i] [rootfile]
\end{verbatim}

rootfileのサブファイルの一覧を取得する。
引数のルートファイルが省略された場合は、\texttt{main.tex}が指定されたものとして処理する。

\begin{itemize}
\tightlist
\item
  オプション無し =\textgreater{}
  ルートファイルが取り込むサブファイル(\texttt{\textbackslash{}begin\{document\}}から\texttt{\textbackslash{}end\{document\}}までの\texttt{include}または\texttt{input}コマンドで指定されたファイル)のリストを標準出力に出力する
\item
  \texttt{-p}
  プリアンブルで取り込まれるサブファイルのリストを標準出力に出力する
\item
  \texttt{-a}
  オプション無しのリストにルートファイルを加えて標準出力に出力する
\item
  \texttt{-i}
  \texttt{include}コマンドで取り込まれるファイルのみを標準出力に出力する。
  ただし、\texttt{includeonly}で指定されなかったファイルは除かれる。
\end{itemize}

注意:出力されるファイルのリストは改行で区切られている。

\begin{verbatim}
$ tftype [-r|-s|-q] files ...
\end{verbatim}

LaTeXのソースファイルの種類を調べるスクリプト。

\begin{itemize}
\tightlist
\item
  \texttt{-r}
  (デフォルト)引数のファイルの中からルートファイルのみを抽出して出力する
\item
  \texttt{-s} 引数のファイルの中からサブファイルのみを抽出して出力する
\item
  \texttt{-q}
  (quiet)上記の出力を抑制する。引数は1つのファイルのみで、そのファイルタイプをexitステータスで返す。
  exitステータスが0はルートファイル、1はサブファイル、エラーが生じた場合は2となる。
\end{itemize}

\texttt{-q}オプションを使うことが最も多い。

\begin{verbatim}
$ gfiles files ...
\end{verbatim}

引数は、latexのソースファイル(の列)である。
与えられたファイルの中で\texttt{\textbackslash{}includegraphics}によって取り込まれる画像ファイルの一覧を返す。

\begin{verbatim}
$ ltxengine rootfile
\end{verbatim}

コンパイルを行うLaTeXエンジンを予想する(本来ユーザが明示すべきだが・・・)
例えば、

\begin{verbatim}
\usepackage[luatex]{graphicx}
\end{verbatim}

というコマンドがプリアンブルにあれば、エンジンはlualatexと予想がつく。

\begin{verbatim}
$ ttex [-b builddir] -e latex_engine [-p dvipdf] [-v previewr] -r rootfile subfile
\end{verbatim}

サブファイルに仮ルートファイルをつけてコンパイルする。
コンパイルは1回だけ。 そのため相互参照は反映されない。
(これはテストのためのスクリプトであって、最終仕上げではないから相互参照はさほど重要ではない、という考えに基いている)。
また、該当のサブファイル以外のファイルにあるラベルを参照することはできない。
単独で使うことも可能だがlbを通して呼び出すのが普通の使い方。
オプションについては下記の通り。

\begin{itemize}
\tightlist
\item
  \texttt{-b}
  作業ディレクトリを指定する。デフォルトは\texttt{\_build}である。
\item
  \texttt{-e}
  latexエンジンを指定する。エンジンの種類についての制限はないが、latex、platex、pdflatex、xelatex、lualatexのいずれかが指定されることを想定している。
\item
  \texttt{-p}
  エンジンがlatexまたはplatexである場合は、dviファイルが出力される。
  そのdviからpdfを出力するためのアプリケーションを指定する。
  デフォルトはdvipdfmxである。
  その他に、dvipdfmやdvipdfを指定することができる。
\item
  \texttt{-v} プリビューアを指定する。
  evinceなど、pdfを表示できるアプリケーションを指定する。
  ソースファイルをtexworksで編集している場合は、ここにtexworksを指定するのが良い。
\item
  \texttt{-r} ルートファイルを指定する。
\end{itemize}

\hypertarget{ux30a4ux30f3ux30b9ux30c8ux30fcux30ebux3068ux30a2ux30f3ux30a4ux30f3ux30b9ux30c8ux30fcux30eb}{%
\subsection{インストールとアンインストール}\label{ux30a4ux30f3ux30b9ux30c8ux30fcux30ebux3068ux30a2ux30f3ux30a4ux30f3ux30b9ux30c8ux30fcux30eb}}

\hypertarget{ux30a4ux30f3ux30b9ux30c8ux30fcux30ebux306bux5fc5ux8981ux306aux74b0ux5883}{%
\paragraph{インストールに必要な環境}\label{ux30a4ux30f3ux30b9ux30c8ux30fcux30ebux306bux5fc5ux8981ux306aux74b0ux5883}}

\begin{itemize}
\item
  Linuxとbash。
  DebianとUbuntuではテストされているが、おそらくその他のlinuxディストリビューションでも動作すると思われる。
  Bashコマンドを用いてスクリプトが記述されているので、bashは必要である。
\item
  LaTeXシステム。 LaTeXのインストールには2つのオプションがある。
  1つはディストリビューションに付属のシステムをインストールすることである。
  他方はTexLiveをインストールすることである。
\item
  makeまたはrake。
  これらのツールはBuildtoolsにとって、必ずしも必要というわけではない。
  しかし、makeまたはrakeのもとで、Buildtoolsを実行することが望ましい。
  この2つに両方をインストールする必要はない。
  どちらか1つを選んでインストールすれば良い。
  makeは長く使われているビルド・ツールで、元々はCコンパイラの制御に用いられてきた。
  rakeはこれに似たツールで、rubyで書かれたアプリケーションである。
  rakeを使うことの利点は、そのスクリプトであるRakefileの中で任意のrubyコードを記述できることである。
  一般に、RakefileはMakefileよりも読みやすく、理解しやすい。
\end{itemize}

\hypertarget{ux30a4ux30f3ux30b9ux30c8ux30fcux30eb}{%
\paragraph{インストール}\label{ux30a4ux30f3ux30b9ux30c8ux30fcux30eb}}

インストール用のスクリプト\texttt{install.sh}を使う。

\begin{verbatim}
$ bash install.sh
\end{verbatim}

シェルスクリプトなどの実行ファイルは\texttt{\$HOME/bin}に保存される。
debianやubuntuでは、ログイン時に\texttt{\$HOME/bin}があれば、bashの実行ディレクトリのパスを表す環境変数\texttt{PATH}に追加される。
インストール時に新規に \texttt{\$HOME/bin}
を作成した場合には、再ログインしないと、それが実行ディレクトリに追加されないので注意が必要。
rootになってインストールすると\texttt{/usr/local/bin}に実行ファイルをインストール。
debianの場合は、

\begin{verbatim}
$ su -
# bash install.sh
\end{verbatim}

ubuntuの場合は

\begin{verbatim}
$ sudo bash install.sh
\end{verbatim}

\hypertarget{ux30a2ux30f3ux30a4ux30f3ux30b9ux30c8ux30fcux30eb}{%
\paragraph{アンインストール}\label{ux30a2ux30f3ux30a4ux30f3ux30b9ux30c8ux30fcux30eb}}

アンインストールは\texttt{uninstall.sh}で行う。
一般ユーザで実行すれば、\texttt{\$HOME}以下のインストールファイルが削除される。

\begin{verbatim}
$ bash uninstall.sh
\end{verbatim}

rootで実行すれば、\texttt{/usr/local}以下のインストールファイルが削除される。
debianの場合は、

\begin{verbatim}
$ su -
# bash uninstall.sh
\end{verbatim}

ubuntuの場合は、

\begin{verbatim}
$ sudo bash uninstall.sh
\end{verbatim}

\hypertarget{ux30e9ux30a4ux30bbux30f3ux30b9}{%
\subsection{ライセンス}\label{ux30e9ux30a4ux30bbux30f3ux30b9}}

Copyright (C) 2020 ToshioCP (関谷 敏雄)

Buildtoolsはフリーソフトウェアであり、フリーソフトウェア財団によって発行されたGNU
一般公衆利用許諾書(バージョン3またはそれ以降のバージョン)の定める条件の下で再頒布または改変することができる。

Buildtoolsは多くの人にとって有用であると考えて頒布されているものであるが、これは\emph{全くの無保証}
である。商業可能性の保証や特定の目的への適合性は、言外に示されたものも含め、全く存在しない。
詳しくは\href{https://www.gnu.org/licenses/gpl-3.0.html}{GNU GENERAL
PUBLIC LICENSE}をご覧いただきたい。
また、その参考として、八田真行氏による\href{https://gpl.mhatta.org/gpl.ja.html}{GNU
一般公衆利用許諾書の非公式日本語訳}がある。

\end{document}
\end{verbatim}
更に、8行目の{\textbackslash}tableofcontentsをアンコメントしてコマンドが利くようにしよう。
\begin{verbatim}
 ... ...
\maketitle
% If you want a table of contents here, uncomment the following
 line.
\tableofcontents
 ... ...
\end{verbatim}

2番めは、{\textbackslash}tightlistのマクロを定義することが必要である。
Helper.texがその定義を書くのに最も適した場所である。
\begin{verbatim}
 ... ...
 ... ...
\providecommand{\tightlist}{%
  \setlength{\itemsep}{0pt}\setlength{\parskip}{0pt}}
 ... ...
 ... ...
\end{verbatim}
このコードは\url{https://github.com/jgm/pandoc-templates/blob/master/default.latex}から引用したものである。

rakeを使ってコンパイルする。
\begin{verbatim}
$ rake
\end{verbatim}

\begin{center}
\includegraphics[width=12cm]{tableofcontents.png}
\end{center}

これで、目次にセクション9、そして「Readme.ja.md」の内容が表示された。

このセクションではpandocを手動で走らせた。
もしも、Readme.ja.md がアップグレードされたならば、再びpandocを実行しなければならない。
それは面倒なことであり、本来自動化されるべきことである。
ひとつの方法はRakefileを変更してrakeがコンパイル前に自動的にプリプロセッシングするようにすることである。
次のセクションでその方法を説明する。


\section{Use rake}
  Rake is a build tool similar to make.
Rakefile describes instructions for rake to build source files.
You can write any ruby commands in Rakefile.
Therefore, it has a high ability to describe the build process even if it is complicated, .

Newtex generates a Rakefile, which is enough to compile the source files if there is no preprocessing procedure.
In the previous section, we used pandoc to generate readme.tex.
So, we need to modify Rakefile to put in pandoc.
Modify the Rakefile as follows.
\begin{verbatim}
require 'rake/clean'

# if readme.tex doesn't exist, generate it first.
# This is necessary because readme.tex is accessed by gfiles in
#  line 12.
if File.exist?("readme.tex") == false
  sh "pandoc -o readme.tex ../Readme.md"
end
# use Latex-BuildTools
@tex_files = (`tfiles -a` + `tfiles -p`).split("\n")
@tex_files <<= "readme.tex"
@graphic_files = []
@tex_files.each do |file|
  @graphic_files += `gfiles #{file}`.split("\n")
end

task default: "Tutorial.pdf"

file "Tutorial.pdf" => "_build/main.pdf" do
  sh "cp _build/main.pdf Tutorial.pdf"
end

file "_build/main.pdf" => (@tex_files+@graphic_files) do
  sh "lb main.tex"
end

file "readme.tex" => "../Readme.md" do
  sh "pandoc -o readme.tex ../Readme.md"
end

CLEAN << "_build"
task :clean

task :ar do
  sh "arl main.tex"
  sh "tar -rf main.tar Rakefile"
  sh "gzip main.tar"
  sh "mv main.tar.gz Tutorial.tar.gz"
end

task :zip do
  sh "arl -z main.tex"
  sh "zip main.zip Rakefile"
  sh "mv main.zip Tutorial.zip"
end
\end{verbatim}

Thanks to this modification, you don't need to run pandoc by hand.
What you need is just type `rake'.

There are websites about ruby and rake.
For example,
\begin{itemize}
\item \url{https://www.ruby-lang.org/en/}
\item \url{http://rubylearning.com/}
\item \url{https://ruby.github.io/rake/}
\end{itemize}

The tutorial finishes at this section.
Next section is the copy of Readme.md in Buildtools source files.
It describes the background of Buildtools and features of each script.

\section{Make tarball}
  ソースファイルを配布したい、という場合もあるかもしれない。
そのようなときには、それをアーカイブすることが必要になる。
Buildtoolsに含まれるスクリプトのarlは、ルートファイルが取り込むサブファイルや画像ファイルを検索し、それらアーカイブする。
このアーカイブされたファイルのうち、tarというコマンドで作られたものをtarballという。
\begin{itemize}
\item -gオプションが与えられると、gzipで圧縮されたtarballを作る。
\item -bオプションが与えられると、bzip2で圧縮されたtarballを作る。
\item -zオプションが与えられると、zipファイルを作る。
\item オプションが与えられなければ、非圧縮のtarballを作る。
\end{itemize}

もし、プリプロセッシングで生成されるlatexソースファイルなどがある場合は、arlの実行前にそれらを生成しておかなければならない。
\begin{verbatim}
$ arl
$ tar -tf main.tar
main.tex
edit_tex_files.tex
generate_templates.tex
installation.tex
lb.tex
preprocessing.tex
rake.tex
readme.tex
tarball.tex
test_compile.tex
helper.tex
Tutorial_1.png
Tutorial_2.png
hellolatex.png
tableofcontents.png
test_installation.png
\end{verbatim}

Rakefileも当然ながら、tarballに含めなければならない。
\begin{verbatim}
$ tar -rf main.tar Rakefile
\end{verbatim}
そして、gzipなどに圧縮してtarballが完成する。
\begin{verbatim}
$ gzip main.tar
\end{verbatim}

以上の手続きは、実はすでにRakefileに記述されている。
「rake ar」とタイプすることにより、rakeがtarballを自動生成してくれる。
\begin{verbatim}
$ rm main.tar.gz
$ rake ar
arl main.tex
tar -rf main.tar Rakefile
gzip main.tar
mv main.tar.gz チュートリアル.tar.gz
\end{verbatim}
最後にできあがるtarballの名前は「チュートリアル.tar.gz」である。

もしも、zipファイルを作りたければ、「rake zip」とタイプする。

このセクションで、チュートリアルは終わりである。
次のセクションはBuildtoolsのソースファイルに含まれるReadme.ja.mdのコピーである。
この文書には、Buildtoolsを作成した背景や個々のスクリプトの特長と使い方が書かれている。


\end{document}
\end{verbatim}
The first line specifies a documentclass which is the same as the value of documentclass key in newtex.conf.
The second line has an input command which includes `helper.tex'.
Helper.tex has a role to include packages with \verb|\usepackage| command, define macros with \verb|\newcommand| command and so on.
Most of the lines in the preamble are described in helper.tex.
It is a good idea to make your own helper.tex because users often use the same preamble in different documents.
If you have your helper.tex, copy and overwrite this file.

You need to edit the fourth line.
For example,
\begin{verbatim}
\author{Toshio Sekiya}
\end{verbatim}
You can add `{\textbackslash}date', `{\textbackslash}thanks', `{\textbackslash}begin\{abstract\}' and `{\textbackslash}end\{abstract\}' if you like.
If you want to make a table of contents, then uncomment the eighth line.
After that, the lines are sections and {\textbackslash}input commands to include subfiles.

Rakefile contains instructions to rake.
You don't need to modify it so far.
Try to use it.
\begin{verbatim}
$ rake
 ... ...
 ... ...
$ ls
Makefile            gecko.png               main.tex
Rakefile            generate_templates.tex  preprocessing.tex
Tutorial.pdf        helper.tex              rake.tex
_build              installation.tex        tarball.tex
cover.tex           lb.conf                 test_compile.tex
edit_tex_files.tex  lb.tex
$ ls _build
main.aux  main.fdb_latexmk  main.fls  main.log  main.out  main.pdf
$ evince Tutorial.pdf
\end{verbatim}
Rake ran lb to compile main.tex and after that it copied \_build/main.pdf to Tutorial.pdf.
Evince shows Tutorial.pdf as follows.
\begin{center}
\includegraphics[width=8cm]{Tutorial_1.png}
\end{center}


\section{Edit tex files}
  There are eight sections and subfiles.
Each subfile is empty just after it is generated by newtex.
Editing subfiles is the main work and you need to allocate most of your time to this work.

The first section and subfile are `Installation' and installation.tex respectively.
Maybe you edit a part of the section and test-compile to see how it looks like in the pdf file.
Usually we come and go between editing and test-compiling repeatedly.

If the document is not so big, using rake is the best to test-compile, because it doesn't take much time and the pdf shows the whole document.
This tutorial is rather a small document, so using rake for test-compiling is fine.

\begin{verbatim}
\subsection{Prerequisite}
Buildtools requires the following items.
\begin{enumerate}
\item Linux OS and bash
\item LaTeX system
\item Make or Rake
\end{enumerate}
 ... ...
 ... ...
\end{verbatim}

Then, type the following line to see the pdf.
\begin{verbatim}
$ rake
$ evince Tutorial.pdf
\end{verbatim}

\begin{center}
\includegraphics[width=8cm]{Tutorial_2.png}
\end{center}

\section{Test compile}
  もし大きな文書、例えば100ページを越える書籍を作るような場合、テスト・コンパイルにrakeを使う(rakeで文書全体をコンパイルするという意味)のは良い方法ではない。
ドキュメントが大きくなればなるほど、コンパイル時間が長くなるからである。

それよりは、lbを使ってサブファイルを単独でコンパイルするのが良い方法である。
lbは一時的なルートファイル(テンポラリ・ルートファイルという)を作り、その中にサブファイルを取り込む命令を記述し、1度だけコンパイルする。
この方法の良い点は短時間でコンパイルできることである。
しかし、良くない点もある。
サブファイルのみのコンパイルであるから、文書全体のpdfを見ることはできない。
また、1度だけのコンパイルなので、相互参照は反映されない。
どちらが良いかは一概には言えない。
それは作成している文書によるが、もしそれが非常に大きな文書であれば、テスト・コンパイルをlbでやるほうがより良いといえる。

次のようにタイプしてほしい。
\begin{verbatim}
$ lb installation
\end{verbatim}
この引数はサブファイル名である。
拡張子は省略することもできる。

すると、lbはテンポラリ・ルートファイル「\_build/test\_installation.tex」を作る。
そのプリアンブルは元のルートファイルのプリアンブルのコピーである。
そして、{\textbackslash}inputコマンドでサブファイルの「installation.tex」を取り込むようになっている。
lbは「\_build/test\_installation.tex」をコンパイルし、lb.confで指定されたpdfビューワを起動してpdfファイルを表示する。
\begin{center}
\includegraphics[width=12cm]{test_installation.png}
\end{center}

lbはコンパイルするときにsynctexのオプションをオンにする。
したがって、ソースファイルの「\_build/test\_installation.tex」と「installation.tex」をエディタで開いておけば、ソースとpdfの間で前方参照と後方参照をすることができる。
もしもgeditをエディタに、evinceをビューワに使っていれば、コントロール・キーを押したまま左クリックすれば後方参照(pdfからソースへの参照)が可能である。
しかし、「installation.tex」からpdfへの前方参照は働かない。
それを機能させるためには、サブファイルの先頭に次の1行を加えなければならない。
\begin{verbatim}
% mainfile: _build/test_installation.tex
\end{verbatim}
しかし、前方参照を使うことは後方参照に比べそう多くはない。
上記の1行を加えるのは、たいていは必要ないだろう。


\section{Preprocessing}
  ときにはlatexソースファイルをコンパイルする前に何かしておきたい、ということがあるかもしれない。
例えば、
\begin{itemize}
\item gnuplotなどのプログラムを使って画像ファイルを生成したい
\item pandocを使ってlatexソースファイルを自動生成したい
\end{itemize}

このチュートリアルではプリプロセッサとしてpandocを使う方法を紹介したい。
pandocは文書コンバータである。
markdown、latex、html、pdfなど多くの種類の文書を変換することができる。
このチュートリアルでは、Buildtoolsのソースファイルの中にある「Readme.ja.md」を「readme.tex」というlatexソースファイルに変換してみる。

たいていのディストリビューションではpandocパッケージが備わっているので、インストールは簡単である。
例えばubuntuでは
\begin{verbatim}
$ sudo apt-get install pandoc
\end{verbatim}
でインストールできる。

readme.texを生成するには次のようにタイプする。
\begin{verbatim}
$ pandoc -o readme.tex ../Readme.ja.md
\end{verbatim}

この他に2つほどmain.texとhelper.texに変更を加える必要がある。
まず、{\textbackslash}inputコマンドを用いてreadme.texを取り込む命令をmain.texの最後に記述する。
\begin{verbatim}
\documentclass{article}
% helper.tex

% 注意
% graphicxのドライバーがluatexになっている。他のドライバーを使う場合は書き換えが必要。

% パッケージのとりこみ
\usepackage{amsmath,amssymb}
\usepackage[luatex]{graphicx}
\usepackage{tikz}
\usetikzlibrary{calc}
\usetikzlibrary{topaths}
\usetikzlibrary{plotmarks}
\usetikzlibrary{intersections}
\usetikzlibrary{arrows,decorations.pathmorphing,backgrounds,positioning,fit,petri}
\usetikzlibrary{arrows.meta}
\usetikzlibrary{3d}
%\usepackage{gnuplot-lua-tikz}

\usepackage{fancybox}
\usepackage{booktabs}

\usepackage[margin=2.4cm]{geometry}

\usepackage[colorlinks=true,linkcolor=black,pdfencoding=auto]{hyperref}

% マクロのサンプル(解答、証明、q.e.d.、解答)
%\newcommand{\solution}{\begin{flushleft}\textbf{解:}\end{flushleft}}
%\newcommand{\proof}{\begin{flushleft}\textbf{証明}\end{flushleft}}
%\newcommand{\qed}{\begin{flushright}\textbf{証明終}\end{flushright}}

% 定理環境(定理、補題、系、定義、例、練習問題)
\newtheorem{theorem}{定理}[section]
\newtheorem{lemma}{補題}[section]
\newtheorem{corollary}{系}[section]
\newtheorem{definition}{定義}[section]
\newtheorem{example}{例}[section]
\newtheorem{exercise}{問題}[section]

% 凹凸増減表の矢印 ----------------------------------------------------------------------------
% concave north east arrow 上に凸で増加
\newcommand{\ccnearrow}{
\begin{tikzpicture}
  \draw[very thin,->] (0,0) .. controls (0,0.2) and (0.05,0.25) .. (0.25,0.25);
\end{tikzpicture}
}
% concave south east arrow 上に凸で減少
\newcommand{\ccsearrow}{
\begin{tikzpicture}
  \draw[very thin,->] (0,0) .. controls (0.2,0) and (0.25,-0.05) .. (0.25,-0.25);
\end{tikzpicture}
}
% convex north east arrow 下に凸で増加
\newcommand{\cvnearrow}{
\begin{tikzpicture}
  \draw[very thin,->] (0,0) .. controls (0.2,0) and (0.25,0.05) .. (0.25,0.25);
\end{tikzpicture}
}
% convex south east arrow 下に凸で減少
\newcommand{\cvsearrow}{
\begin{tikzpicture}
  \draw[very thin,->] (0,0) .. controls (0,-0.2) and (0.05,-0.25) .. (0.25,-0.25);
\end{tikzpicture}
}
%-----------------------------------------------------------------------------------------------

% for long division (割り算の筆算のためのマクロ、Tikzの中で使う)=======================================================
% \divisoroffset is the distance between x-axis 0 and the center of the divisor. It is negative number.
\newcommand{\divisor}[2]{\node[anchor=base] at (#1,0) {#2}}
% dividendoffset is the offset from x-axis 0 to the first term bound of the dividend.
\newcommand{\dividendoffset}{0.8}
% \termwidth is the width between the consecutive term.
\newcommand{\termwidth}{0.9}
% expressionheight is the height of each expression.
\newcommand{\expressionheight}{0.5}
\newcommand{\pt}[3]{\node[base left] at (\dividendoffset+\termwidth*#1,\expressionheight*#2) {#3}}
\newcommand{\hseparator}[3]{\draw ({\dividendoffset+(#1-1)*\termwidth},{-(#3+0.3)*\expressionheight})
                            -- ({\dividendoffset+(#2)*\termwidth},{-(#3+0.3)*\expressionheight})}
\newcommand{\divisionbox}[1]{\draw (0,\expressionheight*0.7) -- (\dividendoffset+\termwidth*#1,\expressionheight*0.7)
                                   (0,\expressionheight*0.7) arc[start angle=27,end angle=-27,radius=0.5]}
%=========================================================================================================================

 ... ...
 ... ...
\section{Make tarball}
  ソースファイルを配布したい、という場合もあるかもしれない。
そのようなときには、それをアーカイブすることが必要になる。
Buildtoolsに含まれるスクリプトのarlは、ルートファイルが取り込むサブファイルや画像ファイルを検索し、それらアーカイブする。
このアーカイブされたファイルのうち、tarというコマンドで作られたものをtarballという。
\begin{itemize}
\item -gオプションが与えられると、gzipで圧縮されたtarballを作る。
\item -bオプションが与えられると、bzip2で圧縮されたtarballを作る。
\item -zオプションが与えられると、zipファイルを作る。
\item オプションが与えられなければ、非圧縮のtarballを作る。
\end{itemize}

もし、プリプロセッシングで生成されるlatexソースファイルなどがある場合は、arlの実行前にそれらを生成しておかなければならない。
\begin{verbatim}
$ arl
$ tar -tf main.tar
main.tex
edit_tex_files.tex
generate_templates.tex
installation.tex
lb.tex
preprocessing.tex
rake.tex
readme.tex
tarball.tex
test_compile.tex
helper.tex
Tutorial_1.png
Tutorial_2.png
hellolatex.png
tableofcontents.png
test_installation.png
\end{verbatim}

Rakefileも当然ながら、tarballに含めなければならない。
\begin{verbatim}
$ tar -rf main.tar Rakefile
\end{verbatim}
そして、gzipなどに圧縮してtarballが完成する。
\begin{verbatim}
$ gzip main.tar
\end{verbatim}

以上の手続きは、実はすでにRakefileに記述されている。
「rake ar」とタイプすることにより、rakeがtarballを自動生成してくれる。
\begin{verbatim}
$ rm main.tar.gz
$ rake ar
arl main.tex
tar -rf main.tar Rakefile
gzip main.tar
mv main.tar.gz チュートリアル.tar.gz
\end{verbatim}
最後にできあがるtarballの名前は「チュートリアル.tar.gz」である。

もしも、zipファイルを作りたければ、「rake zip」とタイプする。

このセクションで、チュートリアルは終わりである。
次のセクションはBuildtoolsのソースファイルに含まれるReadme.ja.mdのコピーである。
この文書には、Buildtoolsを作成した背景や個々のスクリプトの特長と使い方が書かれている。


\hypertarget{latex-buildtools}{%
\section{Latex-Buildtools}\label{latex-buildtools}}

\hypertarget{latex-buildtoolsux4f5cux6210ux306eux80ccux666f}{%
\subsection{Latex-Buildtools作成の背景}\label{latex-buildtoolsux4f5cux6210ux306eux80ccux666f}}

\hypertarget{buildtoolsux306flatexux3067ux5927ux304dux306aux6587ux66f8ux3092ux4f5cux308dux3068ux304dux306bux7528ux3044ux308bux30c4ux30fcux30ebux306eux3072ux3068ux3064}{%
\paragraph{BuildtoolsはLaTeXで大きな文書を作ろときに用いるツールのひとつ}\label{buildtoolsux306flatexux3067ux5927ux304dux306aux6587ux66f8ux3092ux4f5cux308dux3068ux304dux306bux7528ux3044ux308bux30c4ux30fcux30ebux306eux3072ux3068ux3064}}

大きな文書、例えば100ページを越える本などをLaTeXで作る場合は、小さい文書の作成と異なる様々な問題を考える必要がある。

\begin{itemize}
\tightlist
\item
  ソースファイルの分割
\item
  分割コンパイル
\item
  文書の一括置換
\item
  前処理(LaTeXコンパイル前に処理する作業)
\end{itemize}

これらを支援する2つのツールがある。

\begin{itemize}
\tightlist
\item
  Latex-Buildtools、または単にBuildtools。ソースファイルの新規作成、ビルド、分割コンパイルを支援するツール群
\item
  Latex-Substools、または単にSubstools。ソースファイルに対する一括置換をするツール群
\end{itemize}

Buildtoolsは上記の作業の中でその中核をなすツール群である。
このドキュメントではBuildtoolsを解説する。

\hypertarget{ux30bdux30fcux30b9ux30d5ux30a1ux30a4ux30ebux306eux5206ux5272}{%
\paragraph{ソースファイルの分割}\label{ux30bdux30fcux30b9ux30d5ux30a1ux30a4ux30ebux306eux5206ux5272}}

LaTeXソースファイルを単にここではソースファイルと呼ぶ。
大きな文書を1つのソースファイルに記述するのは適切でない。
なぜなら、ファイルが大きくなると、エディタで編集するのが極めて困難になるからである。
そこで、文書を分割することになる。
通常は\texttt{\textbackslash{}begin\{document\}}と\texttt{\textbackslash{}end\{document\}}を含む1つのファイルと、そのファイルから\texttt{\textbackslash{}include}または\texttt{\textbackslash{}input}で呼び出される複数のファイルに分割する。
前者をルートファイル、後者をサブファイルという。
\texttt{\textbackslash{}include}と\texttt{\textbackslash{}input}はどちらもサブファイルの取り込みのコマンドだが、違いがある。

\begin{itemize}
\tightlist
\item
  \texttt{\textbackslash{}include}はネストはできない。\texttt{\textbackslash{}include}はボディ(\texttt{\textbackslash{}begin\{document\}}と\texttt{\textbackslash{}end\{document\}}の間の部分)にのみ書くことができる。\texttt{\textbackslash{}includeonly}でファイルのリストが指定されている場合は、そのリストに書かれているファイルのみが\texttt{\textbackslash{}include}で取り込め、リストにないファイルの\texttt{\textbackslash{}include}は無視される。\texttt{\textbackslash{}includeonly}はプリアンブルのみに書くことができる。\texttt{\textbackslash{}include}はファイルを取り込む前に\texttt{\textbackslash{}clearpage}をする。
\item
  \texttt{\textbackslash{}input}は単にファイルを取り込むだけで\texttt{\textbackslash{}clearpage}はしない。このコマンドはネストできる。
\end{itemize}

文書をコンパイルによって作成することをビルドと呼ぶ。
ビルドはLaTeXのコンパイルだけでなく、前処理(例えばgnuplotによる画像生成だったり、データからtikzのグラフを生成したりなど)も含める。
ビルドは最終的にルートファイルをコンパイルすることによって完了する。
LaTeXソースファイルをコンパイルするプログラムにはいくつかあり、それをエンジンと呼んでいる。
Buildtoolsでサポートしているエンジンには、latex、platex、pdflatex、xelatex、lualatexがある。
コンパイルは文書が大きくなればなるほど時間がかかるようになる。
1箇所だけを変更する場合も同じだけ時間がかかる。
文書の編集の途中で出来栄えをチェックするためにコンパイルすること(これをテストするなどということがある)は頻繁に起こるが、そのたびに長い時間がかかる。
文書が大きくなればなるほどこの問題は深刻になる。
そこで、サブファイルのみをコンパイルできるようにする方法がいろいろ考えられた。

\begin{itemize}
\tightlist
\item
  \texttt{\textbackslash{}includeonly}コマンドの引数(サブファイルのリスト)からコンパイルしたくないファイルをコメントアウトする
\item
  subfilesパッケージを用いる
\end{itemize}

とくにsubfilesパッケージはよくできており、推薦する方も多い。
ただ、パッケージを取り込み、そのコマンドを適切に使うことが必要である。
もちろん、上記のコンパイル時間の問題からすれば、それくらいは何ともないことであるが。

これ以外の方法としては、サブファイル単独のコンパイル用に特別のプリアンブル部などを付け加える方法がある。
具体的には、サブファイルを「\texttt{\textbackslash{}documentclass}から\texttt{\textbackslash{}begin\{document\}}まで」と「\texttt{\textbackslash{}end\{document\}}」でサンドイッチにする。
そのときにサブファイル自体を変更するのではなく、プリアンブルなどを記述したファイルからサブファイルを\texttt{\textbackslash{}input}で取り込むようにする。
そのサブファイルを取り込むファイルを「仮のルートファイル」と呼んだりする。
それに対して元々のルートファイルを「オリジナルのルートファイル」と呼ぶこともある。
仮のルートファイルのプリアンブル部は、オリジナルのルートファイルのコピーである。
この方法の良い点は

\begin{itemize}
\tightlist
\item
  ソースファイルにパッケージの取り込みや特別な命令を書き込む必要がない。
\item
  したがって、ソースファイルを配布するにあたって、被配布者に特定のパッケージをインストールさせる必要がない。
\end{itemize}

つまり、ソースファイルをなんら変更することなく、サブファイルのみのコンパイルが可能だというのが長所である。
ただ、そのためには仮のルートファイルを生成するプログラムが必要である。
Buildtoolsでは、\texttt{ttex}というシェルスクリプトでそれを行っている。

\hypertarget{ux30b3ux30f3ux30d1ux30a4ux30ebux306eux30eaux30d4ux30fcux30c8ux7e70ux308aux8fd4ux3057}{%
\paragraph{コンパイルのリピート(繰り返し)}\label{ux30b3ux30f3ux30d1ux30a4ux30ebux306eux30eaux30d4ux30fcux30c8ux7e70ux308aux8fd4ux3057}}

最終的に文書をコンパイルするときには、相互参照などを実現するためにコンパイルを複数回行わなければならない。
その回数は2回であったり、3回であったりするらしいが、筆者はその事情をよく知らない。
が、ここに優れたプログラムがあり、必要回数を判断し、必要なだけコンパイルしてくれる。
latexmkというプログラムである。
latexmkを用いることによって、ビルドは非常に楽になる。
Buildtoolsではルートファイルのコンパイルにlatexmkを使用する。

\hypertarget{ux4f5cux696dux7528ux30c7ux30a3ux30ecux30afux30c8ux30eaux306eux8a2dux7f6e}{%
\paragraph{作業用ディレクトリの設置}\label{ux4f5cux696dux7528ux30c7ux30a3ux30ecux30afux30c8ux30eaux306eux8a2dux7f6e}}

LaTeXでコンパイルすると、様々な補助ファイルやログファイルが生成されるので、きれい好きの人はそれについて不満を感じることがあると思う。
それで、作業用ディレクトリを設置して、そういうファイルの一切合財を入れてしまうとソースディレクトリをきれいに保つことができる。
そういうソフトにcluttexがあり、きれい好きな人にはお勧めである。
Buildtoolsでは作業用ディレクトリ\texttt{\_build}(名前は変更も可能)を設置し、補助ファイルなどを格納する。
このことにより、ソースディレクトリを汚さずに済む。
また、コンパイルのログや補助ファイルを見たいときは\texttt{\_build}の中を見れば良い。
非常に単純なのである。
蛇足になるが、最近Cのビルドツールとして人気のあるmesonも作業ディレクトリを使う。
これは、一般に作業用ディレクトリをソースディレクトリと区別することが人間にとって非常に分かりやすくなるということの表れである。

Buildtoolsでは、生成された最終文書(pdfファイル)も作業用ディレクトリにできあがる。
それをソースファイルディレクトリに置きたいというのは自然は発想だが、それにはmakeまたはrakeを使うと良い。
rakeはスクリプト言語rubyで書かれたmakeとでもいうべきものであるが、特長はRakefile(rakeのスクリプト)にruby言語を使うことができることである。
そのことによって、makeよりもはるかに強力で分かりやすい記述ができる。
話を元に戻すが、最終文書をソースディレクトリに置くには、MakefileまたはRakefileに、作業用ディレクトリからソースディレクトリに最終文書をコピーするコマンドを書いておくのである。
また、makeやrakeを使うことの利点は、前処理を記述できることである。
前処理はその文書やユーザの使うツールに依存するので、Buildtoolsでカバーするのは難しい。
それに比べて、MakefileやRakefileはフレキシブルなので、前処理を記述するのに適しているのである。

Buildtoolsでは、makeまたはrakeを併用することを推奨している。

\hypertarget{texworksux3068ux306eux9023ux643a}{%
\paragraph{Texworksとの連携}\label{texworksux3068ux306eux9023ux643a}}

Buildtoolsでは、\texttt{lb}というコマンドでルートファイルのビルドもサブファイルのテストコンパイルも行うことができる。
それをTexworksのタイプセッティングに登録すると、Texworksから起動できて大変便利である。
「設定」ー>「タイプセッティング」タブで設定する。
+をクリックして新たなコマンドを設定する。名前=\textgreater{}\texttt{lb}、コマンド=\textgreater{}\texttt{lb}、引数=\textgreater{}\texttt{\$fullname}で良い。
設定後はルートファイルで全体のコンパイル、サブファイルは単独のテスト・コンパイルがワンクリックでできるようになる。

\hypertarget{buildtoolsux306eux69cbux6210}{%
\subsection{Buildtoolsの構成}\label{buildtoolsux306eux69cbux6210}}

Buildtoolsは大きく分けて次のような部分から構成されている。

\begin{itemize}
\tightlist
\item
  \texttt{newtex}: 新規ソースファイルの作成を支援するツール。
\item
  \texttt{lb}: Latex
  Build。ルートファイルをコンパイル、または\texttt{ttex}を使ってサブファイルのテストコンパイルをする。
\item
  アーカイブ作成を支援するツール。
\item
  インストーラ
\item
  ユーティリティ群。上記のプログラムを下支えする。
\end{itemize}

文書作成は、次の手順で行われる。

\begin{enumerate}
\def\labelenumi{\arabic{enumi}.}
\tightlist
\item
  文書全体の構成を決める。章立てを決める。
\item
  \texttt{newtex}を使ってフォルダとソースファイルの雛形を作成する。
\item
  \texttt{Makefile}、\texttt{Rakefile}、表紙(\texttt{cover.tex})、プリアンブル部(\texttt{helper.tex})の雛形を必要に応じて書き換える。
\item
  本文の作成と、テストコンパイル。
\item
  前処理の作成。
\item
  最終ビルド。
\end{enumerate}

上記の3から5は行ったり来たりすることになり、必ずしもこの順に作業が進むわけではない。

\hypertarget{ux4e3bux8981ux306aux30c4ux30fcux30ebux306bux3064ux3044ux3066}{%
\subsection{主要なツールについて}\label{ux4e3bux8981ux306aux30c4ux30fcux30ebux306bux3064ux3044ux3066}}

それぞれのツールの簡単なヘルプは\texttt{-\/-help}オプションをつけて実行することにより表示される。
例えば、\texttt{newtex}に\texttt{-\/-help}オプションをつけて実行すると、次のようなメッセージが表示される。

\begin{verbatim}
$ newtex --help
Usage:
  newtex --help
    Show this message.
  Newtex.conf needs to be edited before running newtex.
  newtex
    A directory is made and some template files are generated under the directory.
\end{verbatim}

各ツールを説明する文書は

\begin{itemize}
\tightlist
\item
  各ツールの\texttt{-\/-help}オプションによるヘルプ・メッセージ
\item
  このドキュメントにおける以下の記述
\end{itemize}

だけである。 より詳細を知りたい場合はソースコードを見ていただきたい。
Buildtoolsのすべてのツールはシェル・スクリプトで書かれている。
それぞれのスクリプトは短く、シェル・スクリプトをご存知の方であれば、比較的簡単にソースコードを理解できる。

\hypertarget{newtex}{%
\paragraph{newtex}\label{newtex}}

\begin{verbatim}
$ newtex
\end{verbatim}

新規にLaTeXの文書を作るときに使うスクリプト。
\texttt{newtex}を使う前に全体の構成と章立てを決めておき、それを事前に\texttt{newtex.conf}に書いておく。
このスクリプトは、\texttt{newtex.conf}に書かれた指示に従って、新しくディレクトリを作り、テンプレート・ファイルを生成する。

このプログラムは2回に分けて使う。

\begin{enumerate}
\def\labelenumi{\arabic{enumi}.}
\tightlist
\item
  Buildtoolsのソースファイルの中に\texttt{newtex.conf}ファイルがある。
  これを書き直して、ユーザの環境やこれから作るLaTeXファイルに合うようにする。
\item
  \texttt{newtex}を実行する。
  このスクリプトは、\texttt{newtex.conf}の中で指定されたタイトル名と同じ名前のディレクトリを新たに作成する。
  ただし、タイトル中の空白文字はアンダースコアに変換されてディレクトリ名となる。
  スクリプトは、そのディレクトリの下にテンプレート・ファイルを生成する。
\end{enumerate}

\hypertarget{lb}{%
\paragraph{lb}\label{lb}}

\begin{verbatim}
$ lb [LaTeXfile]
\end{verbatim}

引数省略の場合は\texttt{main.tex}が引数で与えられた場合と同じ動作をする。
\texttt{lb}は引数のLaTeXファイルをビルドするスクリプトであり、これだけで足りることが多い。

\begin{itemize}
\tightlist
\item
  引数がルートファイルの場合はそれを\texttt{latexmk}を使ってビルドする。サブファイルの場合は\texttt{ttex}でビルドする。
\item
  引数がルートファイルの場合は、synctexを使わない。
\item
  引数がサブファイルの場合は、synctexを使い、コンパイル後に\texttt{lb.conf}で指定されたプリビューワを起動する。
\item
  カレント・ディレクトリ(通常はルートファイルのあるディレクトリになる)に\texttt{lb.conf}があれば、それを読み込んで変数の初期化をする
\item
  \texttt{lb.conf}でエンジン指定を省略すると\texttt{lb}が自分でエンジンを予測する。しかし、\texttt{lb.conf}でエンジンを指定するほうが好ましい。
\end{itemize}

\texttt{lb.conf}で初期値の設定ができる。

\begin{verbatim}
rootfile=main.tex
builddir=_build
engine=
latex_option=-halt-on-error
preview=texworks
\end{verbatim}

\begin{itemize}
\tightlist
\item
  \texttt{rootfile}はルートファイルの名前。ただし、\texttt{lb}の引数でルートファイルを指定した場合は、引数を優先する。
\item
  \texttt{builddir}は作業ディレクトリを指定する。
  そのディレクトリには補助ファイルや出力ファイル、対象がサブファイルの場合は仮のルートファイルが出力される。
  空文字列を指定すると、作業ディレクトリは生成されず、ソースファイルの置かれているディレクトリが作業ディレクトリになる。
\item
  \texttt{engine}はLaTeXエンジンを指定する。latex、platex、pdflatex、xelatex、lualatexを指定することができる。その他のエンジンはサポートしていない。
\item
  \texttt{latex\_option}は\texttt{latexmk}を通じてエンジンに与えるオプション。\texttt{lb.conf}が存在しない場合でも、\texttt{-output-directory}は\texttt{lb}が自動的にエンジンに与える。
\item
  \texttt{preview}はできあがったpdfを見るためのプリビューワ。ただし、サブファイルのときのみ動作する。
\end{itemize}

\hypertarget{arl}{%
\paragraph{arl}\label{arl}}

\begin{verbatim}
$ arl [-b|-g|-z] [rootfile]
\end{verbatim}

\texttt{arl}という名前は、ARchive LaTeX filesから。
ルートファイルの関連ファイル(下記参照)を検索してアーカイブを作る。
ルートファイルが省略された場合は、\texttt{main.tex}を指定されたものとして処理する。

\begin{itemize}
\tightlist
\item
  前処理プログラムがある場合、そのプログラムを実行してから\texttt{arl}を起動する必要がある。
\item
  \texttt{arl}がアーカイブするのは、LaTeXソースファイルと、\texttt{includegraphics}される画像ファイルのみ。
\item
  したがって\texttt{Makefile}や、前処理のソースファイル(例えばgnuplotのソース)などはアーカイブされない。
\end{itemize}

Makefileにターゲットを作り(例えば\texttt{ar}という名前のターゲット)、\texttt{arl}で作ったアーカイブにtarでMakefileや前処理ソースファイルを追加するスクリプトを書いておくと便利である。
同様のことはRakefileでもdきる。

アーカイブを圧縮するオプション
\texttt{-g}、\texttt{-b}、\texttt{-z}で、それぞれ、tar.gz, tar.bz2,
zipをサポート。
オプションが与えられなかった場合は、圧縮なしのtarballを作る。

\hypertarget{ux30e6ux30fcux30c6ux30a3ux30eaux30c6ux30a3ux7fa4}{%
\subsection{ユーティリティ群}\label{ux30e6ux30fcux30c6ux30a3ux30eaux30c6ux30a3ux7fa4}}

この項はスクリプトをメンテナンスするのでなければ読む必要はない。

\begin{verbatim}
$ srf subfile
\end{verbatim}

\texttt{subfile}からルートファイルを探し、その結果(絶対パス)を出力する。
\texttt{srf}は「Search Root File」の意味。

\begin{verbatim}
$ tfiles [-p|-a|-i] [rootfile]
\end{verbatim}

rootfileのサブファイルの一覧を取得する。
引数のルートファイルが省略された場合は、\texttt{main.tex}が指定されたものとして処理する。

\begin{itemize}
\tightlist
\item
  オプション無し =\textgreater{}
  ルートファイルが取り込むサブファイル(\texttt{\textbackslash{}begin\{document\}}から\texttt{\textbackslash{}end\{document\}}までの\texttt{include}または\texttt{input}コマンドで指定されたファイル)のリストを標準出力に出力する
\item
  \texttt{-p}
  プリアンブルで取り込まれるサブファイルのリストを標準出力に出力する
\item
  \texttt{-a}
  オプション無しのリストにルートファイルを加えて標準出力に出力する
\item
  \texttt{-i}
  \texttt{include}コマンドで取り込まれるファイルのみを標準出力に出力する。
  ただし、\texttt{includeonly}で指定されなかったファイルは除かれる。
\end{itemize}

注意:出力されるファイルのリストは改行で区切られている。

\begin{verbatim}
$ tftype [-r|-s|-q] files ...
\end{verbatim}

LaTeXのソースファイルの種類を調べるスクリプト。

\begin{itemize}
\tightlist
\item
  \texttt{-r}
  (デフォルト)引数のファイルの中からルートファイルのみを抽出して出力する
\item
  \texttt{-s} 引数のファイルの中からサブファイルのみを抽出して出力する
\item
  \texttt{-q}
  (quiet)上記の出力を抑制する。引数は1つのファイルのみで、そのファイルタイプをexitステータスで返す。
  exitステータスが0はルートファイル、1はサブファイル、エラーが生じた場合は2となる。
\end{itemize}

\texttt{-q}オプションを使うことが最も多い。

\begin{verbatim}
$ gfiles files ...
\end{verbatim}

引数は、latexのソースファイル(の列)である。
与えられたファイルの中で\texttt{\textbackslash{}includegraphics}によって取り込まれる画像ファイルの一覧を返す。

\begin{verbatim}
$ ltxengine rootfile
\end{verbatim}

コンパイルを行うLaTeXエンジンを予想する(本来ユーザが明示すべきだが・・・)
例えば、

\begin{verbatim}
\usepackage[luatex]{graphicx}
\end{verbatim}

というコマンドがプリアンブルにあれば、エンジンはlualatexと予想がつく。

\begin{verbatim}
$ ttex [-b builddir] -e latex_engine [-p dvipdf] [-v previewr] -r rootfile subfile
\end{verbatim}

サブファイルに仮ルートファイルをつけてコンパイルする。
コンパイルは1回だけ。 そのため相互参照は反映されない。
(これはテストのためのスクリプトであって、最終仕上げではないから相互参照はさほど重要ではない、という考えに基いている)。
また、該当のサブファイル以外のファイルにあるラベルを参照することはできない。
単独で使うことも可能だがlbを通して呼び出すのが普通の使い方。
オプションについては下記の通り。

\begin{itemize}
\tightlist
\item
  \texttt{-b}
  作業ディレクトリを指定する。デフォルトは\texttt{\_build}である。
\item
  \texttt{-e}
  latexエンジンを指定する。エンジンの種類についての制限はないが、latex、platex、pdflatex、xelatex、lualatexのいずれかが指定されることを想定している。
\item
  \texttt{-p}
  エンジンがlatexまたはplatexである場合は、dviファイルが出力される。
  そのdviからpdfを出力するためのアプリケーションを指定する。
  デフォルトはdvipdfmxである。
  その他に、dvipdfmやdvipdfを指定することができる。
\item
  \texttt{-v} プリビューアを指定する。
  evinceなど、pdfを表示できるアプリケーションを指定する。
  ソースファイルをtexworksで編集している場合は、ここにtexworksを指定するのが良い。
\item
  \texttt{-r} ルートファイルを指定する。
\end{itemize}

\hypertarget{ux30a4ux30f3ux30b9ux30c8ux30fcux30ebux3068ux30a2ux30f3ux30a4ux30f3ux30b9ux30c8ux30fcux30eb}{%
\subsection{インストールとアンインストール}\label{ux30a4ux30f3ux30b9ux30c8ux30fcux30ebux3068ux30a2ux30f3ux30a4ux30f3ux30b9ux30c8ux30fcux30eb}}

\hypertarget{ux30a4ux30f3ux30b9ux30c8ux30fcux30ebux306bux5fc5ux8981ux306aux74b0ux5883}{%
\paragraph{インストールに必要な環境}\label{ux30a4ux30f3ux30b9ux30c8ux30fcux30ebux306bux5fc5ux8981ux306aux74b0ux5883}}

\begin{itemize}
\item
  Linuxとbash。
  DebianとUbuntuではテストされているが、おそらくその他のlinuxディストリビューションでも動作すると思われる。
  Bashコマンドを用いてスクリプトが記述されているので、bashは必要である。
\item
  LaTeXシステム。 LaTeXのインストールには2つのオプションがある。
  1つはディストリビューションに付属のシステムをインストールすることである。
  他方はTexLiveをインストールすることである。
\item
  makeまたはrake。
  これらのツールはBuildtoolsにとって、必ずしも必要というわけではない。
  しかし、makeまたはrakeのもとで、Buildtoolsを実行することが望ましい。
  この2つに両方をインストールする必要はない。
  どちらか1つを選んでインストールすれば良い。
  makeは長く使われているビルド・ツールで、元々はCコンパイラの制御に用いられてきた。
  rakeはこれに似たツールで、rubyで書かれたアプリケーションである。
  rakeを使うことの利点は、そのスクリプトであるRakefileの中で任意のrubyコードを記述できることである。
  一般に、RakefileはMakefileよりも読みやすく、理解しやすい。
\end{itemize}

\hypertarget{ux30a4ux30f3ux30b9ux30c8ux30fcux30eb}{%
\paragraph{インストール}\label{ux30a4ux30f3ux30b9ux30c8ux30fcux30eb}}

インストール用のスクリプト\texttt{install.sh}を使う。

\begin{verbatim}
$ bash install.sh
\end{verbatim}

シェルスクリプトなどの実行ファイルは\texttt{\$HOME/bin}に保存される。
debianやubuntuでは、ログイン時に\texttt{\$HOME/bin}があれば、bashの実行ディレクトリのパスを表す環境変数\texttt{PATH}に追加される。
インストール時に新規に \texttt{\$HOME/bin}
を作成した場合には、再ログインしないと、それが実行ディレクトリに追加されないので注意が必要。
rootになってインストールすると\texttt{/usr/local/bin}に実行ファイルをインストール。
debianの場合は、

\begin{verbatim}
$ su -
# bash install.sh
\end{verbatim}

ubuntuの場合は

\begin{verbatim}
$ sudo bash install.sh
\end{verbatim}

\hypertarget{ux30a2ux30f3ux30a4ux30f3ux30b9ux30c8ux30fcux30eb}{%
\paragraph{アンインストール}\label{ux30a2ux30f3ux30a4ux30f3ux30b9ux30c8ux30fcux30eb}}

アンインストールは\texttt{uninstall.sh}で行う。
一般ユーザで実行すれば、\texttt{\$HOME}以下のインストールファイルが削除される。

\begin{verbatim}
$ bash uninstall.sh
\end{verbatim}

rootで実行すれば、\texttt{/usr/local}以下のインストールファイルが削除される。
debianの場合は、

\begin{verbatim}
$ su -
# bash uninstall.sh
\end{verbatim}

ubuntuの場合は、

\begin{verbatim}
$ sudo bash uninstall.sh
\end{verbatim}

\hypertarget{ux30e9ux30a4ux30bbux30f3ux30b9}{%
\subsection{ライセンス}\label{ux30e9ux30a4ux30bbux30f3ux30b9}}

Copyright (C) 2020 ToshioCP (関谷 敏雄)

Buildtoolsはフリーソフトウェアであり、フリーソフトウェア財団によって発行されたGNU
一般公衆利用許諾書(バージョン3またはそれ以降のバージョン)の定める条件の下で再頒布または改変することができる。

Buildtoolsは多くの人にとって有用であると考えて頒布されているものであるが、これは\emph{全くの無保証}
である。商業可能性の保証や特定の目的への適合性は、言外に示されたものも含め、全く存在しない。
詳しくは\href{https://www.gnu.org/licenses/gpl-3.0.html}{GNU GENERAL
PUBLIC LICENSE}をご覧いただきたい。
また、その参考として、八田真行氏による\href{https://gpl.mhatta.org/gpl.ja.html}{GNU
一般公衆利用許諾書の非公式日本語訳}がある。

\end{document}
\end{verbatim}
更に、8行目の{\textbackslash}tableofcontentsをアンコメントしてコマンドが利くようにしよう。
\begin{verbatim}
 ... ...
\maketitle
% If you want a table of contents here, uncomment the following
 line.
\tableofcontents
 ... ...
\end{verbatim}

2番めは、{\textbackslash}tightlistのマクロを定義することが必要である。
Helper.texがその定義を書くのに最も適した場所である。
\begin{verbatim}
 ... ...
 ... ...
\providecommand{\tightlist}{%
  \setlength{\itemsep}{0pt}\setlength{\parskip}{0pt}}
 ... ...
 ... ...
\end{verbatim}
このコードは\url{https://github.com/jgm/pandoc-templates/blob/master/default.latex}から引用したものである。

rakeを使ってコンパイルする。
\begin{verbatim}
$ rake
\end{verbatim}

\begin{center}
\includegraphics[width=12cm]{tableofcontents.png}
\end{center}

これで、目次にセクション9、そして「Readme.ja.md」の内容が表示された。

このセクションではpandocを手動で走らせた。
もしも、Readme.ja.md がアップグレードされたならば、再びpandocを実行しなければならない。
それは面倒なことであり、本来自動化されるべきことである。
ひとつの方法はRakefileを変更してrakeがコンパイル前に自動的にプリプロセッシングするようにすることである。
次のセクションでその方法を説明する。


\section{Use rake}
  Rake is a build tool similar to make.
Rakefile describes instructions for rake to build source files.
You can write any ruby commands in Rakefile.
Therefore, it has a high ability to describe the build process even if it is complicated, .

Newtex generates a Rakefile, which is enough to compile the source files if there is no preprocessing procedure.
In the previous section, we used pandoc to generate readme.tex.
So, we need to modify Rakefile to put in pandoc.
Modify the Rakefile as follows.
\begin{verbatim}
require 'rake/clean'

# if readme.tex doesn't exist, generate it first.
# This is necessary because readme.tex is accessed by gfiles in
#  line 12.
if File.exist?("readme.tex") == false
  sh "pandoc -o readme.tex ../Readme.md"
end
# use Latex-BuildTools
@tex_files = (`tfiles -a` + `tfiles -p`).split("\n")
@tex_files <<= "readme.tex"
@graphic_files = []
@tex_files.each do |file|
  @graphic_files += `gfiles #{file}`.split("\n")
end

task default: "Tutorial.pdf"

file "Tutorial.pdf" => "_build/main.pdf" do
  sh "cp _build/main.pdf Tutorial.pdf"
end

file "_build/main.pdf" => (@tex_files+@graphic_files) do
  sh "lb main.tex"
end

file "readme.tex" => "../Readme.md" do
  sh "pandoc -o readme.tex ../Readme.md"
end

CLEAN << "_build"
task :clean

task :ar do
  sh "arl main.tex"
  sh "tar -rf main.tar Rakefile"
  sh "gzip main.tar"
  sh "mv main.tar.gz Tutorial.tar.gz"
end

task :zip do
  sh "arl -z main.tex"
  sh "zip main.zip Rakefile"
  sh "mv main.zip Tutorial.zip"
end
\end{verbatim}

Thanks to this modification, you don't need to run pandoc by hand.
What you need is just type `rake'.

There are websites about ruby and rake.
For example,
\begin{itemize}
\item \url{https://www.ruby-lang.org/en/}
\item \url{http://rubylearning.com/}
\item \url{https://ruby.github.io/rake/}
\end{itemize}

The tutorial finishes at this section.
Next section is the copy of Readme.md in Buildtools source files.
It describes the background of Buildtools and features of each script.

\section{Make tarball}
  ソースファイルを配布したい、という場合もあるかもしれない。
そのようなときには、それをアーカイブすることが必要になる。
Buildtoolsに含まれるスクリプトのarlは、ルートファイルが取り込むサブファイルや画像ファイルを検索し、それらアーカイブする。
このアーカイブされたファイルのうち、tarというコマンドで作られたものをtarballという。
\begin{itemize}
\item -gオプションが与えられると、gzipで圧縮されたtarballを作る。
\item -bオプションが与えられると、bzip2で圧縮されたtarballを作る。
\item -zオプションが与えられると、zipファイルを作る。
\item オプションが与えられなければ、非圧縮のtarballを作る。
\end{itemize}

もし、プリプロセッシングで生成されるlatexソースファイルなどがある場合は、arlの実行前にそれらを生成しておかなければならない。
\begin{verbatim}
$ arl
$ tar -tf main.tar
main.tex
edit_tex_files.tex
generate_templates.tex
installation.tex
lb.tex
preprocessing.tex
rake.tex
readme.tex
tarball.tex
test_compile.tex
helper.tex
Tutorial_1.png
Tutorial_2.png
hellolatex.png
tableofcontents.png
test_installation.png
\end{verbatim}

Rakefileも当然ながら、tarballに含めなければならない。
\begin{verbatim}
$ tar -rf main.tar Rakefile
\end{verbatim}
そして、gzipなどに圧縮してtarballが完成する。
\begin{verbatim}
$ gzip main.tar
\end{verbatim}

以上の手続きは、実はすでにRakefileに記述されている。
「rake ar」とタイプすることにより、rakeがtarballを自動生成してくれる。
\begin{verbatim}
$ rm main.tar.gz
$ rake ar
arl main.tex
tar -rf main.tar Rakefile
gzip main.tar
mv main.tar.gz チュートリアル.tar.gz
\end{verbatim}
最後にできあがるtarballの名前は「チュートリアル.tar.gz」である。

もしも、zipファイルを作りたければ、「rake zip」とタイプする。

このセクションで、チュートリアルは終わりである。
次のセクションはBuildtoolsのソースファイルに含まれるReadme.ja.mdのコピーである。
この文書には、Buildtoolsを作成した背景や個々のスクリプトの特長と使い方が書かれている。


\end{document}
\end{verbatim}
The first line specifies a documentclass which is the same as the value of documentclass key in newtex.conf.
The second line has an input command which includes `helper.tex'.
Helper.tex has a role to include packages with \verb|\usepackage| command, define macros with \verb|\newcommand| command and so on.
Most of the lines in the preamble are described in helper.tex.
It is a good idea to make your own helper.tex because users often use the same preamble in different documents.
If you have your helper.tex, copy and overwrite this file.

You need to edit the fourth line.
For example,
\begin{verbatim}
\author{Toshio Sekiya}
\end{verbatim}
You can add `{\textbackslash}date', `{\textbackslash}thanks', `{\textbackslash}begin\{abstract\}' and `{\textbackslash}end\{abstract\}' if you like.
If you want to make a table of contents, then uncomment the eighth line.
After that, the lines are sections and {\textbackslash}input commands to include subfiles.

Rakefile contains instructions to rake.
You don't need to modify it so far.
Try to use it.
\begin{verbatim}
$ rake
 ... ...
 ... ...
$ ls
Makefile            gecko.png               main.tex
Rakefile            generate_templates.tex  preprocessing.tex
Tutorial.pdf        helper.tex              rake.tex
_build              installation.tex        tarball.tex
cover.tex           lb.conf                 test_compile.tex
edit_tex_files.tex  lb.tex
$ ls _build
main.aux  main.fdb_latexmk  main.fls  main.log  main.out  main.pdf
$ evince Tutorial.pdf
\end{verbatim}
Rake ran lb to compile main.tex and after that it copied \_build/main.pdf to Tutorial.pdf.
Evince shows Tutorial.pdf as follows.
\begin{center}
\includegraphics[width=8cm]{Tutorial_1.png}
\end{center}

