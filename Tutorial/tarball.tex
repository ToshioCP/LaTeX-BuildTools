You might want to distribute your source files.
Then, you need to archive them.
The script `arl' looks for the subfiles and the graphics files included by the rootfile, and archive them.
\begin{itemize}
\item If -g option is given, it makes gzip compressed tarball.
\item If -b option is given, it makes bzip2 compressed tarball.
\item If -z option is given, it makes zip file.
\item If no option is given, it makes non-compressed tarball.
\end{itemize}

If some latex source files are generated by preprocessing, you need to generate them before running arl.
\begin{verbatim}
$ arl
$ tar -tf main.tar
main.tex
edit_tex_files.tex
generate_templates.tex
installation.tex
lb.tex
preprocessing.tex
rake.tex
readme.tex
tarball.tex
test_compile.tex
helper.tex
Tutorial_1.png
Tutorial_2.png
hellolatex.png
tableofcontents.png
test_installation.png
\end{verbatim}

Rakefile needs to be added to the tarball.
\begin{verbatim}
$ tar -rf main.tar Rakefile
\end{verbatim}
Then, compress it into gzip.
\begin{verbatim}
$ gzip main.tar
\end{verbatim}

The procedure above is already written in the Rakefile.
Type `rake ar', then rake makes a tarball.
\begin{verbatim}
$ rm main.tar.gz
$ rake ar
arl main.tex
tar -rf main.tar Rakefile
gzip main.tar
mv main.tar.gz Tutorial.tar.gz
\end{verbatim}
Now the name of the tarball is `Tutorial.tar.gz'.

If you want to make a zip file, type `rake zip'.

The tutorial finishes at this section.
Next section is the copy of Readme.md in Buildtools source files.
It describes the background of Buildtools and features of each script.
