Rake is a build tool similar to make.
Rakefile describes instructions for rake to build source files.
You can write any ruby commands in Rakefile.
Therefore, it has a high ability to describe the build process even if it is complicated, .

Newtex generates a Rakefile, which is enough to compile the source files if there is no preprocessing procedure.
In the previous section, we used pandoc to generate readme.tex.
So, we need to modify Rakefile to put in pandoc.
Modify the Rakefile as follows.
\begin{verbatim}
require 'rake/clean'

# if readme.tex doesn't exist, generate it first.
# This is necessary because readme.tex is accessed by gfiles in
#  line 12.
if File.exist?("readme.tex") == false
  sh "pandoc -o readme.tex ../Readme.md"
end
# use Latex-BuildTools
@tex_files = (`tfiles -a` + `tfiles -p`).split("\n")
@tex_files <<= "readme.tex"
@graphic_files = []
@tex_files.each do |file|
  @graphic_files += `gfiles #{file}`.split("\n")
end

task default: "Tutorial.pdf"

file "Tutorial.pdf" => "_build/main.pdf" do
  sh "cp _build/main.pdf Tutorial.pdf"
end

file "_build/main.pdf" => (@tex_files+@graphic_files) do
  sh "lb main.tex"
end

file "readme.tex" => "../Readme.md" do
  sh "pandoc -o readme.tex ../Readme.md"
end

CLEAN << "_build"
task :clean

task :ar do
  sh "arl main.tex"
  sh "tar -rf main.tar Rakefile"
  sh "gzip main.tar"
  sh "mv main.tar.gz Tutorial.tar.gz"
end

task :zip do
  sh "arl -z main.tex"
  sh "zip main.zip Rakefile"
  sh "mv main.zip Tutorial.zip"
end
\end{verbatim}

Thanks to this modification, you don't need to run pandoc by hand.
What you need is just type `rake'.

There are websites about ruby and rake.
For example,
\begin{itemize}
\item \url{https://www.ruby-lang.org/en/}
\item \url{http://rubylearning.com/}
\item \url{https://ruby.github.io/rake/}
\end{itemize}

The tutorial finishes at this section.
Next section is the copy of Readme.md in Buildtools source files.
It describes the background of Buildtools and features of each script.
