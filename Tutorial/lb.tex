\subsection{First step}
Lb is the main script in Buildtools.
This section describes how to use it with a small example.

First, make a directory named \verb|example| and change your current directory to it.
\begin{verbatim}
$ makedir example
$ cd example
\end{verbatim}

Then make a tex source file in the directory.
Run your favourite editor and copy the following text, then save it as the name \verb|main.tex|.
\begin{verbatim}
\documentclass{article}
\begin{document}
Hello \LaTeX !!
\end{document}
\end{verbatim}

Then, just type \verb|lb|.
\begin{verbatim}
$ lb
\end{verbatim}
Then, it runs latexmk and pdflatex and compile \verb|main.tex| with them.
Messages appear on your screen, and that shows the process of the compilation.
If there is a line 
\begin{verbatim}
Output written on _build/main.pdf (1 page, 19263 bytes).
\end{verbatim}
then the compilation completes correctly.
Check the directory.
\begin{verbatim}
$ ls -l
total 8
drwxrwxr-x 2 user user 4096 Dec  6 11:59 _build
-rw-rw-r-- 1 user user   72 Dec  6 11:59 main.tex
\end{verbatim}
A new directory \verb|_build| is generated.
Look at the files in the directory.
\begin{verbatim}
$ cd _build
$ ls
\end{verbatim}
There are auxliary files and the target file \verb|main.pdf|.
See \verb|main.pdf| with your pdf-viewer, for example evince.
\begin{verbatim}
$ evince main.pdf
\end{verbatim}
\begin{center}
\includegraphics[width=3cm]{hellolatex.png}
\end{center}

\subsection{Use lb.conf}
In the previous subsection, lb runs pdflatex.
The reason why lb chose pdflatex is the documentclass `article'.
It can also be compiled by lualatex or xelatex, but pdflatex has been a standard latex engine for ages.

If you want to use, for example, lualatex to compile, you need to specify it in \verb|lb.conf|.
This configuration file has six items.
\begin{description}
\item[rootfile] Rootfile is the main tex file, which usually includes {\textbackslash}begin\{document\} and {\textbackslash}end\{document\}. Other tex files are called `subfile'.
\item[builddir] This is a temporary directory includes all the auxliary files and the target file, which is usually a pdf file.
\item[engine] This specifies a latex engine, which is one of pdflatex, xelatex, lualatex, latex and platex.
\item[latex\_option] This specifies options to give \verb|latexmk|. The option `-halt-on-error' is given to \verb|latexmk| even if lb.conf doesn't exist.
\item[dvipdf] This is a program which converts dvi into pdf, which is used only with latex or platex. It is unnecessary with other latex engines. `dvipdfmx' is the best at present.
\item[preview] Pdf viewer. This is used to preview the pdf file when lb is given a subfile as an argument.
\end{description}

Run your editor, type the following and save it as the name \verb|lb.conf|.
\begin{verbatim}
rootfile=main
builddir=_build
engine=lualatex
latex_option=-halt-on-error
dvipdf=
preview=evince
\end{verbatim}
Then, type
\begin{verbatim}
$ lb
\end{verbatim}
it used lualatex to compile.

If you want to change the name of the tex file to `example.tex', then modify the first line in lb.conf to
\begin{verbatim}
rootfile=example
\end{verbatim}
or
\begin{verbatim}
rootfile=example.tex
\end{verbatim}
The suffix can be left out.

In addition, if you want to put all the axiliary files and the target file in the source directory, change the second line in lb.conf to:
\begin{verbatim}
builddir=
\end{verbatim}
This specifies null string for builddir item and that means no build directory is made.

Let's try to run \verb|lb| with the following \verb|lb.conf|.
\begin{verbatim}
rootfile=example
builddir=
engine=latex
latex_option=-halt-on-error
dvipdf=dvipdfmx
preview=evince
\end{verbatim}
Now, the engine is latex and dvipdf program is dvipdfmx.
\begin{verbatim}
$ rm -r _build
$ mv main.tex example.tex
$ lb
\end{verbatim}
Then messages appear.
It includes the following line.
\begin{verbatim}
This is pdfTeX, Version 3.14159265-2.6-1.40.21 (TeX Live 2020)
 (preloaded format=latex)
  ... ...
  ... ...
Output written on example.dvi (1 page, 332 bytes).
Transcript written on example.log.
Latexmk: Examining 'example.log'
=== TeX engine is 'pdfTeX'
Latexmk: Log file says output to 'example.dvi'
Latexmk: All targets (example.dvi) are up-to-date
example.dvi -> example.pdf
[1]
3662 bytes written
\end{verbatim}
This tells us that the engine was `latex'%
\footnote{
In Texlive2020, `latex' command calls `pdftex' instead of `tex' which is the original TeX program.
}.
It generates dvi file instead of pdf file.
After that, dvipdfmx is run by latexmk and it translates the dvi file into a pdf file.
The name `dvipdfmx' doesn't appear in the message but `example.dvi -{\textgreater} example.pdf' is outputed by dvipdfmx.
So we know that dvipdfmx was run by latexmk in the build process.
\begin{verbatim}
$ ls
example.aux  example.fdb_latexmk  example.log  example.tex
example.dvi  example.fls          example.pdf  lb.conf
\end{verbatim}
There's no temprary directory like \_build because we specified null string for builddir.

One of the important feature of lb is compiling a subfile separately.
This will be explained in the section  \ref{sec:testcompile} `Test compile' (p. \pageref{sec:testcompile}).
