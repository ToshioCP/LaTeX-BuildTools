\subsection{Prerequisite}
Buildtools requires the following items.
\begin{enumerate}
\item Linux OS and bash
\item LaTeX system
\item Make or Rake
\end{enumerate}

\subsubsection{Linux OS and bash}
Buildtools is tested on Debian and Ubuntu.
However, it probably works on other linux distributions.
Bash is required because the shell scripts in Buildtools include bash commands.
\subsubsection{LaTeX system}
There are two options to install LaTeX.

One is installing the LaTeX applications provided by your distribution.
If your distribution is Ubuntu, you can install it by typing the following line.
\begin{verbatim}
$ sudo apt-get install texlive-full
\end{verbatim}
If you have another distribution, refer to the distribution's document to install.

The other way is installing TexLive from \url{https://www.tug.org/texlive}
Refer the documentations in the web to install TexLive system.

\subsubsection{Make or rake}
These applications are not necessarily required to run the tools in Buildtools.
However, it is recommended that they should be used under the control of make or rake.
You don't need to install both of them.
Choose one which you like.

Make is a traditional build tool originally aimed at C compiler.
In Ubuntu, type the following line to install make.
\begin{verbatim}
$ sudo apt-get install make
\end{verbatim}

Rake is a build tool similar to make.
It is one of the ruby application.
The advantage to use rake is that you can put any ruby codes into Rakefile, which is the script file of rake.
Generally speaking, Rakefile is easy to understand than Makefile.
In Ubuntu, type the following line to install rake.
\begin{verbatim}
$ sudo apt-get install rake
\end{verbatim}

If you want to install the latest version of ruby, use rbenv and ruby-build.
See the following github repository and refer to the documentations there.
\begin{itemize}
\item \url{https://github.com/rbenv/rbenv}
\item \url{https://github.com/rbenv/ruby-build}
\end{itemize}

\subsection{Installation}
\subsubsection{Download}
First, access the following github repository.
\begin{itemize}
\item \url{https://github.com/ToshioCP/LaTeX-BuildTools}
\end{itemize}
Click the \verb|Code| button, then popup menu appears.
Click \verb|DOWNLOAD ZIP| menu.
Unzip the download zip file.
\subsubsection{Installation}
Open your terminal.
Change your current directory into the directory you extracted the zip file. 
Then type:
\begin{verbatim}
$ bash install.sh
\end{verbatim}
This script installs the executable files into \verb|\$HOME/bin|.
Debian and Ubuntu adds the directory \verb|\$HOME/bin| into \verb|PATH| environment variable if it exists at the login time.
The script makes the directory \verb|\$HOME/bin| if it doesn't exist.
In that case, you need to re-login to put the directory into the \verb|PATH| environment variable.
This installs scripts into your private directory, so any other users can't access the scripts.
This is called user level installation or private installation.

If you want to install the scripts in \verb|/user/local/bin|, you need to have root privilege.
If your OS is ubuntu, then type:
\begin{verbatim}
$ sudo bash install.sh
\end{verbatim}

